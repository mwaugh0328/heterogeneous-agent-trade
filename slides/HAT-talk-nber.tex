\documentclass[9pt,pdftex,aspectratio=1610]{beamer}
\usepackage{amsmath, amsthm, amssymb}
\usepackage{color}
\usepackage{hyperref}
\usepackage{subfigure}
\usepackage{tabularx}
\usepackage{ragged2e}
\usepackage{booktabs}
\usepackage{multirow}
\usepackage{natbib}
\usepackage{bbm}
\usepackage[listings, most]{tcolorbox}

\usecolortheme{dolphin}
\linespread{1.3}
\definecolor{nblue}{RGB}{0,0,128}

\bibliographystyle{ecta}
\setbeamercovered{transparent}

\newcolumntype{Y}{>{\RaggedRight\arraybackslash}X}
%\setbeamerfont{alerted text}{series=\bfseries}

\hypersetup{colorlinks=true, linkcolor=nblue,
citecolor=nblue, urlcolor=nblue, bookmarks=false,
pdfpagemode=UseNone,
pdfstartview={XYZ null null 1.0},
pdftitle={Heterogeneous Agent Trade},
pdfauthor={ Michael E. Waugh},
pdfkeywords={economics, trade, dynamics, quant econ, consumption, data science,
waugh, incomplete markets, inequality, Ricardo, julia, Armington, China, trade war, tariffs, python, matplotlib}}

%\usepackage[pdftex,colorlinks=true, bookmarks=false,
%pdfstartview={XYZ null null 1.0},
%pdftitle={Heterogeneous Agent Trade},
%pdfauthor={Michael E. Waugh},
%pdfkeywords={economics, trade, dynamics, quant econ, consumption, data science,
%waugh, incomplete markets, inequality, Ricardo, julia, Armington, China, trade war, tariffs, python, matplotlib},
%colorlinks=true,linkcolor=darkgray,citecolor=darkgray,urlcolor=darkgray,
%breaklinks]{hyperref}

\setbeamertemplate{navigation symbols}{}
\setbeamertemplate{footline}[frame number]
\setbeamertemplate{theorems}[numbered]
\setbeamertemplate{itemize subitem}[circle]
\setbeamertemplate{enumerate items}[default]

\setbeamerfont{frametitle}{size= \large}
\setbeamerfont{ framesubtitle }{size = \footnotesize}
\setbeamertemplate{frametitle}
{
\medskip
\smallskip
{\textsf{\underline{\insertframetitle\phantom{))))))))}}}}}
\setbeamertemplate{items}[circle]
\setbeamertemplate{itemize subitem}[circle]

\theoremstyle{definition}


\newtheorem{as}{Assumption}
\newtheorem{df}{Definition}
\newtheorem{lm}{Lemma}
\newtheorem{prp}{Proposition}

\tcolorboxenvironment{prp}{%
boxrule = 0mm, breakable, colframe=white,
before skip=5pt,after skip=5pt,
colback=gray!5!white,
top = 2mm,
bottom = 2mm%,
%borderline north={1pt}{1pt}{gray},
%borderline south={1pt}{1pt}{gray}
}

\tcolorboxenvironment{df}{%
boxrule = 0mm, breakable, colframe=white,
before skip=5pt,after skip=5pt,
colback=gray!5!white,
top = 2mm,
bottom = 2mm%,
%borderline north={1pt}{1pt}{gray},
%borderline south={1pt}{1pt}{gray}
}


\usepackage[normalem]{ulem}
\newcommand\redout{\bgroup\markoverwith
{\textcolor{red}{\rule[.5ex]{1pt}{1pt}}}\ULon}

\makeatletter
\def\blfootnote{\xdef\@thefnmark{}\@footnotetext}
\makeatother

%%%%%%%%%%%%%%%%%%%%%%%%%%%%%%%%%%%%%%%%%%%%%%%%%%%%%%%%%%%%%%%%%%%%%%%%%%%%%%%%%%%%%%%%%%%%%%%%%
%%%%%%%%%%%%%%%%%%%%%%%%%%%%%%%%%%%%%%%%%%%%%%%%%%%%%%%%%%%%%%%%%%%%%%%%%%%%%%%%%%%%%%%%%%%%%%%%%

%%%%%%%%%%%%%%%%%%%%%%%%%%%%%%%%%%%%%%%%%%%%%%%%%%%%%%%%%%%%%%%%%%%%%%%%%%%%%%%%%%%%%%%%%%%%%%%%%
%%%%%%%%%%%%%%%%%%%%%%%%%%%%%%%%%%%%%%%%%%%%%%%%%%%%%%%%%%%%%%%%%%%%%%%%%%%%%%%%%%%%%%%%%%%%%%%%%

\title{\Large Heterogeneous Agent Trade}
\institute[Foo and Bar]{\normalsize\begin{tabular}[h]{c}
Michael E. Waugh  \\
Federal Reserve Bank of Minneapolis\blfootnote{The views expressed herein are those of the author and not necessarily those of the Federal
Reserve Bank of Minneapolis or the Federal Reserve System. This project was developed with research support from the National Science Foundation (NSF Award number 1948800). Thomas Hasenzagl and Teerat Wongrattanapiboon provided excellent research assistance.} and NBER\\
\href{https://twitter.com/tradewartracker}{@tradewartracker}
\end{tabular}}

\date{\today}

\begin{document}

\begin{frame}
\titlepage
\setcounter{framenumber}{0}
\section{}
\end{frame}

%\begin{frame}[t]{Heterogenous Price Elasticities and Trade}
%\smallskip
%To trade economists, household heterogeneity is interesting because of the notion that some benefit from trade and others don't.\\
%\medskip
%One mechanism behind this notion is heterogeneity in \begin{alert}{\textbf{elasticities}}\end{alert}.\\
%\bigskip
%This paper: A model of household heterogeneity that results in heterogenous price elasticities and I use it as a laboratory to think about aggregate trade, the gains from trade and how they are distributed.\\
%\end{frame}

\begin{frame}[t]{Heterogenous Price Elasticities and Trade}
\smallskip
Two ingredients:
\begin{itemize}
\item Trade as in Armington, but households have random utility over varieties | \citet{mcfadden1974frontiers}.
\smallskip
\item Standard incomplete markets model with households facing incomplete insurance against idiosyncratic productivity and taste shocks | \citet{bewley1979optimum}.
\end{itemize}
\bigskip
Two core results:
\begin{itemize}
\item A household's price elasticity, in essence, is about the marginal gain in utility from a percent change in consumption $\Rightarrow$ the poor are the most price sensitive.
\smallskip
\item The gains from trade | to a first order | depend on these elasticities. And can $\Rightarrow$ pro-poor gains from trade.
\end{itemize}
\bigskip
Quantitatively, these forces are powerful | the poorest households gain 4.5X more than the richest; the average gains from trade are 3X than representative agent benchmarks.
\end{frame}

%%%%%%%%%%%%%%%%%%%%%%%%%%%%%%%%%%%%%%%%%%%%%%%%%%%%%%%%%%%%%%%%%%%%%%%%%%%%%%%%%%%%%%%%%%%%%%%%%
%%%%%%%%%%%%%%%%%%%%%%%%%%%%%%%%%%%%%%%%%%%%%%%%%%%%%%%%%%%%%%%%%%%%%%%%%%%%%%%%%%%%%%%%%%%%%%%%%

\begin{frame}[t]{Outline}
\smallskip
\uncover<1-2>{
\textbf{1.} Illustrative model
\begin{itemize}
\item Static framework | intended to illustrate how everything works.
\end{itemize}}
\bigskip
\uncover<1>{
\textbf{2.} Main model
\begin{itemize}
\item Dynamic framework | production side + the standard incomplete markets model.
\end{itemize}
\bigskip
\textbf{3.} Quantitative results
\begin{itemize}
\item Calibrated to bilateral trade flows and micro evidence. Gains from trade calculations.
\end{itemize}}
\end{frame}

%%%%%%%%%%%%%%%%%%%%%%%%%%%%%%%%%%%%%%%%%%%%%%%%%%%%%%%%%%%%%%%%%%%%%%%%%%%%%%%%%%%%%%%%%%%%%%%%%
%%%%%%%%%%%%%%%%%%%%%%%%%%%%%%%%%%%%%%%%%%%%%%%%%%%%%%%%%%%%%%%%%%%%%%%%%%%%%%%%%%%%%%%%%%%%%%%%%

\begin{frame}[t]{Illustrative Model | Environment}
\smallskip
Armington model with two countries home and foreign.\\
\medskip
Continuum of agents with names $k \in [0, \ 1]$ in the home country. Agents are heterogenous in wealth $w^k$.
Agents have preferences:
\begin{align*}
u(c_{hh}^k) + \epsilon_{h}^k  \ \ \ \  \mbox{and} \ \ \ \ u(c_{hf}^k) + \epsilon_{f}^k,
\end{align*}
\begin{itemize}
\item discrete choice\ldots so each household chooses only the home or foreign variety.
\smallskip
\item $\epsilon^k_{j}$s are iid across agents and distributed Type 1 Extreme Value with parameter $\sigma_{\epsilon}$.
\smallskip
\item For now, $u$ is well behaved.
\end{itemize}
\medskip
Agents \textbf{maximize expected utility} by formulating state contingent plans of variety choice based upon all possible realizations of $\epsilon_h, \epsilon_f$.
\end{frame}

%%%%%%%%%%%%%%%%%%%%%%%%%%%%%%%%%%%%%%%%%%%%%%%%%%%%%%%%%%%%%%%%%%%%%%%%%%%%%%%%%%%%%%%%%%%%%%%%%
%%%%%%%%%%%%%%%%%%%%%%%%%%%%%%%%%%%%%%%%%%%%%%%%%%%%%%%%%%%%%%%%%%%%%%%%%%%%%%%%%%%%%%%%%%%%%%%%%

\begin{frame}[t]{Illustrative Model | Trade}
\smallskip
Micro-level trade
\begin{align*}
\pi_{hf}^k = \exp \left( \frac{ u( w^k / p_{hf} ) }{\sigma_{\epsilon}} \right) \Bigg / \Phi_{h}^k, \ \ \ \ \mbox{where} \ \ \ \Phi_{h}^k := \sum_{j\in {h,f}} \exp \left( \frac{ u( w^k / p_{hj} )  }{\sigma_{\epsilon}} \right),
\end{align*}
which is the standard result with the Type 1 EV taste shocks.\\
\bigskip
\medskip
Aggregate trade aries from the \textbf{explicit aggregation} of these decisions
\begin{align*}
M_{hf} = \int w^k \pi_{hf}^k dk.
\end{align*}\\
\bigskip
The \textbf{aggregate trade elasticity} is just a expenditure weighted average of all agent $k$s elasticities.
\begin{itemize}
\smallskip
\item Micro-elasticities\ldots next slide
\end{itemize}
\end{frame}

%%%%%%%%%%%%%%%%%%%%%%%%%%%%%%%%%%%%%%%%%%%%%%%%%%%%%%%%%%%%%%%%%%%%%%%%%%%%%%%%%%%%%%%%%%%%%%%%%
%%%%%%%%%%%%%%%%%%%%%%%%%%%%%%%%%%%%%%%%%%%%%%%%%%%%%%%%%%%%%%%%%%%%%%%%%%%%%%%%%%%%%%%%%%%%%%%%%

\begin{frame}[t]{Illustrative Model | Heterogeneous Micro-Level Trade Elasticities}
\smallskip
Agent $k$'s trade elasticity
\begin{align*}
-\theta_{hf,f}^k = \frac{1}{\sigma_{\epsilon}} \times \bigg[ u'( c_{hf}^k )c_{hf}^k \bigg]
\end{align*}\\
\bigskip
\textbf{Lesson \#1}: This model easily leads to heterogeneous price sensitivity and specifically the idea that \begin{alert}{\textbf{the poor are the most price sensitive}}\end{alert}.
\begin{itemize}
\smallskip
\item Generally depends on how curved $u$ is \ldots with CRRA and curvature parameter $>$ 1 $\Rightarrow$ elasticities are larger for the poor.
\smallskip
\item Intuition: A price reduction delivers a lot of extra utility for high marginal utility ( poor ) households and this induces strong substitution by the poor.
\end{itemize}
\bigskip
A second interesting case is when $u$ is $\log$ \ldots elasticities are constant and this (\emph{and} the Type 1 EV taste shocks) leads to the {\small \citet{anderson1987ces}} CES aggregation result.
\end{frame}

%%%%%%%%%%%%%%%%%%%%%%%%%%%%%%%%%%%%%%%%%%%%%%%%%%%%%%%%%%%%%%%%%%%%%%%%%%%%%%%%%%%%%%%%%%%%%%%%
%%%%%%%%%%%%%%%%%%%%%%%%%%%%%%%%%%%%%%%%%%%%%%%%%%%%%%%%%%%%%%%%%%%%%%%%%%%%%%%%%%%%%%%%%%%%%%%%


\begin{frame}[t]{Example: Micro-Level Trade Elasticities by Income and Wealth}
\vspace{-.5cm}
\begin{figure}[t]
\centerline{
\includegraphics[scale = 0.4]{../notes/figures/micro-elasticity.pdf}}
\end{figure}
\end{frame}

%%%%%%%%%%%%%%%%%%%%%%%%%%%%%%%%%%%%%%%%%%%%%%%%%%%%%%%%%%%%%%%%%%%%%%%%%%%%%%%%%%%%%%%%%%%%%%%%%
%%%%%%%%%%%%%%%%%%%%%%%%%%%%%%%%%%%%%%%%%%%%%%%%%%%%%%%%%%%%%%%%%%%%%%%%%%%%%%%%%%%%%%%%%%%%%%%%%

\begin{frame}[t]{Illustrative Model | The Gains from Trade}
\smallskip
\textbf{What are the gains from a reduction in the price of the foreign good?}\\
\bigskip
In equivalent variation units, the first order gains from trade are\ldots
\begin{align*}
\mathtt{EV^k} \approx -\pi_{hf}^k \times \bigg [ \frac{ \theta_{hf,f}^k }{ \sum\limits_{j} \pi_{hj}^k \times \theta_{hj,j}^k}  \bigg] \times \Delta\log{p}_{hf}
\end{align*}
The gains are about (i) exposure ( {\small \citet{deaton1989rice}, \citet{borusyak2021distributional} }) and (ii) elasticities ({\small \citet*{auer2022unequal}}).\\
\bigskip
\textbf{Lesson \#2}: In this model, elasticities matter for the gains from trade and how they are distributed.
\begin{itemize}
\smallskip
\item The standard result is that elasticities don't matter to a first order.
\smallskip
\item \begin{alert}{\textbf{Here they do}}\end{alert}. And under certain conditions, these gains are pro-poor.
\end{itemize}
\medskip
Why? Next slide\ldots
\end{frame}

\begin{frame}[t]{Illustrative Model | Market Incompleteness Matters}
\smallskip
\textbf{Why do elasticities matter?}\\
\bigskip
The same equivalent variation formula can be rewritten as
\begin{align*}
\mathtt{EV^k} \approx -\pi_{hf}^k \times \bigg [ \frac{ u'(c_{hf}^k)/p_{hf}}{ \sum\limits_{j} \pi_{hj}^k \times  u'(c_{hj}^k) /p_{hj}} \bigg] \times \Delta\log{p}_{hf}
\end{align*}\\
\only<1>{\textbf{Lesson \#3}: Elasticities matter because \begin{alert}{\textbf{markets are incomplete}}\end{alert}. Agents lack markets to insure against the desire to consume a good. An efficient allocation / complete markets yields
\begin{align*}
u'( c_{hj}^k )/ p_{hj}  = u'( c_{hj'}^k )/ p_{hj'}  \ \ \ \forall\: j, \ j'
\end{align*}
and the term in brackets disappears.\\
\medskip
\textbf{Intuition:} If the price reduction occurs on the high price good, this helps equate marginal utility across goods $\Rightarrow$ an additional, first order gain from trade.\\
\smallskip
And the gains can be pro-poor when the poor are more ``misallocated.'' \\
}
%\only<2>{\textbf{Lesson \#3}: Elasticities matter because \begin{alert}{\textbf{markets are incomplete}}\end{alert}. Agents lack markets to insure against the desire to consume a good. An efficient allocation / complete markets yields
%\begin{align*}
%u'( c_{hj}^k )/ p_{hj}  = u'( c_{hj'}^k )/ p_{hj'}  \ \ \ \forall\: j, \ j'
%\end{align*}
%and the term in brackets disappears.\\
%\medskip
%\textbf{Special Case:} With $\log$ preferences, the term in brackets disappears even under incomplete markets!   \\
%}
\only<2>{\textbf{Lesson \#3}: Elasticities matter because \begin{alert}{\textbf{markets are incomplete}}\end{alert}. Agents lack markets to insure against the desire to consume a good. An efficient allocation / complete markets yields
\begin{align*}
u'( c_{hj}^k )/ p_{hj}  = u'( c_{hj'}^k )/ p_{hj'}  \ \ \ \forall\: j, \ j'
\end{align*}
and the term in brackets disappears.  \\
\medskip
Connections and implications:
\begin{itemize}
\item \textbf{Theory:} Relates to the spatial analysis of {\small \citet*{miyauchi2024unpacking}} and complete markets decentralization of discrete choice models in {\small \citet{mongey-waugh-2}}.
\smallskip
\item \textbf{Measurement:} Assumptions when micro-level shares (the $\pi^k$s ) are constructed?
\end{itemize}
}
\end{frame}

%%%%%%%%%%%%%%%%%%%%%%%%%%%%%%%%%%%%%%%%%%%%%%%%%%%%%%%%%%%%%%%%%%%%%%%%%%%%%%%%%%%%%%%%%%%%%%%%%
%%%%%%%%%%%%%%%%%%%%%%%%%%%%%%%%%%%%%%%%%%%%%%%%%%%%%%%%%%%%%%%%%%%%%%%%%%%%%%%%%%%%%%%%%%%%%%%%%

\begin{frame}[t]{Outline}
\smallskip
\uncover<1>{
\textbf{1.} Illustrative model
\begin{itemize}
\item Static framework | intended to illustrate how everything works.
\end{itemize}}
\bigskip
\uncover<1-2>{
\textbf{2.} Main model
\begin{itemize}
\item Dynamic framework | production side + the standard incomplete markets model.
\end{itemize}}
\bigskip\uncover<1>{
\textbf{3.} Quantitative results
\begin{itemize}
\item Calibrated to bilateral trade flows and micro evidence. Gains from trade calculations.
\end{itemize}}
\end{frame}

%%%%%%%%%%%%%%%%%%%%%%%%%%%%%%%%%%%%%%%%%%%%%%%%%%%%%%%%%%%%%%%%%%%%%%%%%%%%%%%%%%%%%%%%%%%%%%%%%
%%%%%%%%%%%%%%%%%%%%%%%%%%%%%%%%%%%%%%%%%%%%%%%%%%%%%%%%%%%%%%%%%%%%%%%%%%%%%%%%%%%%%%%%%%%%%%%%%

\begin{frame}[t]{Motivating the Main Model}
\smallskip
Two enrichments:\\
\medskip
\textbf{1.} Production side to connect prices with trade flows as in quantitative trade models.\\
\bigskip
\textbf{2.} The standard incomplete markets model on the household side. This is important because\ldots
\begin{itemize}
\smallskip
\item In the illustrative model, there was no ability to insure. The SIM model provides an \textbf{intermediate setting} with self-insurance, and thus works against the elasticity effects.
\smallskip
\item In the illustrative model, types are fixed. In the SIM model, household-level dynamics $\Rightarrow$ poor people today are rich in the future, and this also works against the elasticity effects.
\medskip
\item Less important reasons (i) I like it (ii) I started there (iii) provides an interface with large body of research in macro (iv) a theory as to why there are rich and poor people.
\end{itemize}
\end{frame}

%%%%%%%%%%%%%%%%%%%%%%%%%%%%%%%%%%%%%%%%%%%%%%%%%%%%%%%%%%%%%%%%%%%%%%%%%%%%%%%%%%%%%%%%%%%%%%%%%
%%%%%%%%%%%%%%%%%%%%%%%%%%%%%%%%%%%%%%%%%%%%%%%%%%%%%%%%%%%%%%%%%%%%%%%%%%%%%%%%%%%%%%%%%%%%%%%%%

\begin{frame}[t]{Main Model: Overview}
$M$ countries. Each country produces a nationally differentiated product as in Armington; competitive firms with linear technologies in labor and trade costs $d_{ij}$.\\
\bigskip
Continuum of households $k \in [0, \ L_i]$ in each country $i$. Household preferences:
\begin{align*}
\mathrm{E}\sum_{t = 0}^{\infty} \beta^{t} \ \tilde{u}^k_{ijt},
\end{align*}
where conditional direct utility for good $j$ is
\begin{align*}
\tilde{u}^k_{ijt} =  u(c^k_{ijt}) + \epsilon^k_{jt}, \ \ \ j = 1, \ldots, M.
\end{align*}\\
\begin{itemize}
\item  discrete-continuous choice\ldots so first chose one variety, then continuous choice over quantity.
\smallskip
\item $\epsilon^k_{jt}$s are iid across hh and time; distributed Type 1 Extreme Value with parameter $\sigma_{\epsilon}$.
\end{itemize}
\bigskip
Household $k$'s efficiency units $z_t$ evolve according to a Markov Chain, and can borrow or accumulate a non-state contingent asset, $a$, with gross return $R_{i}$.
\end{frame}

%%%%%%%%%%%%%%%%%%%%%%%%%%%%%%%%%%%%%%%%%%%%%%%%%%%%%%%%%%%%%%%%%%%%%%%%%%%%%%%%%%%%%%%%%%%%%%%%
%%%%%%%%%%%%%%%%%%%%%%%%%%%%%%%%%%%%%%%%%%%%%%%%%%%%%%%%%%%%%%%%%%%%%%%%%%%%%%%%%%%%%%%%%%%%%%%%

\begin{frame}[t]{The HA Trade Elasticity}
\textbf{How do imports change relative to domestic consumption due to a permanent change in $d_{ij}$?}
\smallskip
\uncover<2->{{\small
\begin{prp}[{\normalsize The HA Trade Elasticity} ] \label{prp:GET} The trade elasticity between country $i$ and country $j$ is:
\begin{align}
\theta_{ij} = 1 + \int_{z,a} \bigg \{ \theta_{ij}(a,z)^{I} + \theta_{ij}(a,z)^{E} \bigg \}\omega_{ij}(a,z)da \ dz - \int_{z,a} \bigg \{ \theta_{ii,j}(a,z)^{I} + \theta_{ii,j}(a,z)^{E} \bigg \}\omega_{ii}(a,z)da \ dz \nonumber
\end{align}
which is an expenditure-weighted average of micro-level elasticities. The micro-level elasticities are decomposed into an intensive margin and extensive margin
{\footnotesize
\begin{align}
\nonumber
\begin{alert}<3>{\theta_{ij}(a,z)^{I} = \frac{\partial c_{i}(a,z,j)/ c_{i}(a,z,j)}{\partial d_{ij} / d_{ij}}}\end{alert}, \ \ \ \ \ \ \begin{alert}<4->{\theta_{ij}(a,z)^{E} = \frac{\partial \pi_{ij}(a,z) / \pi_{ij}(a,z)}{\partial d_{ij} / d_{ij}}}\end{alert}, \ \ \ \
\end{align}
}
and $\omega_{ij}(a,z)$ are the expenditure weights.
\end{prp}
}}
\only<3>{
\medskip
{\small \begin{align*}
\theta_{ij}(a,z)^{I} = \bigg [-\frac{\partial a'_{i}(a,z,j)/ p_{ij}c_{i}(a,z,j)}{\partial p_{ij}/ p_{ij}} - 1 \bigg ]\frac{\partial p_{ij}/p_{ij}}{\partial d_{ij}/ d_{ij}}.
\end{align*}}\\
\bigskip
The idea: a lower $d_{ij}$ relaxes the bc and then the division of new resources between assets and expenditure determines the intensive margin. \begin{alert}<3>{\textbf{This is larger for the poor, smaller for the rich.}}\end{alert}\\
}
\only<4>{
\medskip
{\small
\begin{align*}
\theta_{ij}(a,z)^{E} = \frac{1}{\sigma_{\epsilon}}\frac{\partial v_{i}(a,z,j)}{\partial d_{ij}/d_{ij}}  - \frac{\partial \Phi_{i}(a,z) / \Phi_{i}(a,z)}{\partial d_{ij}/d_{ij}} .
\end{align*}}\\
\bigskip
Now assume the number of countries is large\ldots\\
}
\only<5>{
\medskip
{\small
\begin{align*}
\theta_{ij}(a,z)^{E} \approx -\frac{1}{\sigma_{\epsilon}}\bigg[u'(c_{i}(a,z,j))c_{i}(a,z,j)\bigg] .
\end{align*}}\\
\bigskip
Exactly the same as in the illustrative model. With CRRA and relative risk aversion $ > 1$ then \begin{alert}<5>{\textbf{poor hh's are the most price sensitive on the extensive margin.}}\end{alert}
}
\end{frame}

%%%%%%%%%%%%%%%%%%%%%%%%%%%%%%%%%%%%%%%%%%%%%%%%%%%%%%%%%%%%%%%%%%%%%%%%%%%%%%%%%%%%%%%%%%%%%%%%
%%%%%%%%%%%%%%%%%%%%%%%%%%%%%%%%%%%%%%%%%%%%%%%%%%%%%%%%%%%%%%%%%%%%%%%%%%%%%%%%%%%%%%%%%%%%%%%%

\begin{frame}[t]{HA Gains from Trade}
\textbf{How does utility change under the heuristic of an immediate jump to the new steady state?}
\smallskip
\uncover<2->{
{\small
\begin{prp}[{\normalsize HA Gains from Trade} ] \label{prp:gains-trade} Household level gains are given by
\begin{align}
\nonumber
\frac{\mathrm{d} v_i(a, z)}{\mathrm{d} d_{ij} / d_{ij}} = \mathbb{E}_{z} \sum_{t = 0}^{\infty} \beta^{t} \bigg \{ \begin{alert}<2>{ A(a_{t},z_{t}) }\end{alert} + \begin{alert}<3>{B(a_{t},z_{t})}\end{alert} + \begin{alert}<4>{C(a_{t},z_{t})}\end{alert} \bigg \}.
\end{align}
\end{prp}}
}
\only<2>{
\bigskip
This term is what I call the ``gains from substitution'':
{\small \begin{align*}
A(a,z) = -\sigma_{\epsilon} \frac{\mathrm{d} \pi_{ii}(a,z) / \pi_{ii}(a,z)}{\mathrm{d}d_{ij} / d_{ij}}  \\
\\
\approx \sigma_{\epsilon} \times \pi_{ij}(a,z) \times \bar{\theta}(a,z) ^E_{ij,j}
\end{align*}
}\\
\medskip
Same idea as in illustrative model | the last line says these gains are about (i) exposure and (ii) elasticities.
}
\only<3>{
\bigskip
This term is what I call ``valuation effects'':
{\small \begin{align*}
 B(a,z) = u'(c_{i}(a,z,i)) \times  a \times \frac{\mathrm{d} R_{i} / w_{i}}{\mathrm{d} d_{ij} / d_{ij}}
\end{align*}
}\\
\bigskip
How hh's real wealth ($+$ or $-$) change through GE effects on prices | all evaluated at the hh's marginal utility of home consumption.
}
\only<4>{
\bigskip
This term is what I call the ``gains from changes in asset holdings''
\vspace{.15cm}
{\small
\begin{align*}
C(a,z) = \bigg \{\underbrace{ - \frac{u'(c_{i}(a,z,i))}{p_{ii}} + \beta \mathbb{E}_{z'} \bigg [-\sigma_{\epsilon} \frac{\partial \pi_{ii}(a',z') / \pi_{ii}(a',z')}{\partial a'} + \frac{u'(c_{i}(a',z',i))R_{i}}{p_{ii}} \bigg ] }_{\mbox{Euler equation}} \  \bigg \}\frac{\mathrm{d} g_{i}(a',z',i)}{\mathrm{d} d_{ij} / d_{ij}}
\end{align*}
}
\\
\medskip
which is zero for small changes as hh's are either (i) on their Euler equation or (ii) constrained and can't adjust their asset position.
}
\end{frame}


\begin{frame}[t]{HA Gains from Trade: $\log$ Preferences $\Rightarrow$ Separation of Trade and Heterogeneity}
\vspace{-.25cm}
\smallskip
{\small
\begin{prp}[{\normalsize Separation of Trade and Micro-Heterogeneity} ] In the heterogenous agent trade model where preferences are logarithmic over the physical commodity, the trade elasticity is
\begin{align}
\theta = -\frac{1}{\sigma_{\epsilon}}, \nonumber
\end{align}
and trade flows satisfy a standard gravity relationship
\begin{align}
\frac{M_{ij}}{M_{ii}} = \left( \frac{  w_{j} / A_{j} }{  w_{i} / A_{i} } \right)^{\frac{-1}{\sigma_{\epsilon}}} d_{ij}^{\frac{-1}{\sigma_{\epsilon}}}, \nonumber
\end{align}
and both are independent of the household heterogeneity. \uncover<2->{And the welfare gains from trade for an individual household are
\begin{align}
\nonumber
\frac{\mathrm{d} v_i(a, z)}{\mathrm{d} d_{ij} / d_{ij}} = \frac{1}{\theta (1-\beta)} \times \frac{\mathrm{d} \pi_{ii} / \pi_{ii}}{\mathrm{d}d_{ij} / d_{ij}} \ \ + \ \
\mathbb{E}_{z} \sum_{t = 0}^{\infty} \beta^{t} \bigg \{ B(a_{t},z_{t}) + C(a_{t},z_{t}) \bigg \}.
\end{align}
}
\end{prp}
}
\medskip
\only<1>{This mimics the results of \citet*{anderson1987ces}. This was not obvious to me given the environment \ldots risk, market incompleteness, borrowing constraints, etc.}
\only<2>{And we are back to \citet{arkolakis2012new} $+$ what's going on with factor prices and borrowing constraints.}
\end{frame}

%%%%%%%%%%%%%%%%%%%%%%%%%%%%%%%%%%%%%%%%%%%%%%%%%%%%%%%%%%%%%%%%%%%%%%%%%%%%%%%%%%%%%%%%%%%%%%%%
%%%%%%%%%%%%%%%%%%%%%%%%%%%%%%%%%%%%%%%%%%%%%%%%%%%%%%%%%%%%%%%%%%%%%%%%%%%%%%%%%%%%%%%%%%%%%%%%

%%%%%%%%%%%%%%%%%%%%%%%%%%%%%%%%%%%%%%%%%%%%%%%%%%%%%%%%%%%%%%%%%%%%%%%%%%%%%%%%%%%%%%%%%%%%%%%%%
%%%%%%%%%%%%%%%%%%%%%%%%%%%%%%%%%%%%%%%%%%%%%%%%%%%%%%%%%%%%%%%%%%%%%%%%%%%%%%%%%%%%%%%%%%%%%%%%%

\begin{frame}[t]{Outline}
\smallskip\uncover<1>{
\textbf{1.} Illustrative model
\begin{itemize}
\item Static framework | intended to illustrate how everything works.
\end{itemize}
\bigskip
\textbf{2.} Main model
\begin{itemize}
\item Dynamic framework | production side + the standard incomplete markets model.
\end{itemize}}
\bigskip
\uncover<1-2>{
\textbf{3.} Quantitative results
\begin{itemize}
\item Calibrated to bilateral trade flows and micro evidence. Gains from trade calculations.
\end{itemize}}
\end{frame}

%%%%%%%%%%%%%%%%%%%%%%%%%%%%%%%%%%%%%%%%%%%%%%%%%%%%%%%%%%%%%%%%%%%%%%%%%%%%%%%%%%%%%%%%%%%%%%%%
%%%%%%%%%%%%%%%%%%%%%%%%%%%%%%%%%%%%%%%%%%%%%%%%%%%%%%%%%%%%%%%%%%%%%%%%%%%%%%%%%%%%%%%%%%%%%%%%

\begin{frame}[t]{Quantitative Analysis}
\smallskip
\smallskip
This is what I'll do\ldots\\
\bigskip
\textbf{1.} Calibrate my model using my ``gravity as a guide'' approach on the 19 country data set of \citet{eaton2002technology} and targeting micro-evidence from \citet{borusyak2021distributional} and \citet{auer2022unequal}. \\
\bigskip
\textbf{2.} Gains from trade calculations.\\
\end{frame}

%%%%%%%%%%%%%%%%%%%%%%%%%%%%%%%%%%%%%%%%%%%%%%%%%%%%%%%%%%%%%%%%%%%%%%%%%%%%%%%%%%%%%%%%%%%%%%%%
%%%%%%%%%%%%%%%%%%%%%%%%%%%%%%%%%%%%%%%%%%%%%%%%%%%%%%%%%%%%%%%%%%%%%%%%%%%%%%%%%%%%%%%%%%%%%%%%


\begin{frame}[t]{Bilateral Trade: Model vs. Data}
\begin{figure}[!t]
\hspace{-.5cm}\centering{\includegraphics[scale = .43]{../notes/figures/trade-fit.pdf}}
\end{figure}
\end{frame}

%%%%%%%%%%%%%%%%%%%%%%%%%%%%%%%%%%%%%%%%%%%%%%%%%%%%%%%%%%%%%%%%%%%%%%%%%%%%%%%%%%%%%%%%%%%%%%%%
%%%%%%%%%%%%%%%%%%%%%%%%%%%%%%%%%%%%%%%%%%%%%%%%%%%%%%%%%%%%%%%%%%%%%%%%%%%%%%%%%%%%%%%%%%%%%%%%

\begin{frame}[t]{US Trade Elasticities: $-\theta_{us,j}$}
\begin{figure}[!t]
\hspace{-.5cm}\centering{\includegraphics[scale = .43]{../notes/figures/us-elasticity.pdf}}
\end{figure}
\end{frame}

%%%%%%%%%%%%%%%%%%%%%%%%%%%%%%%%%%%%%%%%%%%%%%%%%%%%%%%%%%%%%%%%%%%%%%%%%%%%%%%%%%%%%%%%%%%%%%%%
%%%%%%%%%%%%%%%%%%%%%%%%%%%%%%%%%%%%%%%%%%%%%%%%%%%%%%%%%%%%%%%%%%%%%%%%%%%%%%%%%%%%%%%%%%%%%%%%
\begin{frame}[t]
\frametitle{Micro Moments | Model Consistent with HH-Level Elasticities}
\begin{figure}[!t]
\centering{\includegraphics[scale = .45]{../notes/figures/elasticity-micro.pdf}}
\end{figure}
\begin{itemize}
\smallskip
\item Household-level elasticities consistent with those in \citet*{auer2022unequal}, i.e. rich less elastic than the poor.
\end{itemize}
\end{frame}

\begin{frame}[t]
\frametitle{Micro Moments | Model Consistent with HH-Level Expenditure Patterns}
\begin{figure}[!t]
\centering{\includegraphics[scale = .45]{../notes/figures/expenditure-share.pdf}}
\end{figure}
\begin{itemize}
\smallskip
\item Household-level import shares consistent with facts from  \citet{borusyak2021distributional}, i.e. rich and poor do not spend unequally on imports.
\end{itemize}
\end{frame}

%%%%%%%%%%%%%%%%%%%%%%%%%%%%%%%%%%%%%%%%%%%%%%%%%%%%%%%%%%%%%%%%%%%%%%%%%%%%%%%%%%%%%%%%%%%%%%%%
%%%%%%%%%%%%%%%%%%%%%%%%%%%%%%%%%%%%%%%%%%%%%%%%%%%%%%%%%%%%%%%%%%%%%%%%%%%%%%%%%%%%%%%%%%%%%%%

\begin{frame}[t]{U.S. Welfare: 10\% Reduction in $d_{us,j}$}
\begin{figure}[!t]
\centering{\includegraphics[scale = .55]{../notes/figures/ge-welfare-household-vs-log.pdf}}
\end{figure}
\end{frame}


%%%%%%%%%%%%%%%%%%%%%%%%%%%%%%%%%%%%%%%%%%%%%%%%%%%%%%%%%%%%%%%%%%%%%%%%%%%%%%%%%%%%%%%%%%%%%%%%
%%%%%%%%%%%%%%%%%%%%%%%%%%%%%%%%%%%%%%%%%%%%%%%%%%%%%%%%%%%%%%%%%%%%%%%%%%%%%%%%%%%%%%%%%%%%%%%%

\begin{frame}[t]
\frametitle{Final Thoughts\ldots}
\smallskip
This paper has prompted even more questions\ldots
\begin{itemize}
\smallskip
\item The efficient pattern of trade? In a companion paper, I show that near-shoring is an outcome that a global planner likes.
\smallskip
\item Can trade policy improve outcomes? Put in tariffs and redistribute.
\smallskip
\item The interaction between trade goods and trade in assets?
\end{itemize}
\medskip
\bigskip
One more thing: My github repository provides the code and supplementary work behind this paper at \url{https://github.com/mwaugh0328/heterogeneous-agent-trade}.

\end{frame}

%%%%%%%%%%%%%%%%%%%%%%%%%%%%%%%%%%%%%%%%%%%%%%%%%%%%%%%%%%%%%%%%%%%%%%%%%%%%%%%%%%%%%%%%%%%%%%%%
%%%%%%%%%%%%%%%%%%%%%%%%%%%%%%%%%%%%%%%%%%%%%%%%%%%%%%%%%%%%%%%%%%%%%%%%%%%%%%%%%%%%%%%%%%%%%%%%

%%%%%%%%%%%%%%%%%%%%%%%%%%%%%%%%%%%%%%%%%%%%%%%%%%%%%%%%%%%%%%%%%%%%%%%%%%%%%%%%%%%%%%%%%%%%%%%%
%%%%%%%%%%%%%%%%%%%%%%%%%%%%%%%%%%%%%%%%%%%%%%%%%%%%%%%%%%%%%%%%%%%%%%%%%%%%%%%%%%%%%%%%%%%%%%%%

\appendix

\newcounter{finalframe}
\setcounter{finalframe}{\value{framenumber}}

\begin{frame}[allowframebreaks]
\frametitle{References}
\scriptsize
\bibliography{../notes/bibtex/micro_price_bibtex}
\end{frame}


\begin{frame}[t]{Log Model | Fit of Trade Data}
\begin{figure}[!t]
\centering{\includegraphics[scale = .43]{../notes/figures/log-trade-fit.pdf}}
\end{figure}
\end{frame}

%%%%%%%%%%%%%%%%%%%%%%%%%%%%%%%%%%%%%%%%%%%%%%%%%%%%%%%%%%%%%%%%%%%%%%%%%%%%%%%%%%%%%%%%%%%%%%%%%
%%%%%%%%%%%%%%%%%%%%%%%%%%%%%%%%%%%%%%%%%%%%%%%%%%%%%%%%%%%%%%%%%%%%%%%%%%%%%%%%%%%%%%%%%%%%%%%%%

\begin{frame}[t]{Model: Production and Trade}
\smallskip
$M$ countries. Each country produces a nationally differentiated product as in Armington.\\
\bigskip
\medskip
In country $i$, competitive firms' produce variety $i$ with:
\begin{align*}
Q_i = A_i N_i,
\end{align*}
where $A_i$ is TFP; $N_i$ are efficiency units of labor supplied by households.\\
\bigskip
\medskip
Cross-country trade faces obstacles:
\begin{itemize}
\smallskip
\item iceberg trade costs $d_{ij} > 1$ for one unit from supplier $j$ to go to buyer $i$.
\end{itemize}
\bigskip
\medskip
This structure leads to the following prices that households face
\begin{align*}
p_{ij} = \frac{d_{ij}w_{j}}{A_{j}}.
\end{align*}
\end{frame}

%%%%%%%%%%%%%%%%%%%%%%%%%%%%%%%%%%%%%%%%%%%%%%%%%%%%%%%%%%%%%%%%%%%%%%%%%%%%%%%%%%%%%%%%%%%%%%%%%
%%%%%%%%%%%%%%%%%%%%%%%%%%%%%%%%%%%%%%%%%%%%%%%%%%%%%%%%%%%%%%%%%%%%%%%%%%%%%%%%%%%%%%%%%%%%%%%%%

\begin{frame}[t]{Model: Households I}
\smallskip
Continuum of households $k \in [0, \ L_i]$ in each country $i$. Household preferences:
\begin{align*}
\mathrm{E}\sum_{t = 0}^{\infty} \beta^{t} \ \tilde{u}^k_{ijt},
\end{align*}
where conditional direct utility for good $j$ is
\begin{align*}
\tilde{u}^k_{ijt} =  u(c^k_{ijt}) + \epsilon^k_{jt}, \ \ \ j = 1, \ldots, M.
\end{align*}\\
\medskip
Assumptions:
\begin{itemize}
\item discrete-continuous choice\ldots so first chose one variety, then continuous choice over quantity.
\smallskip
\item $\epsilon^k_{jt}$s are iid across hh and time; distributed Type 1 Extreme Value with dispersion parameter $\sigma_{\epsilon}$.
\smallskip
\item For now, $u$ is well behaved.
\end{itemize}
\end{frame}

%%%%%%%%%%%%%%%%%%%%%%%%%%%%%%%%%%%%%%%%%%%%%%%%%%%%%%%%%%%%%%%%%%%%%%%%%%%%%%%%%%%%%%%%%%%%%%%%%
%%%%%%%%%%%%%%%%%%%%%%%%%%%%%%%%%%%%%%%%%%%%%%%%%%%%%%%%%%%%%%%%%%%%%%%%%%%%%%%%%%%%%%%%%%%%%%%%%

\begin{frame}[t]{Model: Households II}
\smallskip
Household $k$'s efficiency units $z_t$ evolve according to a Markov Chain. They face the wage per efficiency unit $w_{it}$.\\
\bigskip
\medskip
Households borrow or accumulate a non-state contingent asset, $a$, with gross return $R_{i}$. Household's face the debt limit
\begin{align*}
a^k_{t+1} \geq - \phi_{i}.
\end{align*}\\
\bigskip
\medskip
Conditional on a variety choice, a household's budget constraint is
\begin{align*}
p_{ij}c^k_{ijt} +  a^k_{t+1} \leq    R_{i} a^k_{t} + w_{it} z^k_{t}.
\end{align*}
\end{frame}


%%%%%%%%%%%%%%%%%%%%%%%%%%%%%%%%%%%%%%%%%%%%%%%%%%%%%%%%%%%%%%%%%%%%%%%%%%%%%%%%%%%%%%%%%%%%%%%%%
%%%%%%%%%%%%%%%%%%%%%%%%%%%%%%%%%%%%%%%%%%%%%%%%%%%%%%%%%%%%%%%%%%%%%%%%%%%%%%%%%%%%%%%%%%%%%%%%%

\begin{frame}[t]{What Households Do I}
Focus on a stationary setting. A hh's state are its asset holdings $a$ and shock $z$.\\
\smallskip
\textbf{1.} Condition on variety choice their problem is:
\begin{align}
v_{i}(a, z, j) =   \max_{\ a', \ c_{ij}  \ }\bigg  \{ u(c_{ij})  + \beta \, \mathbb{E} [v_{i}(a', z')]  \bigg\}, \nonumber \\
\nonumber \\
\mbox{subject to}  \ \ \  \  p_{ij}c_{ij} +  a' \leq    R_{i} a + w_{i} z \ \ \  \mbox{and} \ \ \ \ a' \geq - \phi_{i}. \nonumber
\end{align}\\
\bigskip
\textbf{2.} The ex-post value function of a household in country $i$ is
\begin{align}
\max_{j} \big  \{ \  v_{i}(a, z, j) + \epsilon_{j}  \ \big \}. \nonumber
\end{align}
\end{frame}

%%%%%%%%%%%%%%%%%%%%%%%%%%%%%%%%%%%%%%%%%%%%%%%%%%%%%%%%%%%%%%%%%%%%%%%%%%%%%%%%%%%%%%%%%%%%%%%%%
%%%%%%%%%%%%%%%%%%%%%%%%%%%%%%%%%%%%%%%%%%%%%%%%%%%%%%%%%%%%%%%%%%%%%%%%%%%%%%%%%%%%%%%%%%%%%%%%%

\begin{frame}[t]{What Households Do II}
Three equations characterizing the commodity choice, value functions, consumption / savings\ldots\\
\smallskip
\textbf{1.} The choice probability is:
\begin{align*}
\pi_{ij}(a, z) = \exp \left( \frac{ v_{i}(a, z, j) }{\sigma_{\epsilon}} \right) \Bigg / \Phi_{i}(a,z) ,\\
\\
\mbox{where} \ \ \Phi_{i}(a,z) := \sum_{j'} \exp \left( \frac{ v_{i}(a, z, j') }{\sigma_{\epsilon}}  \right).
\end{align*}\\
\bigskip
\textbf{2.} The ex-ante value function of a household in country $i$ is
\begin{align*}
v_i(a, z) = \sigma_{\epsilon} \log \left\{ \Phi_{i}(a,z)  \right\}.
\end{align*}\\
\bigskip
\textbf{3.} Away from the constraint, consumption and asset choices must respect this Euler equation:
\begin{align*}
\frac{u'(c_{i}(a, z, j))}{p_{ij}} = \beta R_{i} \mathrm{E}_{z'} \left[ \sum_{j'} \pi_{ij'}(a', z') \frac{u'(c_{i}(a', z', j'))}{p_{ij'}} \right].
\end{align*}
\end{frame}
%%%%%%%%%%%%%%%%%%%%%%%%%%%%%%%%%%%%%%%%%%%%%%%%%%%%%%%%%%%%%%%%%%%%%%%%%%%%%%%%%%%%%%%%%%%%%%%%%
%%%%%%%%%%%%%%%%%%%%%%%%%%%%%%%%%%%%%%%%%%%%%%%%%%%%%%%%%%%%%%%%%%%%%%%%%%%%%%%%%%%%%%%%%%%%%%%%%

%%%%%%%%%%%%%%%%%%%%%%%%%%%%%%%%%%%%%%%%%%%%%%%%%%%%%%%%%%%%%%%%%%%%%%%%%%%%%%%%%%%%%%%%%%%%%%%%%
%%%%%%%%%%%%%%%%%%%%%%%%%%%%%%%%%%%%%%%%%%%%%%%%%%%%%%%%%%%%%%%%%%%%%%%%%%%%%%%%%%%%%%%%%%%%%%%%%
\begin{frame}[t]{Equilibrium}
\vspace{-.25cm}
\smallskip
{\small
\begin{df}[ {\normalsize The Decentralized Stationary Equilibrium}] \normalfont A Decentralized Stationary Equilibrium are asset policy functions and commodity choice probabilities $\{\  g_{i}(a, z, j), \pi_{ij}(a, z) \ \}_{i}$, probability distributions $\{ \ \lambda_i(a, z) \ \}_{i}$ and positive real numbers $\left \{w_i, p_{ij}, R_i\right \}_{i,j}$ such that
\begin{itemize}
\item[i]  Prices ($w_i, p_{ij}$) satisfy firms problem;
\item[ii] The policy functions and choice probabilities solve the household's problem;
\item[iii] The probability distribution $\lambda_i(a, z)$ induced by the policy functions, choice probabilities, and primitives satisfies the law of motion and is stationary;
\item[iv] Goods market clears:
\begin{align*}
p_{i} Y_{i} - \sum_{j}  X_{ji} = 0, \ \ \forall i
\end{align*}
\item[v] Bond market clears with either
\begin{align*}
\mathrm{A_i'} = 0, \ \ \forall i \ \ \ \mbox{or} \ \ \ \sum_{i}\mathrm{A_i'} = 0
\end{align*}
\end{itemize}
\end{df}
}
\end{frame}

%%%%%%%%%%%%%%%%%%%%%%%%%%%%%%%%%%%%%%%%%%%%%%%%%%%%%%%%%%%%%%%%%%%%%%%%%%%%%%%%%%%%%%%%%%%%%%%%
%%%%%%%%%%%%%%%%%%%%%%%%%%%%%%%%%%%%%%%%%%%%%%%%%%%%%%%%%%%%%%%%%%%%%%%%%%%%%%%%%%%%%%%%%%%%%%%%

\begin{frame}[t]{The HA Trade Elasticity}
How do $i$'s imports from $j$ change relative to domestic consumption due to a permanent change in $d_{ij}$?
\smallskip
\uncover<2->{{\small
\begin{prp}[{\normalsize The HA Trade Elasticity} ] \label{prp:GET} The trade elasticity between country $i$ and country $j$ is:
\begin{align}
\theta_{ij} = 1 + \int_{z,a} \bigg \{ \theta_{ij}(a,z)^{I} + \theta_{ij}(a,z)^{E} \bigg \}\omega_{ij}(a,z)da \ dz - \int_{z,a} \bigg \{ \theta_{ii,j}(a,z)^{I} + \theta_{ii,j}(a,z)^{E} \bigg \}\omega_{ii}(a,z)da \ dz \nonumber
\end{align}
which is an expenditure-weighted average of micro-level elasticities. The micro-level elasticities are decomposed into an intensive margin and extensive margin
{\footnotesize
\begin{align}
\nonumber
\begin{alert}<3>{\theta_{ij}(a,z)^{I} = \frac{\partial c_{i}(a,z,j)/ c_{i}(a,z,j)}{\partial d_{ij} / d_{ij}}}\end{alert}, \ \ \ \ \ \ \begin{alert}<4->{\theta_{ij}(a,z)^{E} = \frac{\partial \pi_{ij}(a,z) / \pi_{ij}(a,z)}{\partial d_{ij} / d_{ij}}}\end{alert}, \ \ \ \
\end{align}
}
and $\omega_{ij}(a,z)$ are the expenditure weights.
\end{prp}
}}
\only<3>{
\medskip
{\small \begin{align*}
\theta_{ij}(a,z)^{I} = \bigg [-\frac{\partial a'_{i}(a,z,j)/ p_{ij}c_{i}(a,z,j)}{\partial p_{ij}/ p_{ij}} - 1 \bigg ]\frac{\partial p_{ij}/p_{ij}}{\partial d_{ij}/ d_{ij}}.
\end{align*}}\\
\bigskip
The idea: a lower $d_{ij}$ relaxes the bc and then the division of new resources between assets and expenditure determines the intensive margin. \begin{alert}<3>{\textbf{This is larger for the poor, smaller for the rich.}}\end{alert}\\
}
\only<4>{
\medskip
{\small
\begin{align*}
\theta_{ij}(a,z)^{E} = \frac{1}{\sigma_{\epsilon}}\frac{\partial v_{i}(a,z,j)}{\partial d_{ij}/d_{ij}}  - \frac{\partial \Phi_{i}(a,z) / \Phi_{i}(a,z)}{\partial d_{ij}/d_{ij}} .
\end{align*}}\\
\bigskip
Now assume the number of countries is large\ldots\\
}
\only<5>{
\medskip
{\small
\begin{align*}
\theta_{ij}(a,z)^{E} \approx -\frac{1}{\sigma_{\epsilon}}\bigg[u'(c_{i}(a,z,j))c_{i}(a,z,j)\bigg] .
\end{align*}}\\
\bigskip
With CRRA and relative risk aversion $ > 1$ then \begin{alert}<5>{\textbf{poor hh's are the most price sensitive on the extensive margin.}}\end{alert}
}
\end{frame}

\begin{frame}[t]
\frametitle{Micro Moments | Model Consistent with HH-Level MPCs}
\begin{figure}[!t]
\centering{\includegraphics[scale = .45]{../notes/figures/mpc.pdf}}
\end{figure}
\begin{itemize}
\smallskip
\item Household MPCs consistent with \citet{kaplan2022marginal}.
\end{itemize}
\end{frame}


%%%%%%%%%%%%%%%%%%%%%%%%%%%%%%%%%%%%%%%%%%%%%%%%%%%%%%%%%%%%%%%%%%%%%%%%%%%%%%%%%%%%%%%%%%%%%%%%
%%%%%%%%%%%%%%%%%%%%%%%%%%%%%%%%%%%%%%%%%%%%%%%%%%%%%%%%%%%%%%%%%%%%%%%%%%%%%%%%%%%%%%%%%%%%%%%

\begin{frame}[t]{U.S. Welfare: Global 10\% Reduction in $d$ }
\begin{figure}[!t]
\centering{\includegraphics[scale = .55]{../notes/figures/global-welfare-household-vs-log.pdf}}
\end{figure}
\end{frame}

%%%%%%%%%%%%%%%%%%%%%%%%%%%%%%%%%%%%%%%%%%%%%%%%%%%%%%%%%%%%%%%%%%%%%%%%%%%%%%%%%%%%%%%%%%%%%%%%
%%%%%%%%%%%%%%%%%%%%%%%%%%%%%%%%%%%%%%%%%%%%%%%%%%%%%%%%%%%%%%%%%%%%%%%%%%%%%%%%%%%%%%%%%%%%%%%%

\begin{frame}[t]
\frametitle{Measuring Welfare}
\smallskip
Want is a measure of welfare in interpretable units. I'm going to focus on equivalent variation. \\
\medskip
Reminder: Given some price change delivering utility level $v'$, equivalent variation asks ``at the old prices, $p$, how much extra income must be provided to be indifferent between $v'$ and $v(p)$?''\\
\bigskip
\bigskip
My measure is a permanent, proportional increase in wealth $\tau_{i,a,z}$, at the old prices such that the new level of utility $v'_i$ is achieved:
\begin{align}
v'_i(a,z ; \mathbf{p'}) = v_i(a,z ; \mathbf{p}, \tau_{i, a,z}).  \nonumber
\end{align}\\
\bigskip
\bigskip
Also, I'm doing this across steady states, not transitions.\\ Sorry :(
\end{frame}


%%%%%%%%%%%%%%%%%%%%%%%%%%%%%%%%%%%%%%%%%%%%%%%%%%%%%%%%%%%%%%%%%%%%%%%%%%%%%%%%%%%%%%%%%%%%%%%%
%%%%%%%%%%%%%%%%%%%%%%%%%%%%%%%%%%%%%%%%%%%%%%%%%%%%%%%%%%%%%%%%%%%%%%%%%%%%%%%%%%%%%%%%%%%%%%%%

\begin{frame}[t]
\frametitle{Preferences, Shocks, and Constraints}
\begin{table}[t]
\small
\begin{center}
\refstepcounter{table}
\setlength {\tabcolsep}{4.5mm}
\renewcommand{\arraystretch}{1.60}\label{tb-calibration}
\begin{tabular}[t]{l c l}
\multicolumn{3}{c}{{\normalsize\textbf{Preferences, Shocks, and Constraints | Calibrated Parameters}} }
\\\hline \hline
Description & Value & \multicolumn{1}{c}{Target}\\
\cmidrule(lr){1-1} \cmidrule(lr){2-2} \cmidrule(lr){3-3}
Discount Factor, \ $\beta$                          & $0.92$ & \phantom{\} } Global Interest Rate of $1\%$ \\
CRRA parameter, \ $\gamma$                          & $1.45$ & \multirow{2}{*}{\Bigg \} Micro elasticities of {\small \citet{auer2022unequal}} }\\
Type One E-V parameter, \ $1 / \sigma_{\epsilon}$    & $3.0\phantom{0}$ & \\
Slope of Quality Shifter, \ $\psi_{ii}(z)$          & $0.72$ & \phantom{\} } Micro moments of {\small \citet{borusyak2021distributional} } \\
Borrowing Constraint \ $\phi_{i}$                   & --- & \phantom{\} } $50\%$ of $i$'s autarky labor income \\
Income Process on \ $z$                             & --- & \phantom{\} } {\small \citet*{krueger2016macroeconomics}} \\
\hline
\end{tabular}
\\[0.5ex]
%\parbox{5.95in}{\footnotesize \textbf{Note:} }
\end{center}
\end{table}
\bigskip
\begin{itemize}
\smallskip
\item Everything is done under financial globalization case.
\end{itemize}
\end{frame}

%%%%%%%%%%%%%%%%%%%%%%%%%%%%%%%%%%%%%%%%%%%%%%%%%%%%%%%%%%%%%%%%%%%%%%%%%%%%%%%%%%%%%%%%%%%%%%%%
%%%%%%%%%%%%%%%%%%%%%%%%%%%%%%%%%%%%%%%%%%%%%%%%%%%%%%%%%%%%%%%%%%%%%%%%%%%%%%%%%%%%%%%%%%%%%%%%

\begin{frame}[t]{County Specific Parameters | Using Gravity as a Guide}
\smallskip
\only<1>{
The problem: no closed form map from trade flows to parameters as in standard trade models. But I want the model to replicate the geographic pattern of activity seen in the data.}
\only<2>{The solution: use the gravity regression ``as a guide'' where I estimate parameters of the model so that the regression coefficients run on my model's data match that seen in the data.}
\medskip
\uncover<2->{
\begin{itemize}
\smallskip
\item Step 0. Impose a trade cost function to reduce the parameter space
\begin{align}
\log d_{ij} = d_{k} + b + l + e_{h} + m_{i}. \nonumber
\end{align}
\smallskip
\item Step 1. Run this gravity regression on the data
\begin{align}
\log \left( {\frac{M_{ij}}{M_{ii}}} \right) = {Im_{i}} + {Ex_{j}} + {d_{k}} + {b} + {l} + {e_{h}} + \delta_{ij}. \nonumber
\end{align}
\smallskip
\item Step 2. Guess TFP terms and coefficients on the trade cost function, compute an equilibrium, run the same regression from above on model generated data.
\smallskip
\item Step 3. Evaluate difference between model and data and update parameters until convergence.
\end{itemize}
}
\end{frame}

%%%%%%%%%%%%%%%%%%%%%%%%%%%%%%%%%%%%%%%%%%%%%%%%%%%%%%%%%%%%%%%%%%%%%%%%%%%%%%%%%%%%%%%%%%%%%%%%
%%%%%%%%%%%%%%%%%%%%%%%%%%%%%%%%%%%%%%%%%%%%%%%%%%%%%%%%%%%%%%%%%%%%%%%%%%%%%%%%%%%%%%%%%%%%%%%%

\begin{frame}[t]{Aggregation}
\smallskip
Aggregates arise from explicit aggregation of hh-level actions. Two examples:\\
\medskip
\medskip
\textbf{1.} Aggregate, bilateral imports are
\begin{align*}
M_{ij} = L_i \int_{z} \int_{a}  p_{ij} c_{i}(a, z, j) \pi_{ij}(a, z) \lambda_i(a, z)
\end{align*}
where $\lambda_i$ is the \emph{endogenous} distribution of hhs across states. Here trade flows take on a mixed-logit form similar to \citet*{berry1995automobile}, but everything is tied down in equilibrium. \\
\bigskip
\bigskip
\textbf{2.} The national income accounting identity (GDP = C + I + G + X - M) \ldots
\begin{align*}
p_{i} Y_{i}  =  \underbrace{L_{i} \sum_{j} \int_{z} \int_{a}  p_{ij} c_{i}(a, z, j) \pi_{ij}(a, z) \lambda_i(a, z)}_{\widetilde{P_{i} C_i}} \ + \ \underbrace{\bigg[\ \sum_{j\neq i}X_{ji} -  \sum_{j\neq i}M_{ij} \bigg]}_{-R_{i}A_i + A_{i}'}.
\end{align*}
\end{frame}

%%%%%%%%%%%%%%%%%%%%%%%%%%%%%%%%%%%%%%%%%%%%%%%%%%%%%%%%%%%%%%%%%%%%%%%%%%%%%%%%%%%%%%%%%%%%%%%%
%%%%%%%%%%%%%%%%%%%%%%%%%%%%%%%%%%%%%%%%%%%%%%%%%%%%%%%%%%%%%%%%%%%%%%%%%%%%%%%%%%%%%%%%%%%%%%%%

\begin{frame}[t]{HA Gains from Trade under Efficiency}
\vspace{-.25cm}
\smallskip
{\small
\begin{prp}[{\normalsize Trade Elasticities and Welfare Gains in the Efficient Allocation}]\label{prp:gains-efficient-allocation} The elasticity of trade to a change in trade costs between $ij$ in the efficient allocation is:
\begin{align*}
\theta_{ij} =  -\frac{1}{\sigma_{\epsilon}} \bigg [ u'(c_{i}(j)) c_{i}(j) \bigg].
\end{align*}
\uncover<2->{And the welfare gains from a reduction in trade costs between $i,j$ are
\begin{align*}
= \sigma_{\epsilon} \times  \theta_{ij} \times  \pi_{ij} \times \frac{L_i}{1-\beta},
\end{align*}
which is the discounted, direct effect from relaxing the aggregate resource constraint.} \uncover<3->{And this can be expressed as
\begin{align*}
= -\sigma_{\epsilon} \times \frac{\mathrm{d} \pi_{ii} / \pi_{ii}}{\mathrm{d} d_{ij} / d_{ij}} \times \frac{L_i}{1 - \beta}.
\end{align*}}
\end{prp}
}
\medskip
\only<1>{Same idea as in decentralized allocation, but now everyone substitutes in a common way...}

\only<2>{Mimics the results of \citet{AtkesonBurstein2010} but with household (not firm) heterogeneity.}

\only<3>{And | again| we are back to an \citet{arkolakis2012new}-like expression and with $\log$ its exact.}
\end{frame}

%%%%%%%%%%%%%%%%%%%%%%%%%%%%%%%%%%%%%%%%%%%%%%%%%%%%%%%%%%%%%%%%%%%%%%%%%%%%%%%%%%%%%%%%%%%%%%%%
%%%%%%%%%%%%%%%%%%%%%%%%%%%%%%%%%%%%%%%%%%%%%%%%%%%%%%%%%%%%%%%%%%%%%%%%%%%%%%%%%%%%%%%%%%%%%%%%

\begin{frame}[t]{Household Parameters}
\smallskip
Parameters common across countries:
\begin{itemize}
\smallskip
\item CRRA for $u$ with relative risk aversion $\gamma$ \uncover<2->{| varied to fit elasticities in \citet{auer2022unequal}}. \\
\smallskip
\item Earnings process as in \citet*{krueger2016macroeconomics}.
\smallskip
\item Discount factor $\beta$ jiggled to target a world interest rate of $1.0 \%$ in financial globalization case.
\end{itemize}
\bigskip
Parameters scaled across countries to deliver balanced-growth-like properties.
\begin{itemize}
\smallskip
\item Set $\sigma_{\epsilon,i} = \sigma_{\epsilon}\times A_i^{1-\gamma}$, \uncover<2->{| $\sigma_{\epsilon}$ varied to fit elasticities in \citet{auer2022unequal}}.
\medskip
\item Set the borrowing constraint so $\phi_{i} = \phi \times A_i$ where $\phi = 0.50$.
\end{itemize}
\bigskip
Household-specific quality shifters | a home bias term $\psi_{ii}(z)$ which additively shifts utility
\begin{itemize}
\smallskip
\item Without this prices and price elasticities determine shares, so to fit the data interactions between quality and household characteristics is a way; same idea as in \citet{berry1995automobile}.
\smallskip
\uncover<2->{\item Slope of $\psi_{ii}(z)$ wrt $z$ varied to fit \citet{borusyak2021distributional} facts.}
\end{itemize}
\end{frame}

%%%%%%%%%%%%%%%%%%%%%%%%%%%%%%%%%%%%%%%%%%%%%%%%%%%%%%%%%%%%%%%%%%%%%%%%%%%%%%%%%%%%%%%%%%%%%%%%
%%%%%%%%%%%%%%%%%%%%%%%%%%%%%%%%%%%%%%%%%%%%%%%%%%%%%%%%%%%%%%%%%%%%%%%%%%%%%%%%%%%%%%%%%%%%%%%%

\begin{frame}[t]
\frametitle{Estimates of Geographic Barriers}
\begin{table}[t]
\small
\begin{center}
\refstepcounter{table}
\setlength {\tabcolsep}{5.5mm}
\renewcommand{\arraystretch}{1.10}\label{tb-grav-est}
\begin{tabular}[t]{l c c c}
\multicolumn{4}{c}{{\normalsize\textbf{Table \ref{tb-grav-est}: Estimation Results}} }
\\\hline \hline
& & \multicolumn{2}{c}{\textbf{HAT-Model}}  \\
\cmidrule(lr){3-4}
Barrier& Moment & Model Fit & Parameter \\
\hline $[0,375)$                &$-3.10 $           & $-3.10 $              & $1.92$           \\
$[375,750)$                     &$-3.67 $           & $-3.67 $              & $2.39$           \\
$[750,1500)$                    &$-4.03 $           & $-4.03 $              & $2.64$           \\
$[1500,3000)$                   &$-4.22 $           & $-4.22 $              & $2.74$           \\
$[3000,6000)$                   &$-6.06 $           & $-6.06 $              & $4.10$           \\
$[6000,\mbox{maximum}]$         &$-6.56 $           & $-6.56 $              & $4.83$           \\
Shared border                   &$\phantom{-}0.30$  & $\phantom{-}0.30$     & $0.92$  \\
Language                        &$\phantom{-}0.51$  & $\phantom{-}0.51$     & $0.85$  \\
EFTA                            &$\phantom{-}0.04$  & $\phantom{-}0.04$     & $0.96$  \\
European Community              &$\phantom{-}0.54$  & $\phantom{-}0.54$     & $0.91$  \\
\hline
\end{tabular}
\\[0.5ex]
\parbox{4.2in}{\footnotesize \textbf{Note:} The first column reports data moments the HAT-model targets. The second reports the model moments. The third column reports the estimated parameter values.}
\end{center}
\end{table}
\bigskip
\end{frame}
%%%%%%%%%%%%%%%%%%%%%%%%%%%%%%%%%%%%%%%%%%%%%%%%%%%%%%%%%%%%%%%%%%%%%%%%%%%%%%%%%%%%%%%%%%%%%%%%
%%%%%%%%%%%%%%%%%%%%%%%%%%%%%%%%%%%%%%%%%%%%%%%%%%%%%%%%%%%%%%%%%%%%%%%%%%%%%%%%%%%%%%%%%%%%%%%%


\setcounter{framenumber}{\value{finalframe}}

%%%%%%%%%%%%%%%%%%%%%%%%%%%%%%%%%%%%%%%%%%%%%%%%%%%%%%%%%%%%%%%%%%%%%%%%%%%%%%%%%%%%%%%%%%%%%%%%%
%%%%%%%%%%%%%%%%%%%%%%%%%%%%%%%%%%%%%%%%%%%%%%%%%%%%%%%%%%%%%%%%%%%%%%%%%%%%%%%%%%%%%%%%%%%%%%%%%



\end{document} 