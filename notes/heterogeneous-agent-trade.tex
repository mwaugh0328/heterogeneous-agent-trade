\documentclass[12pt,pdftex]{article}
\usepackage[pdftex]{graphicx,color}
\usepackage{setspace,palatino,multirow}
\usepackage{amsmath,amssymb}
\usepackage{titlesec}
\usepackage{lscape}
%\usepackage{subfigure}
\usepackage{threeparttable}
\usepackage{natbib}
\bibliographystyle{ecta}
\usepackage{cite}
\usepackage{booktabs}
\usepackage{subcaption}
\usepackage{pdflscape}
\usepackage{afterpage}
\usepackage{xcolor}
\usepackage{rotating}
\usepackage[listings, most]{tcolorbox}

\definecolor{nblue}{RGB}{0,0,128}

\usepackage[pdftex,colorlinks=true, bookmarks=false,
pdfstartview={XYZ null null 0.85},
pdftitle={Heterogeneous Agent Trade},
pdfauthor={ Michael E. Waugh},
pdfkeywords={economics, trade, dynamics, quant econ, consumption, data science,
waugh, incomplete markets, inequality, julia, Armington, Minneapolis Fed, price elasticity, distance, python, matplotlib},
colorlinks=true,linkcolor=darkgray,citecolor=darkgray,urlcolor=darkgray,
breaklinks]{hyperref}

\newcounter{saveeqni}%
\newcounter{saveeqn01i}%
\newcommand{\alpheqni}{\setcounter{saveeqni}{\value{section}}%
%\setcounter{saveeqn01i}{\value{subsectioni}}%
\renewcommand{\theequation}
    {\alph{saveeqni}\mbox{.\arabic{equation}}}}%
\newcommand{\reseteqni}{\setcounter{equation}{\value{saveeqni}}%
\renewcommand{\theequation}{\arabic{equation}}}%

\newtheorem{as}{Assumption}
\newtheorem{reg}{Regularity Condition}
\newtheorem{conjecture}{Conjecture}
\newtheorem{corr}{Corollary}
\newtheorem{df}{Definition}
\newtheorem{lemma}{Lemma}
\newtheorem{prp}{Proposition}
\newtheorem{rmk}{Remark}
\newenvironment{prf}{{\bf Proof}}{\hfill { }}

\tcolorboxenvironment{corr}{%
boxrule = 0mm, breakable, colframe=white,
before skip=5pt,after skip=5pt,
colback=gray!5!white,
top = 2mm,
bottom = 2mm%,
%borderline north={1pt}{1pt}{gray},
%borderline south={1pt}{1pt}{gray}
}

\tcolorboxenvironment{prp}{%
boxrule = 0mm, breakable, colframe=white,
before skip=5pt,after skip=5pt,
colback=gray!5!white,
top = 2mm,
bottom = 2mm%,
%borderline north={1pt}{1pt}{gray},
%borderline south={1pt}{1pt}{gray}
}

\tcolorboxenvironment{df}{%
boxrule = 0mm, breakable, colframe=white,
before skip=5pt,after skip=5pt,
colback=gray!5!white,
top = 2mm,
bottom = 2mm%,
%borderline north={1pt}{1pt}{gray},
%borderline south={1pt}{1pt}{gray}
}

\DeclareMathOperator*{\plim}{plim}
\DeclareMathOperator*{\umax}{max}

\special{papersize=8.5in,11in}
\onehalfspacing
\setlength{\parindent}{0.1em}
\setlength{\parskip}{.09in}
\textwidth15.75cm
\evensidemargin 1.5in
\oddsidemargin 1.5in
\topmargin 8.5cm
\textheight 10in
\hyphenation{over-lapping}

\titleformat{\section}{\color{black}\large\bf}{\color{black}{\thesection.}}{.25cm}{}
\titleformat{\subsection}{\color{black}\normalsize\bf}{\thesubsection.}{.5em}{}
\titleformat{\subsubsection}{\color{black}\normalsize\bf}{\thesubsubsection.}{.5em}{}

\titlespacing{\section}{0pt}{*1.5}{*.5}
\titlespacing{\subsection}{0pt}{*1.5}{*.5}
\titlespacing{\subsubsection}{0pt}{*1.5}{*.5}

\def\thesection{\arabic{section}}
\def\thesubsection{\arabic{section}.\arabic{subsection}}
\def\thesubsubsection{\arabic{section}.\arabic{subsection}.\Alph{subsubsection}}

\def\citeapos#1{\citeauthor{#1}'s (\citeyear{#1})}

\renewcommand{\arraystretch}{1.1}
\usepackage[margin=2cm]{geometry}

\begin{document}

\begin{onehalfspacing}

{\large \textbf{\href{https://www.waugheconomics.com/uploads/2/2/5/6/22563786/heterogeneous-agent-trade.pdf}{Heterogeneous Agent Trade}}}

\vspace{0.5cm}

\href{http://www.waugheconomics.com/}{Michael E. Waugh} \\ Federal Reserve Bank of Minneapolis and NBER

\vspace{0.5cm}

This draft: September 2023

\vspace{1.5cm}


\normalsize

ABSTRACT ------------------------------------------------------------------------------------------------------------

This paper studies the implications of household heterogeneity for trade. I develop a model where household heterogeneity is induced via the standard incomplete markets model and results in heterogenous price elasticities. Conditional on exposure to trade, heterogenous price elasticities imply that different households value price changes differently and, thus, rich and poor households experience different gains from trade. I calibrate the model to match bilateral trade flows and micro-facts about household-level expenditure patterns and elasticities. I find that gains from trade are pro-poor and large with a ten percent reduction in trade costs delivering welfare gains equivalent to a permanent, three percent increase in income for the poorest households. These gains are four and a half times larger than those for the richest households and the average gains from trade are three times larger than representative agent benchmarks.

------------------------------------------------------------------------------------------------------------------------------
%%\vspace{0.25cm}
%
%%JEL Classification:
%%
%%
%%Keywords:

\vspace{6.0cm}

\footnotesize Email: michael.e.waugh@gmail.com. The views expressed herein are those of the author and not necessarily those of the Federal Reserve Bank of Minneapolis or the Federal Reserve System. This project was developed with research support from the National Science Foundation (NSF Award number 1948800). Thomas Hasenzagl provided excellent research assistance. My github repository provides the code and supplementary work behind this paper at \url{https://github.com/mwaugh0328/heterogeneous-agent-trade}.

\hspace{-0.05cm}



\thispagestyle{empty}
\newpage
\normalsize

This paper studies the implications of household heterogeneity for trade. From the perspective of trade, household heterogeneity is interesting because of the notion that some benefit from trade and others don't. One aspect of these unequal gains relates to the idea that rich and poor consumers have different sensitivities to price and, thus, they shape the gains from trade.  I develop this idea in a model that results in heterogenous price elasticities and I study it's implications for trade qualitatively and quantitatively.

The core issue in my model is that heterogeneity in price sensitivity reflects heterogeneity in the marginal utility of consumption across households. Then even if rich and poor households are equally exposed to changes in prices | heterogeneity in price sensitivity implies that they value a price change differently. Thus, poor, high marginal utility households | who are very sensitive to price | benefit more from trade than rich households. Quantitatively, I find that this mechanism is powerful with the poorest households gaining four and a half times more than the richest. And the average gains from trade are nearly three times larger than standard, representative agent benchmarks.

The model that I develop builds upon workhorse frameworks. Trade in goods follows the Armington tradition with producers in each country producing a national variety. The important twist is that households have random utility over these varieties and they make a discrete choice over the varieties to consume (\citet{mcfadden1974frontiers}).  Household heterogeneity is induced via the standard incomplete markets model (\citet{bewley1979optimum}, \citet{imrohorouglu1989cost}, \citet{huggett1993risk}, \citet{aiyagari1994uninsured}) with households facing incomplete insurance against idiosyncratic productivity and taste shocks. This setting naturally leads to dispersion the marginal utility of consumption.

Together, the discrete choice and incomplete markets model interact with the key force being household-level trade (price) elasticities that endogenously vary with income and wealth. Income and wealth matter because a household's price elasticity, in essence, is about the marginal gain in utility from a percent change in consumption. Under certain conditions on preferences, a price reduction delivers a lot of extra utility for poor, high marginal utility households and this induces strong substitution. In contrast, rich households' marginal utility is low, a price reduction delivers little incremental gain in utility and, thus, substitution is weak. In aggregate, the distribution of households | how many rich and poor people are in a country | then determines the aggregate response of economy to changes in trade frictions and the aggregate pattern of trade.

The issues behind heterogeneity in price sensitivity leads to new perspectives on the welfare gains from trade. I show how one aspect of the gains from trade reflects the expected, discounted stream of changes in a household's home choice probability, similar in spirit to the result of \citet*{arkolakis2012new}. Unpacking this component reveals that the change in the home choice probability is essentially about two forces: (i) how exposed a household is to trade and (ii) its own price elasticity. Because the elasticity part reflects the marginal utility of consumption, it delivers the intuitive idea that one aspect of the gains from trade is a households' individual valuation of the price reduction. So even if a rich and poor household have similar expenditure patterns, the reduction in price is more valuable on the margin for the poor, high marginal utility households.

Before moving on to the quantitative work, I explore two special cases to highlight the role that market incompleteness and preferences play in shaping these results. The first case is the efficient allocation where a planner can reallocate resources and overcome market incompleteness. In this case, I recover ``first-best intuition'' with the gains from trade only reflecting the direct savings associated with a reduction in trade costs. In this allocation, changes in expenditure patterns are not relevant via an envelope theorem argument | the planner already sources goods from the correct places. And heterogeneity in a household's valuations of cost reductions are irrelevant because marginal utility is equated. While my economy is about heterogeneity on the household side, this result is reminiscent of \citet{AtkesonBurstein2010} and the irrelevance of firm heterogeneity in an economy where the allocation is efficient. Thus, the core issues at play in my model are not household heterogeneity per se, but inefficiencies induced by market incompleteness.

The second special case is when the utility function over the physical commodity is $\log$. With $\log$ utility, I obtain a separation result where aggregate trade outcomes ``separate'' from household heterogeneity and all households gain through lower commodity prices in the same way.\footnote{This case is also interesting because \citet*{anderson1987ces} showed that in a static model with log utility and additive logit shocks, the economy behaves \emph{as if} there were a representative agent CES consumer. In my economy, my suspicion was that market incompleteness and intertemporal behavior would nullify \citeapos{anderson1987ces} result|it does not.} Trade takes a constant elasticity, gravity form with the trade elasticity pinned down by the dispersion parameter on the taste shocks similar to \citet{eaton2002technology}. The welfare impact of lower commodity prices is common across households and takes the same form as in \citet{arkolakis2012new} with the trade elasticity and the change in the share of home purchases summarizing the gains from trade. The reason behind these results is that the marginal gain in utility from a percent change in consumption | $u'(c)c$ | is independent of the level of consumption with $\log$ preferences. Thus, both rich and poor households substitute in the same way and they gain the same amount from lower prices.

Quantitatively, I make a contribution by computing and calibrating the model at a scale typically reserved for static trade models. As a testing ground, I focus on the data set of \citet{eaton2002technology}. The 19 countries in this data set is about the right size to easily illustrate how a very rich model like this can work in a multi-country setting. Moreover, the \citet{eaton2002technology} data set provides a well defined benchmark disciplined by bilateral trade flows and gravity variables|so it's a nice laboratory to explore new issues in.

The calibration challenge is the following. The model does not admit a gravity representation that allows researchers to invert trade frictions and productivity levels from trade flows as done in \citet{eaton2002technology} and many subsequent papers. Similarly, the model does not admit the use of exact-hat algebra which allows the research to construct counterfactuals without the knowledge of primitives like trade frictions or productivity (see, e.g., \citet{costinot2014trade} or the dynamic extension in \citet*{caliendo2015trade}).

My solution is to use the insight that the regressions employed in gravity frameworks provide very accurate descriptions of the data generating process. Rather than treating the gravity regression as a structural relationship, I use it as a ``guide'' and use an indirect inference procedure where I estimate parameters of the model so that the regression coefficients from a standard gravity regression run on my model's data match that seen in the data. This procedure works well and, thus, the model is able to match spatial distribution of economic activity in the data|just as well as standard, constant elasticity gravity models.

I then illustrate the quantitative implications of the model by working through several counterfactual changes in trade costs and studying the welfare gains from trade. Figure \ref{prp:gains-trade} illustrates the main quantitative takeaway | that the gains from trade are strongly pro-poor with the poorest households gaining four and a half times more from trade than the richest households. And because the poor gain a lot, while rich households look a lot like \citet{arkolakis2012new}-like households, the average gains from trade are three times larger than representative agent benchmarks. I show that this general theme | pro-poor and large on average | holds both for bilateral reductions and global reductions.

%Behind these large, pro-poor gains are the force I emphasize in the theory section of the paper | heterogenous price elasticities. This is validated in two ways. First, my calibration scheme builds on the evidence of \citet{borusyak2021distributional} by ensuring household-level import expenditure shares are roughly the same across rich and poor households, so differences in exposure do not account for the pro-poor gains found. In contrast, the model was designed and calibrated to match the facts of \citet*{auer2022unequal} | that poor households are very elastic with respect to price and, thus, poor households strongly value a price reduction. This point is further validated by exploiting the example of the $\log$ preference model with households substituting in a common way. In this specification, micro-level heterogeneity plays little role and the gains from trade are nearly uniform across rich and poor households.

Motivating my work has been a sequence of papers focusing on measuring the heterogenous impacts of trade on the consumer side. \citet{fajgelbaum2016measuring}, \citet{carroll2020heterogeneous}, \citet{borusyak2021distributional}, \citet{jaccardtoronto} are recent examples that measure heterogeneity in import exposure. \citet*{auer2022unequal} and \citet*{colicev2022impact} go a step further measuring heterogeneity in price sensitivity across the income distribution and this type of evidence is very much the launching point for my paper.

While this work motivates my paper, I take a conceptually different approach. Rather than focusing on measurement, I develop a model of household heterogeneity that endogenously delivers heterogeneity in price elasticities and study its implications. In this sense, my papers approach is most similar to \citet*{fajgelbaum2011income} who study how inequality and non-homotheticities shape trade in vertically differentiated products. A unique aspect of my work is that I start with a theory behind the distribution of income and wealth. This theory plus a demand system with heterogenous price elasticities then breaks aggregation in the goods market. And because aggregation is broken, it opens the door to new insights about trade, the interaction of trade and financial markets, and how market incompleteness shapes the aggregate pattern of trade.

My paper also relates to a recent series of papers that combines trade models with heterogenous agent, incomplete market models. Some of this is my own work in \citet{lyon2018redistributing}, \citet{lyon2019}, \citet{waugh_consumption}; \citet*{gaston2018}, \citet{carroll2020heterogeneous}, and \citet{dvorkin2023heterogeneous} are important contributions as well. This class of papers primarily focuses on how heterogenous exposure through the labor market passes through to consumption and, thus, welfare. In this paper, I'm doing something different and it is focused on the heterogenous exposure through the consumption side.

This paper also relates to a body of work focusing on the pricing implications in the presence of heterogenous price sensitivity. \citet{nakamura2010accounting} is an early example of a macro-style model with an IO-style demand system similar in spirit to my paper and they focuses on the implications for the incomplete pass through of shocks to prices. My own work in \citet{p-iq} is very much a companion piece to this paper with imperfect competition in product markets and it focuses on the heterogenous pass-through of supply and demand shocks into prices for different types of consumers. \citet{nord2022shopping} takes a search-theoretic approach, but the core issue is the same | how demand composition affects pricing decisions. With that said, this model simplifies matters by focusing on a world with perfect competition and, hence, I turn my focus on how household heterogeneity matters for trade.

%\footnote{Its important for the reader to distinguish the pattern of heterogeneity across sectors vs. within a sector (\citet{cravino2017distributional} makes this distinction very clear). The evidence suggests that the patterns work in different directions with poor households consuming more traded goods, but within traded goods the poor consume lower price varieties and less imported content. My model is of one sector and about expenditure and substitution within that one sector.

\section{The Heterogeneous Agent Trade Model}

This section describes the model and then defines the decentralized competitive equilibrium. Trade is in the Armigton tradition with each country producing a nationally differentiated variety. Households face the ``income fluctuations problem'' as in the standard incomplete markets tradition (see, e.g., Chapter 17 of \citet{ljungqvist2012recursive}).

The key twist is that I do not employ modeling techniques with aggregation at household level across national varieties. Instead, I lean into the household heterogeneity and have households make a discrete choice over the varieties they consume in addition to their savings decisions. Aggregate trade flows, trade elasticities, and the gains from trade are then defined by the explicit aggregation of household-level decisions to purchase different varieties, their elasticity of demand, and their gains from trade.

\subsection{Production and Trade}\label{sec:trade}

There are $M$ locations which I call a country. Each country produces a nationally differentiated product. In country $i$, competitive firms' production technology to produce variety $i$ is:
\begin{align}
Q_i = A_i N_i,
\label{eq:production}
\end{align}
where $A_i$ is total factor productivity and $N_i$ are the efficiency units of labor supplied by households in country $i$.\footnote{Note that lack of physical capital in the model. Households here are saving in via pure exchange of non-state contingent IOUs as in \citet{huggett1993risk} rather than in physical capital as in \citet{aiyagari1994uninsured}.}

I focus on only one type of barrier to trade: there are iceberg trade costs $d_{ij} > 1$ for a good to go from supplier $j$ to buyer $i$.

Profit maximization of the producers in location $i$ results in the wage per efficiency unit reflecting the value of the marginal product of labor
\begin{align}
w_{i} = p_{i} A_{i}.
\label{eq:marginal-product}
\end{align}
Given iceberg trade costs, the unit cost for country $i$ to purchase a good from location $j$ is
\begin{align}
p_{ij} = \frac{d_{ij}w_{j}}{A_{j}}.
\label{eq:marginal-product-ship}
\end{align}
This is the trade and production side of the model. While sparse, it's worth reminding you that with a representative agent and a constant elasticity Armigton aggregator much comes out of this model. There is a gravity equation relating bilateral trade flows to country characteristics with a constant trade elasticity. And there are two sufficient statistics (the trade elasticity and home trade share) that globally characterize the welfare gains from trade. In the next section, I give up on the representative agent.

\subsection{Households}

There is a mass of $L_i$ households in each country. Households are immobile across countries. They are infinite lived and have time-sparable preferences over consumption of varieties:
\begin{align}
E_{0} \sum_{t = 0}^{\infty} \beta^{t} \tilde{u}( \{ c_{ijt} \}_{M}),
\end{align}
where the notation $\{ c_{ijt} \}_{M}$ means that the household has preferences over all $j$ varieties supplied by $M$ countries in the world. Here I'm indexing things by $ij$ to denote the variety $j$ that is consumed in location $i$ at date $t$.

Households' period utility function is of the random utility class and each period households can only consume one variety.\footnote{A more formal statement of preferences is that they $\tilde{u}( \{ c_{ijt} \}_{M}) = \sum_j \imath_{j} \tilde{u}( c_{ijt})$ where $\imath_{j}$ is an indicator function taking the value one if the consumer chooses variety $j$ and zero otherwise.} The utility associated with the choice of variety $j$ is
\begin{align}
\tilde{u}( c_{ijt} ) =  u(c_{ijt}) + \epsilon_{jt}. \label{eq:utility}
\end{align}
where the $\epsilon_{jt}$ are iid random variables across time, households, and countries. For the analysis, I assume that these shocks are distributed Type 1 Extreme Value with CDF
\begin{align}
F(\epsilon) &= \exp(-\exp(-\sigma_{\epsilon}^{-1}\epsilon))
\end{align}
where $\sigma_{\epsilon}$ is the dispersion parameter. A useful generalization of this setting to a multi-sector model is the ``infinite shopping isle'' approach of \citet{p-iq} where these shocks take on a Generalized Extreme Value representation and then households choose the sector and then the variety each period.

For now, all I assume is that the utility function over the physical good $c_{ijt}$ is well behaved. In the analysis below I explore different specifications of the utility function $u$ over the physical commodity. The canonical case for product markets (\citet{anderson1987ces}) or the spatial literature is where $u$ is $\log$ utility. Below, I highlight the rather curious properties of this case.

A household's efficiency units are stochastic and they evolve according to a Markov chain. So, $z$ is a household's efficiently units and $\mathcal{P}(z,z')$ describes the probability of a household with state $z$ efficiency units transiting to state $z'$. Again, I assume that $\mathcal{P}$ is well behaved in the necessary ways.

Households can save and borrow in a non-state contingent asset $a$ that is denominated in the units of the numeraire. One unit of the asset pays out with gross interest rate $R_i$ next period. I discuss this more in depth below, but the determination of $R_{i}$ is that which clears the bond market (local or global). A country specific, exogenous debt limit $\phi_{i}$ constrains borrowing so:
\begin{align}
a_{t+1} \geq - \phi_{i}.
\label{eq:borrowing-constraint}
\end{align}
All these pieces come together in the household's budget constraint, conditional on choosing variety $j$ to consume, and focusing on a stationary setting where prices are constant:
\begin{align}
p_{ij}c_{ijt} +  a_{t+1} \leq    R_{i} a_{t} + w_{i} z_{t}.\label{eq:trade-budget-constraint}
\end{align}
The value of asset purchases and consumption expenditures must be less than or equal to asset payments and labor earnings.

\subsection{The Household Problem}

The state variables of a individual household are its asset holdings and efficiency units. As alluded to above, for now I focus on a stationary setting where aggregates are not changing and, thus, I abstract from carrying the notation associated with them around.\footnote{If you \emph{do} want to carry them around, notice that all that households in each country care about are prices (today and in the future). The distributions of households in other countries, per se, don't matter. Thus, the relevant aggregate states in country $i$ are $\big [ \ \{ w_i \}_{M}, R_i \big ]$ which is the collection wage per efficiency units and the interest rate.}

The value function of a household in country $i$, after the variety shocks are realized, is
\begin{align}
v_{i}(a, z) = &  \max_{j} \big  \{ \  v_{i}(a, z, j)  \ \big \}
\label{eq:valuefun}
\end{align}
which is the maximum across the value functions associated with the discrete choices of different national varieties. The value function conditional on a choice of variety is
\begin{align}
v_{i}(a, z, j) = &  \max_{\ a' \ }\bigg  \{ u(c_{ij}) + \epsilon_{j}  + \beta \, \mathbb{E} [v_{i}(a', z')]  \bigg\}
\label{eq:value_fun_option} \\
\nonumber \\
\mbox{subject to}  \ & (\ref{eq:borrowing-constraint}) \  \mathrm{and} \ (\ref{eq:trade-budget-constraint}) \nonumber
\end{align}
where households choose asset holdings and the level of consumption is residually determined through the budget constraint. Associated with the solution to this problem is a policy function $g_{i}(a,z, j)$ which solves (\ref{eq:value_fun_option}) and maps current states into asset holdings tomorrow $a'$ contingent upon the variety choice $j$. Correspondingly, there is a consumption function $c_{i}(a,z, j)$ mapping states into consumption today, contingent upon the variety choice $j$.

The continuation value function on the right-hand side of (\ref{eq:value_fun_option}) is the expectation over (\ref{eq:valuefun}) with respect to (i) efficiency units next period, $z'$ and (ii) the variety taste shocks. An implication of this is that households understand that their may be situations where they really desire, say, a high priced imported good and, hence, save accordingly.

The Type 1 extreme value distribution on the taste shocks give rise to the following choice probabilities for each differentiated good:
\begin{align}
\pi_{ij}(a, z) = \exp \left( \frac{ v_{i}(a, z, j) }{\sigma_{\epsilon}} \right) \Bigg / \Phi_{i}(a,z), \label{eq:choice-prob} \\
\nonumber \\
\mbox{where} \ \ \ \Phi_{i}(a,z) := \sum_{j'} \exp \left( \frac{ v_{i}(a, z, j') }{\sigma_{\epsilon}} \right). \label{eq:big-phi}
\end{align}
which is the probability that a household with assets $a$ and efficiency units $z$ chooses country variety $j$. The term in the denominator, $\Phi_{i}(a,z)$, has a ``price-index'' interpretation and is very similar in spirit to the same term in \citet{eaton2002technology}. And then the expectation of (\ref{eq:valuefun}) with respect to the taste shocks takes the familiar log-sum form
\begin{align}
v_i(a, z) = \sigma_{\epsilon} \log \left\{ \Phi_{i}(a,z)  \right\}. \label{eq:log_sum}
\end{align}
Associated with this problem in (\ref{eq:value_fun_option}) for non-borrowing-constrained households is an Euler Equation for each variety choice $j$:
\begin{align}
\frac{u'(c_{i}(a, z, j))}{p_{ij}} = \beta R_{i} \mathrm{E}_{z'} \left[ \sum_{j'} \pi_{ij'}(a', z') \frac{u'(c_{i}(a', z', j'))}{p_{ij'}} \right].
\label{eq:euler_equation}
\end{align}
This has a very natural interpretation: a household equates marginal utility of consumption today with expected discounted marginal utility of consumption tomorrow adjusted by the return on delaying consumption. The interesting feature here is that the expected value of the marginal utility of consumption reflects the uncertainty over one's preference over different varieties tomorrow via the choice probabilities. And note that households has some control over these probabilities as the asset choice today influence the choice probabilities tomorrow.

Before moving on to aggregation, I make one useful observation that assists the analysis. Stare at (\ref{eq:choice-prob}) and (\ref{eq:log_sum}) long enough, one can arrive at a dynamic, sufficient statistic representation of $v_i(a, z)$. Appendix \ref{apx-sec:gains-trade} works through the individual steps, but (\ref{eq:log_sum}) can be summarized as
\begin{align}
v_i(a, z) = -\sigma_{\epsilon} \log \pi_{ii}(a,z) + u(c_{i}(a,z,i)) + \beta \mathbb{E}_{z'} v_{i}(a',z').
\label{eq:log_sum-home}
\end{align}
Here the ex-ante value function (prior to the realization of the preference shocks) is expressed as a sum of the log home choice probability, utility over physical consumption of the home good, and recursively the expected value function tomorrow. What's going on here is that the home choice probability $\pi_{ii}$ summarizes the expected value of the taste shocks, their benefits, and how households respond to them in the future.\footnote{Home choice probabilities are not necessarily the same as home trade shares, but this is closely related to Equation (15), Footnote 42 of \citet{eaton2002technology} and I'm heading towards situations where this result plus restrictions on $u$ give rise to the result in \citet{arkolakis2012new}.}

Equation (\ref{eq:log_sum-home}) together with (\ref{eq:euler_equation}) also provides more insight about how households' savings motives interact with the variety choice. Focusing on a household consuming the home good (and note that the left-hand-side below could be for any variety choice), the Euler Equation in (\ref{eq:euler_equation}) becomes:
\begin{align}
\frac{u'(c_{i}(a,z,i))}{p_{ii}} = \beta \mathrm{E}_{z'} \bigg \{ -\sigma_{\epsilon} \frac{\partial \pi_{ii}(a',z') / \pi_{ii}(a',z')}{\partial a'} + \frac{u'(c_{i}(a',z',i))R_i}{p_{ii}} \bigg \},
\label{eq:euler_equation-home}
\end{align}
which says that an unconstrained household should be indifferent between the marginal utility of consumption forgone to hold some more assets and two components: (i) the benefit from how a change in assets changes in their variety choice in the future and this is summarized by the change in the home choice probability and (ii) the direct benefit of the returns on the assets evaluated at the marginal utility of consumption.

\subsection{Aggregation}

\textbf{Aggregation.} At the core of aggregation is a probability distribution $\lambda_{i}(a, z)$ describing the measure of households across the individual states. This distribution evolves according to
\begin{align}
\lambda_i(a', z') = \sum_{j} \int_{z}\int\displaylimits_{a: a' = g_{i}(a, z, j)} \pi_{ij}(a, z) \mathcal{P}(z, z') \lambda_i(a, z) \ da \ dz.
\label{eq:law_motion}
\end{align}
where the inner most term describes the mass of households choosing variety $j$, multiplied by the probability that $z$ transits to $z'$, multiplied by the existing measure of households with states $a$ and $z$. This is integrated with respect to those actually choosing asset holdings $a'$, over all $z$'s, and then summed over the different variety choices.

Given this distribution, everything else follows. First focusing on trade, aggregate bilateral imports are
\begin{align}
M_{ij} = L_i \int_{z} \int_{a}  p_{ij} c_{i}(a, z, j) \pi_{ij}(a, z) \lambda_i(a, z) \ da \ dz.
\label{eq:imports}
\end{align}
Here imports take on a mixed logit formulation that very much mimics that used in the industrial organization literature, e.g, \citet*{berry1995automobile}. There are, however, several interesting differences. First, there is an active intensive margin, not unit demand. Second, inside the choice probability $\pi_{ij}(a, z)$ is the non-linear value function from (\ref{eq:valuefun}).\footnote{A good contrast is \citet{nevo2000practitioner} where inside the choice probability is an indirect utility function of the form $\eta \times (y - p_{ij})$ where $y$ and $p$ are in logs and $\eta$ is a parameter to be estimated and it stands in for the marginal utility of consumption. And related to my next comment, $\lambda$ is just ``read from the data'' and treated as policy invariant.} Because the choice probability reflects the value function, it embeds forward looking behavior of the household.

The third interesting feature is that the mixing distribution (the $\lambda$ ) over which demands are aggregated is endogenous. Through the law of motion in (\ref{eq:law_motion}) household behavior determines the distribution of wealth. In other words, this model imposes cross-equation restrictions between aggregate demand and individual demands through the distribution. So it's not a free parameter and it will change with changes in primitives of the environment.

Similar to imports, aggregate bilateral exports from country $i$ to country $j$ are
\begin{align}
X_{ji} = L_j \int_{z} \int_{a}  p_{ji} c_{j}(a, z, i) \pi_{ji}(a, z) \lambda_i(a, z)da \ dz.
\label{eq:exports}
\end{align}
The value of aggregate consumption is
\begin{align}
\widetilde{P_{i} C_i}  &=  L_{i} \sum_{j} \int_{z} \int_{a}  p_{ij} c_{i}(a, z, j) \pi_{ij}(a, z) \lambda_i(a, z)da \ dz \label{eq:bigC}
\end{align}
In (\ref{eq:bigC}), one can see both a bug and a feature of this model. Here there is an ``index number problem`` in the sense that there is not an ideal price index for which one can decompose aggregate values in to a price and quantity component. This is in contrast to, e.g., a model where households consume a CES bundle of goods.

Finally, the aggregate quantity of asset holdings integrates across the asset choices of individual households
\begin{align}
\mathrm{A}_i' = L_{i}\sum_{j} \int_{z} \int_{a}  g_{i}(a, z, j) \pi_{ij}(a, z) \lambda_i(a, z) da \ dz.
\label{eq:aggregate_asset}
\end{align}
which integrates over the asset choices|given the policy function $g_{i}(a, z, j)$ and variety choices $\pi_{ij}(a, z)$. And then sum's across the different varieties available.

\textbf{National Accounting.} From here, I reconstruct national income and product identities. Starting from the production side, aggregate efficiency units are
\begin{align}
N_i = L_{i}\int_{z} \int_{a}\ z \lambda_i(a, z) da \ dz. \label{eq:ag-labor-supply}
\end{align}
and from here we can connect the value of aggregate production must equal aggregate payments to labor so
\begin{align}
p_{i} Y_{i} = p_{i} A_{i} N_{i} = L_i \int_{z} \int_{a} w_{i} \ z \ \lambda_i(a, z) da \ dz,
\label{eq:value_production}
\end{align}
Then by summing over individual consumers' budget constraint and substituting in (\ref{eq:value_production}), the aggregated budget constraint is:
\begin{align}
p_{i} Y_{i}  = \widetilde{P_{i} C_i}  + \bigg[-R_i\mathrm{A_i} +  \mathrm{A_i'} \bigg],
\label{eq:aggregate_budget_constraint}
\end{align}
where national income equals the value of aggregate consumption $\widetilde{P_{i} C_i}$ and the country's net factor payments and net asset position. To arrive at the standard national income accounting identity, simply work with the relationship between production, exports, and aggregate consumption in (\ref{eq:bigC}) and imports gives rise to
\begin{align}
p_{i} Y_{i}  = \widetilde{P_{i} C_i} + \bigg[\ \sum_{j\neq i}X_{ji} -  \sum_{j\neq i}M_{ij} \bigg],
\label{eq:gdp}
\end{align}
where national production or GDP equals consumption plus exports minus imports. A comparison of (\ref{eq:aggregate_budget_constraint}) and (\ref{eq:gdp}) then makes clear that the trade imbalance is connected with a countries net factor payments and net asset position.

Beyond accounting, this last observation shows how trade flows are interlinked with financial flows. Inspection of the individual elements in (\ref{eq:imports}), (\ref{eq:aggregate_asset}), and the households' budget constraint reveal that household's asset positions are intertwined with trade flows through both the intensive (how much to consume and, hence, save) and the extensive margins (which variety to consume). Thus, a feature of this model is that the trade side is interlinked with the financial side of the economy in a non-trivial way.

\subsection{The Decentralized Equilibrium}

In this section, I discuss the market clearing conditions that an equilibrium must respect and then define the Decentralized Equilibrium of this economy.

\textbf{The Goods Market.} Goods market clearing equates the value of production of commodity $i$  with global demand for country $i$'s commodity:
\begin{align}
p_{i} Y_{i} &= \sum_{j}  X_{ji} \label{eq:goods-supply},
\end{align}
where the left hand side is production and the right hand side is world demand for the commodity (via exports) from (\ref{eq:exports}).

\textbf{The Bond Market.} The second market clearing condition is the bond market. There are two case worth thinking about here. One is of ``financial autarky'' in which there is a local bond market that facilitates within country asset trade, but not across countries. In this case, there is an interest rate $R_i$ for each country and the associated market clearing condition is
\begin{align}
\mathrm{A_i'} = 0, \ \ \forall i
\label{eq:bond-market-country}
\end{align}
which says that net asset demand within each country $i$ must be zero. As is common in the trade literature, this condition implies that trade is balanced|just stare at (\ref{eq:aggregate_budget_constraint}) and (\ref{eq:gdp}). Yet, even with balanced trade, there is still within country trade of financial assets. Some households are savers, others are borrowers and the interest rate is that which the net asset position is zero.

The second case is of ``financial globalization'' where there is a global bond market that facilitates both within country asset trade, and across countries. In this case, there is a single interest rate $R$ and the associated market clearing condition is
\begin{align}
\sum_{i}\mathrm{A_i'} = 0
\label{eq:bond-market-globalization}
\end{align}
In this case trade need not be balanced for each country. Here a specific country might run, say, a trade deficit because at the given prices, the total amount of borrowing within a country is larger than the total amount of saving. However, across all countries total borrowing must equal total saving.

Below I formally define the Decentralized Stationary Equilibrium where private market participants taking prices as given solve their problems, the distribution of households is stationary, and prices are consistent with market clearing.

\begin{df}[\textbf{The Decentralized Stationary Equilibrium}] \normalfont A Decentralized Stationary Equilibrium are asset policy functions and commodity choice probabilities $\{\  g_{i}(a, z, j), \pi_{ij}(a, z) \ \}_{i}$, probability distributions $\{ \ \lambda_i(a, z) \ \}_{i}$ and positive real numbers $\left \{w_i, p_{ij}, R_i\right \}_{i,j}$ such that
\begin{itemize}
\item[i]  Prices ($w_i, p_{ij}$) satisfy (\ref{eq:marginal-product}) and (\ref{eq:marginal-product-ship});
\item[ii] The policy functions and choice probabilities solve the household's optimization problem in (\ref{eq:valuefun}) and (\ref{eq:value_fun_option});
\item[iii] The probability distribution $\lambda_i(a, z)$ induced by the policy functions, choice probabilities, and primitives satisfies (\ref{eq:law_motion}) and is stationary;
\item[iv] Goods market clears:
\begin{align}
p_{i} Y_{i} - \sum_{j}  X_{ji} = 0, \ \ \forall i
\end{align}
\item[v] Bond market clears with either
\begin{align}
\mathrm{A_i'} = 0, \ \ \forall i \ \ \ \mbox{or} \ \ \ \sum_{i}\mathrm{A_i'} = 0
\label{eq:fa-condition}
\end{align}
\end{itemize}
\end{df}


\subsection{Outline of the rest of paper}

This model above has households making individual choices over national varieties, savings, all while facing productivity and taste shocks. Explicit aggregation of household behavior determines the pattern of trade and this is linked with trade in financial assets.  The remaining sections of the paper work through the following questions:
\begin{enumerate}
\item \textbf{What are the model's implications for trade elasticities and the gains from trade in decentralized allocation?} Here I characterize micro and macro level trade elasticities and then the gains from trade and I connect them with heterogeneity in the marginal utility of consumption that is induced by market incompleteness. I explore two special cases: (i) the efficient allocation and (ii) when utility is $\log$ and how this case delivers separation between the heterogenous agent side of the economy and trade.

\item \textbf{What are the quantitative implications of this model?} I then compute and calibrate the model with a key featuring being that the model matches the spatial distribution of economic activity in the data. I then perform several counterfactuals to illustrate the mechanics of the model and how the gains from trade vary across rich and poor households.
\end{enumerate}

\section{Trade Elasticities and the Gains from Trade}

This section focuses on the decentralized equilibrium and works towards understanding core trade outcomes | trade elasticities (Proposition \ref{prp:GET}) and the gains from trade (Proposition \ref{prp:gains-trade}) and how micro-level heterogeneity determines them. I then contrast these results with how elasticities behave in the efficient allocation (Proposition \ref{prp:gains-efficient-allocation}) and the $\log$ preference case when micro-level heterogeneity does not affect aggregate trade outcomes (Corollary \ref{prp:seperation}).

\subsection{Trade Elasticities}\label{sec:trade-elasticity}

My definition of the trade elasticity is the partial equilibrium response of imports from $j$ relative to domestic consumption due to a permanent change in trade costs.\footnote{Because the aggregate distribution of households will adjust|even with prices fixed|the elasticities that I derive are in a sense ``short-run'' elasticities.} By partial equilibrium, I mean that wages, interest rates, and the distribution of agents are fixed at their initial equilibrium values. This is consistent with the definition of the trade elasticity in say, \citet{arkolakis2012new} or \citet{simonovska2014elasticity}. By permanent, I mean that the change in trade costs is for the indefinite future and that households correctly understand this.

Given this discussion, my mathematical definition of the aggregate trade elasticity is
\begin{align}
\theta_{ij} = \frac{\partial M_{ij} / M_{ij}}{\partial d_{ij} / d_{ij}}  - \frac{\partial M_{ii} / M_{ii}}{\partial d_{ij} / d_{ij}}.
\label{eq:def_trade_elasticity}
\end{align}
Then working from the definition of imports in (\ref{eq:imports}), Proposition \ref{prp:GET} connects the aggregate trade elasticity with micro-level behavior:

\begin{prp}[\textbf{The HA Trade Elasticity}] \label{prp:GET} The trade elasticity between country $i$ and country $j$ is:
{\footnotesize
\begin{align}
\theta_{ij} = 1 + \int_{z,a} \bigg \{ \theta_{ij}(a,z)^{I} + \theta_{ij}(a,z)^{E} \bigg \}\omega_{ij}(a,z)da \ dz - \int_{z,a} \bigg \{ \theta_{ii,j}(a,z)^{I} + \theta_{ii,j}(a,z)^{E} \bigg \}\omega_{ii}(a,z)da \ dz
\label{eq:trade-elasticity}
\end{align}
}which is an expenditure-weighted average of micro-level elasticities. The micro-level elasticities are decomposed into an intensive margin and extensive margin
{\footnotesize
\begin{align}
\nonumber
\theta_{ij}(a,z)^{I} = \frac{\partial c_{i}(a,z,j)/ c_{i}(a,z,j)}{\partial d_{ij} / d_{ij}}, \ \ \ \ \ \ \theta_{ij}(a,z)^{E} = \frac{\partial \pi_{ij}(a,z) / \pi_{ij}(a,z)}{\partial d_{ij} / d_{ij}}, \ \ \ \
\end{align}
}
and the expenditure weights are defined as
{\footnotesize
\begin{align}
\nonumber
\omega_{ij}(a,z) = \frac{p_{ij}c_{i}(a,z,j)\pi_{ij}(a,z) \lambda_{i}(a,z) L_i}{M_{ij}}.
\end{align}
}
and the notation $\theta_{ii,j}^I,  \  \theta_{ii,j}^E $ represents how home choice probabilities on the intensive and extensive margin respond to the $ij$ change in trade costs.
\end{prp}

Proposition \ref{prp:GET} says that the aggregate trade elasticity is an expenditure weighted average of micro-level trade elasticities. And these elasticities are decomposed into two components: an intensive margin trade elasticity $\theta_{ij}(a,z)^{I}$ which is the change in spending by a household on variety from $j$ and an extensive margin trade elasticity $\theta(a,z)_{ij}^{E}$ reflecting how households substitute across varieties. And this is all relative to how these margins adjust home choices given the change in $j$, hence, the subscripts $ii,j$ in the second part of equation (\ref{eq:trade-elasticity}).

Proposition \ref{prp:GET} is derived only off the aggregation of imports at the micro level|no market clearing, functional forms, etc. It's essentially an identity that could be applied to any model. The next step inserts my model of household behavior. From the household's budget constraint, I can say more about the intensive margin elasticity. Then with the Type 1 extreme value assumption and the household's problem, I can say more about the extensive margin elasticity.

\textbf{The Intensive Margin Elasticity.} The intensive margin elasticity is about how do quantities change, conditional on a choice. Starting from the budget constraint in (\ref{eq:trade-budget-constraint}) I express the intensive margin elasticity as:
\begin{align}
\underbrace{\frac{\partial c_{i}(a,z,j)/ c_{i}(a,z,j)}{\partial d_{ij} / d_{ij}}}_{\theta_{ij}(a,z)^{I}} &= \bigg [-\frac{\partial g_{i}(a,z,j)/ p_{ij}c_{i}(a,z,j)}{\partial p_{ij}/ p_{ij}} - 1 \bigg ]\frac{\partial p_{ij}/p_{ij}}{\partial d_{ij}/ d_{ij}} ,
\label{eq:intensive-margin}
\end{align}
where $g_{i}(a,z,j)$ is the policy function mapping states into asset holdings next period $a'$.

The inside bracket of Equation (\ref{eq:intensive-margin}) connects the intensive margin elasticity with the household's savings decision, i.e., how it adjusts its wealth relative to expenditure when prices change.\footnote{Outside of the bracket in (\ref{eq:intensive-margin}) is how prices change with trade costs which is also known as ``pass-through.'' In the competitive environment here, it is always one, even though there is an is an active super-elasticity in the background. In the non-competitive environment of \citet{p-iq}, the super-elasticity matters for price responses and pass-through deviates from one.} A way to think about (\ref{eq:intensive-margin}) is that it answers the question: If a household faced with lower prices, how much would go to extra consumption, how much to savings? And this division of resources determines the intensive margin elasticity.

Heterogeneity matters here because this division is not invariant to a household's state $a, z$ and it works through their incentives to save. For example, if a households is constrained, assets can't adjust, and the intensive margin becomes $-1$. In contrast, wealthy households save some stuff from a reduction in prices and the intensive margin for these households will be less than one in absolute value. The result is that poor households are more price sensitive than rich households | on the intensive margin | and the mechanism works through their savings motives.

\textbf{The Extensive Margin Elasticity.} The extensive margin elasticity is about how households substitute across varieties. The elasticity of the choice probability with respect to a change in trade costs is
\begin{align}
\underbrace{ \frac{\partial \pi_{ij}(a,z) / \pi_{ij}(a,z)}{\partial d_{ij} / d_{ij}} }_{\theta_{ij}(a,z)^{E}} &=
\frac{1}{\sigma_{\epsilon}}\frac{\partial v_{i}(a,z,j)}{\partial d_{ij}/d_{ij}}  - \frac{\partial \Phi_{i}(a,z) / \Phi_{i}(a,z)}{\partial d_{ij}/d_{ij}}
\label{eq:extensive-margin}
\end{align}
The second term is how the value of all variety options change (recalling the definition of $\Phi_{i}(a,z)$  in \ref{eq:big-phi}). This is very much similar to how CES models behave except that the price-index-like term is state $a,z$ specific. The leading term is how the choice specific value function changes multiplied by the taste shock parameter.\footnote{As I show in the Appendix, a third term reflecting the effect of borrowing constraints would also be here, but via envelope theorem type arguments, they zero out for small changes in trade costs.} In other words, how elastic or the extensive margin is depends on how much more valuable choice $j$ becomes.

Now assume the number of countries is large. Then the effects on the $\Phi$ term are negligible. Thus the second term in (\ref{eq:extensive-margin}) is negligible. Moreover, because the future impacts in the value function are just functions of $\Phi$, the only non-negligible term in the value function moving around is the effect of the change in utility today and, hence:
\begin{align}
\theta_{ij}(a,z)^{E} \approx -\frac{1}{\sigma_{\epsilon}}\bigg[u'(c_{i}(a,z,j))c_{i}(a,z,j)\bigg]. \label{eq:extensive-margin-large}
\end{align}
The term in the brackets is a core piece of this paper and it shows up repeatedly. Mathematically, this term is the semi-elasticity of utility with respect to a percent change in consumption.\footnote{If this is not clear, note that this semi-elasticity is $ \partial u(c) / ( \partial c / c) = u'(c)c $.} So it determines how many more incremental utils does a household get, given a percent change in consumption. The intuition as to why this matters for the extensive margin elasticity is that if a household receives a lot of utils, on the margin, from the change in trade costs, then the substitution to that variety is stronger. In contrast, if that variety does not yield a lot of utils, on the margin, than substitution is weak.

How does this elasticity depend upon a household's circumstances? Differentiate (\ref{eq:extensive-margin-large}) with respect to assets. The thought experiment here is if a household was a bit wealthier how much more elastic would the household be:
\begin{align}
\frac{\partial (u'(c_{i}(a,z,j))c_{i}(a,z,j))}{\partial a} = u'(c_{i}(a,z,j))\times \mathbf{MPC}_{i}(a,z,j) \times \bigg[-\gamma_{i}(a,z,j) + 1\bigg], \label{eq:elasticity-mpc}
\end{align}
where $\mathbf{MPC}_{i}(a,z,j)$ is the household's marginal propensity to consume and $\gamma_{i}(a,z,j)$ is the Arrow-Pratt measure of relative risk aversion. With constant relative risk aversion (CRRA) preferences then $\gamma_{i}(a,z,j)$ becomes a constant $\gamma$ and log preferences is when $\gamma = 1$. Equation (\ref{eq:elasticity-mpc}) implies that if $\gamma > 1$, then poor, high marginal utility households who are also likely high MPC households are \emph{more elastic relative} to rich households on the extensive margin.

Intuitively, what this means is given a price reduction on some variety, it delivers a lot of utility | on the margin | for poor households. So this induces strong substitution into that variety by poor households. For rich households a price reduction for delivers little incremental gain in utility and, thus, this induces weak substitution into that variety.

Two more points about this that foreshadow several results. It makes clear the specific role that preferences play. For example, with $\log$ preferences the semi-elasticity of utility with respect to a percent change in consumption is always one ,i.e., the term in brackets in (\ref{eq:extensive-margin-large}) is one. Even though market incompleteness generates heterogeneity in marginal utility, a percent change in consumption delvers the same change in utils for rich and poor households alike (this can be seen also in (\ref{eq:elasticity-mpc})). Then rich and poor households substitute on the extensive margin in the same way.

Second, equation (\ref{eq:extensive-margin-large})  mimics the trade elasticity expression in the socially optimal, efficient allocation in equation (\ref{eq:eff-trade-elasticity}). In the efficient allocation, the planner's elasticity reflects the incremental social gain in utils from the change in trade costs. What is different in (\ref{eq:extensive-margin-large}), is that households' private valuations associated with the change in trade costs differ and, thus, they substitute differently given their own specific circumstances (outside of the knife edge case of $\log$ preferences).  Here, heterogeneity in elasticities are a symptom of a conflict between social and private valuations of the change in trade costs.

\begin{figure}[t!]
\centering
\begin{subfigure}{.45\textwidth}
\centering
\centering{\includegraphics[scale = .37]{./figures/micro-elasticity.pdf}}
\caption{Household-Level Elasticities, $-\theta_{ij}(a,z)$}\label{fig:micro-elasticity}
\end{subfigure}
\begin{subfigure}{.45\textwidth}
\centering
\centering{\includegraphics[scale = .37]{./figures/trade-share.pdf}}
\caption{Household-Level Import Shares}\label{fig:micro-trade}
\end{subfigure}
\end{figure}

Figure \ref{fig:micro-elasticity} and \ref{fig:micro-trade} illustrates how this works in a two country economy. Figure \ref{fig:micro-elasticity} plots the absolute value of the trade elasticity (intensive and extensive margin) by household state (assets are on the x-axis, productivity state on the y-axis) and the borrowing constraint $\phi$ is in the south-west corner. This shows is how the trade elasticity systematically varies with assets and income: Poor households|especially those near the borrowing constraint|are very price elastic with a trade elasticity of around $-10$. Richer households are less price elastic with this elasticity declining towards $-3.$

This pattern of trade elasticities has a strong intuitive feel and there is evidence in support of it.\footnote{This idea goes back to \citeapos{harrod1936trade} ``Law of Diminishing Elasticity of Demand'' that says that price sensitivity declines with income which is not to be confused with Marshall's Second Law of Demand which I discuss later.}. This result comes out of estimates in \citet{berry1995automobile} and in a more macro context, \citet{nakamura2010accounting}. \citet{sangani2022markups} is a recent paper that provides evidence in support of this fact from the Kilts-Nielsen data set. The evidence in \citet*{auer2022unequal} most closely relates to the patterns in Figure \ref{fig:micro-elasticity} with poorer households having higher price elasticities; \citet*{colicev2022impact} finds similar results.

One more implication of this result: because rich and poor households face the same prices, differences in elasticities lead to different expenditure shares. Figure \ref{fig:micro-trade} illustrates this point by plotting expenditure on the foreign good relative to total expenditure. Because of trade costs and symmetry across countries, the home good is the relatively cheaper good. Thus, poor, high-price-elastic households spend more on the cheap home good versus the expensive foreign good. In fact, for those near the borrowing constraint in this example, it's near zero. This pattern appears counterfactual per the evidence of \citet{borusyak2021distributional} and, hence, it motivates my introduction of non-price product attributes (quality) in the quantitative application to match micro-level expenditure shares and elasticities.

\subsection{The Gains from Trade}

In this section I compute how welfare changes due to a change in trade costs. The purpose here is to illustrate mechanics and where the gains from trade arise from. To that end, I derive these gains across steady states where I'm thinking a situation where the change is small and there is an immediate jump to the new steady state.  Unlike the trade elasticity, I take total derivatives that encompass general equilibrium changes in wages and interest rates.

The analysis proceeds in several steps before stating the main result in Proposition \ref{prp:gains-trade}. First, I focus only on country $i$ and study a change in trade costs $d_{ij}$. I then start from the top down with the following social welfare function:
\begin{align}
W_{i} = \int_{z} \int_{a}  v_{i}(a,z) L_i \lambda_{i}(a,z)
\label{eq:social-welfare}
\end{align}
where $v_{i}(a,z)$ is a households ex-ante value function in country $i$, with states $a,z$. The total change in total welfare is
\begin{align}
\frac{\mathrm{d} W_{i}}{\mathrm{d} d_{ij} / d_{ij}} \approx \int_{z} \int_{a} \bigg \{ \frac{\mathrm{d} v_i(a, z)}{\mathrm{d} d_{ij} / d_{ij}}  + v_{i}(a,z) \frac{\mathrm{d} \lambda_{i}(a,z)/ \lambda_{i}(a,z)}{\mathrm{d} d_{ij} / d_{ij}}  \bigg \} L_i \lambda_{i}(a,z) \ da \ dz.
\label{eq:social-welfare-change}
\end{align}
where the $\approx$ symbol reinforces that this is a heuristic, thinking about a small change and there is an immediate jump to the new steady state.

At a high-level, (\ref{eq:social-welfare-change}) illustrates that the gains from trade come through two forces. The first component reflects changes in household-level welfare. So conditional on a distribution of households across states, this computes if households are better or worse off. The second component of (\ref{eq:social-welfare-change}) is about reallocation. It says: take the old $v$'s and compute how the change in the distribution (that arise because of behavioral responses of households) effects social welfare. So does trade make it more or less likely that households are in ``good'' parts of the distribution.

The key issue is how household-level welfare changes. Here, I make the use of the observation in Equation (\ref{eq:log_sum-home}) that I can express the ex-ante value function in only in terms of home choice $ii$ values and then recursively push forward. In other words, I can compute the change in the ex-ante value function \emph{as if} the household only consumed the home good for the infinite future. Appendix \ref{apx-sec:gains-trade} provides the details and leads to the following expression:
{\small
\begin{align}
\frac{\mathrm{d} v_i(a, z)}{\mathrm{d} d_{ij} / d_{ij}} =& \underbrace{-\sigma_{\epsilon} \frac{\mathrm{d} \pi_{ii}(a,z) / \pi_{ii}(a,z)}{\mathrm{d}d_{ij} / d_{ij}}}_{A(a,z)} \label{eq:welfare-vterms} \\
\nonumber \\
& + \underbrace{u'(c_{i}(a,z,i)) \bigg[ \frac{\mathrm{d} w_{i} / p_{ii}}{\mathrm{d} d_{ij} / d_{ij}}z  +  \frac{\mathrm{d} R_{i} / p_{ii}}{\mathrm{d} d_{ij} / d_{ij}} a  \bigg]}_{B(a,z)} \nonumber \\
\nonumber \\
& + \underbrace{\bigg \{- \frac{u'(c_{i}(a,z,i))}{p_{ii}} + \beta \mathbb{E}_{z'} \bigg [-\sigma_{\epsilon} \frac{\partial \pi_{ii}(a',z') / \pi_{ii}(a',z')}{\partial a'} + \frac{u'(c_{i}(a',z',i))R_{i}}{p_{ii}} \bigg ] \bigg \}\frac{\mathrm{d} g_{i}(a,z,i)}{\mathrm{d} d_{ij} / d_{ij}}}_{C(a,z)} \nonumber \\
\nonumber \\
& + \beta \mathbb{E}_{z'} \bigg \{\underbrace{-\sigma_{\epsilon} \frac{\mathrm{d} \pi_{ii}(a',z') / \pi_{ii}(a',z')}{\mathrm{d}d_{ij} / d_{ij}}}_{A(a',z')} +  \underbrace{u'(c_{i}(a',z',i)) \bigg[ \frac{\mathrm{d} w_{i} / p_{ii}}{\mathrm{d} d_{ij} / d_{ij}}z'  +  \frac{\mathrm{d} R_{i} / p_{ii}}{\mathrm{d} d_{ij} / d_{ij}} a' \bigg]}_{B(a',z')} \ \  \ldots \nonumber
\end{align}
}Let me walk through the interpretation of each term:


\textbf{Gains from substitution: $A(a,z)$} The $A(a,z)$ term here is a household-specific gains from substitution term and is summarized by the change in the home choice probability and the dispersion parameter on tastes. The change in the home share summarizes two forces: (i) how exposed a household is to the change through the choice probabilities and then (ii) elasticities. To see this, define $\bar{\theta}(a,z) ^E_{ij',j}$ as the extensive margin, cross-price elasticity, in total derivative form (and it's derivation follows what is done in \ref{eq:extensive-margin}). As I show in the Appendix \ref{apx-sec:gains-trade}, the change in the home share can be expressed as:
\begin{align}
-\sigma_{\epsilon} \frac{\mathrm{d} \pi_{ii}(a,z) / \pi_{ii}(a,z) }{\mathrm{d} d_{ij} / d_{ij}} &= \sigma_{\epsilon} \sum_{j'} \pi_{ij'}(a,z) \times \bigg[ \bar{\theta}(a,z) ^E_{ii,j} - \bar{\theta}(a,z) ^E_{ij',j}\bigg], \\
\nonumber \\
&\approx
-\sigma_{\epsilon} \times \pi_{ij}(a,z) \times \bar{\theta}(a,z) ^E_{ij,j}.
\label{eq:gains-substitution}
\end{align}
In words, the top line says that the change in home choice probability is equivalent to a weighted average of relative cross-price elasticities with the weights being the choice probabilities.

The next line assumes that all cross-terms are small, then the gains from substitution depend upon the initial exposure of a household to market $j$ and their own-price elasticity (the total derivative analog to (\ref{eq:extensive-margin}). And because the own-price elasticity is intimately connected to marginal utility of consumption, the elasticity effect picks up the intuitive idea that one aspect of the gains from trade is a household's individual valuation of the price reduction, in addition to the household's exposure.

This expression also connects with two related papers. \citet{borusyak2021distributional} (following an approach dating back to at least \citet{deaton1989rice}) consider an environment where, to a first order, only exposure matters, similar to the exposure term in Equation (\ref{eq:gains-substitution}). \citet{auer2022unequal} work out second order effects with non-homothetic CES preferences and additional effects from elasticities show up, similar to the elasticity term in (\ref{eq:gains-substitution}). Here both are present in a first-order formulation. The interpretation here has a different flavor than articulated in \citet{auer2022unequal}. Because elasticities are intimately connected with the marginal utility of consumption, this relationship is saying that the reason high elasticity households gain (conditional on exposure) more, is because the price reduction from trade is more valuable, on the margin.

\textbf{Gains from factor prices: $B(a,z)$} The second $B(a,z)$ terms is essentially how a reduction in trade costs affects factor prices | the wage relative to the price of the home good and the interest rate relative to the price of the home good. And these effects are all valued at that household's marginal utility of consumption. I can simplify this term in two ways: in the perfect competition world I consider $\frac{w_i}{p_{ii}} = A_{i}$ from (\ref{eq:marginal-product}) and, thus, $\frac{\mathrm{d} w_{i} / p_{ii}}{\mathrm{d} d_{ij} / d_{ij}} = 0$, so households are perfectly ``hedged'' from any effect of trade on labor earnings.

The second simplification is that $\frac{\mathrm{d} R_{i} / p_{ii}}{\mathrm{d} d_{ij} / d_{ij}} = \frac{\mathrm{d} R_{i} / w_{i}}{\mathrm{d} d_{ij} / d_{ij}}$ and so the $B(a,z)$ term becomes
\begin{align}
B(a,z) = u'(c_{i}(a,z,i)) \times  a \times \frac{\mathrm{d} R_{i} / w_{i}}{\mathrm{d} d_{ij} / d_{ij}}.
\end{align}
Two issue that this term presents (i) how does the ratio of the gross interest rate relative to the wage rate change and (ii) how exposed is the household to this change. Unlike labor earnings, households are not perfectly hedged against these changes and because households can have negative positions in $a$ changes in relative factor prices have distributional effects.  For example, if a trade liberalization leads to an increase in $R / w$, net debtors suffer since their terms to borrow deteriorated, while net savers benefit. This force is very much analogous to forces identified in \citet{auclert2019monetary}.

\textbf{Gains from changes in asset holdings: $C(a,z)$} The third term which I'm labeling as $C(a,z)$ is about changes in asset holdings. For the small / local changes that I'm considering it should zero out, but for larger changes this term could be relevant. Let me expand upon this.

First, notice that the inside the bracket term is the Euler Equation from (\ref{eq:euler_equation-home}) and its multiplied by the change in policy function. Now if the household is unconstrained, the inside the bracket term is zero as there is no gain through changes in asset behavior. Asset holdings are already chosen optimally so that margins are equated, thus, on the margin any benefit of lower trade costs on changes in asset behavior is zero. This inside the bracket term may not be zero because of borrowing constrained households. However, the bracket is multiplied by the change in the asset policy function. What this picks up is that if the household is constrained, then assets can't adjust so the outside term is zero and, thus, overall the second term is zero.

Via this logic, the only people that benefit (and contribute to social welfare) through these are those on the margin between constrained and not-constrained. But if they are on the margin between being constrained and not-constrained, then they are on their Euler equation anyways.

Finally, these repeat themselves into the expected future, appropriately discounted. Proposition \ref{prp:gains-trade} summarizes the result below.

\begin{prp}[\textbf{HA Gains from Trade}] \label{prp:gains-trade} The welfare gains from trade are given by
{\footnotesize
\begin{align}
\frac{\mathrm{d} W_{i}}{\mathrm{d} d_{ij} / d_{ij}} = \int_{a}\int_{z}  \bigg \{ \frac{\mathrm{d} v_i(a, z)}{\mathrm{d} d_{ij} / d_{ij}}  + v_{i}(a,z) \frac{\mathrm{d} \lambda_{i}(a,z)/ \lambda_{i}(a,z)}{\mathrm{d} d_{ij} / d_{ij}}  \bigg \} \lambda_{i}(a,z).
\nonumber
\end{align}
}which reflects the change in household level gains and how the distribution of households changes. Household level gains are given by
{\footnotesize
\begin{align}
\nonumber
\frac{\mathrm{d} v_i(a, z)}{\mathrm{d} d_{ij} / d_{ij}} = \mathbb{E}_{z} \sum_{t = 0}^{\infty} \beta^{t} \bigg \{ A(a_{t},z_{t}) + B(a_{t},z_{t}) + C(a_{t},z_{t}) \bigg \}
\end{align}
}where each term represents:
\begin{itemize}
\item Gains from substitution: $A(a_{t},z_{t}) = -\sigma_{\epsilon} \frac{\mathrm{d} \pi_{ii}(a_{t},z_{t}) / \pi_{ii}(a_{t},z_{t})}{\mathrm{d}d_{ij} / d_{ij}}$.

\item Gains from factor prices: $B(a_{t},z_{t}) = u'(c_{i}(a_{t},z_{t},i))\frac{\mathrm{d} R_{i}/p_{ii}}{\mathrm{d} d_{ij} / d_{ij}}a_{t}$

\item Gains from changes in asset holdings:
{\footnotesize
\begin{align}
\nonumber
C(a_{t},z_{t}) = \bigg \{- \frac{u'(c_{i}(a,z,i))}{p_{ii}} + \beta \mathbb{E}_{z'} \bigg [-\sigma_{\epsilon} \frac{\partial \pi_{ii}(a',z') / \pi_{ii}(a',z')}{\partial a'} + \frac{u'(c_{i}(a',z',i))R_{i}}{p_{ii}} \bigg ] \bigg \}\frac{\mathrm{d} g_{i}(a,z,i)}{\mathrm{d} d_{ij} / d_{ij}}
\end{align}}
which is zero for small changes.
\end{itemize}
\end{prp}

\subsection{Elasticities and Gains in the Efficient Allocation}\label{sec:efficient}

One issue behind the results above is market incompleteness. Households are imperfectly insured against the risks they face, this leads to heterogeneity in the marginal utility of consumption and, in turn, heterogeneity in price sensitivity, expenditure patterns, and the gains from trade. Building on work in from my related paper in \citet{waughoptimal}, I contrasts the previous results with the elasticities and gains from trade in an allocation where a social planner can overcome market incompleteness. Appendix \ref{apx-sec:efficient} provides a self-contained discussion of the planning problem and derivation of the results.

The starting point is a utilitarian social welfare function which is analogous to (\ref{eq:social-welfare}). In \citet{waughoptimal}, I fully characterize a Planner's choice of consumption allocations $c_{i}(z, j, t)$ and choice probabilities $\pi_{ij}(z,t)$ for all $i,j$ pairs, $z$ states, and dates $t$ to maximize social welfare.\footnote{In a more general, but static setting, \citet{mongey-waugh-2} characterize complete markets and efficient allocations in additive, random utility discrete choice models. A precursor to my characterization of the gains from trade under efficiency are the efficiency results in \citet*{lagakos2023welfare}.} Given a characterization of the optimal allocation, I can compute trade elasticities and the welfare gains from a change in trade costs and study how welfare changes across the two stationary allocations.\footnote{Unlike the previous section, the move across stationary allocations is no consequence as there is no moving aggregate state variable, so the jump across stationary equilibrium is instantaneous.}

Proposition \ref{prp:gains-efficient-allocation} describes the result, Appendix \ref{apx-sec:efficient} works out the details.

\begin{prp}[\textbf{Trade Elasticities and Welfare Gains in the Efficient Allocation}]\label{prp:gains-efficient-allocation} The elasticity of trade to a change in trade costs between $ij$ in the efficient allocation is:
\begin{align}
\theta_{ij} =  -\frac{1}{\sigma_{\epsilon}} \bigg [ u'(c_{i}(j)) c_{i}(j) \bigg]. \label{eq:eff-trade-elasticity}
\end{align}
And the welfare gains from a reduction in trade costs between $i,j$ are
\begin{align}
\frac{\mathrm{d} W }{\mathrm{d} d_{ij} / d_{ij}} &= \frac{\sigma_{\epsilon} \  \theta_{ij} \  \pi_{ij} \ L_i}{1-\beta},
\label{eq:eff-trade-gains}
\end{align}
which is the discounted, direct effect from relaxing the aggregate resource constraint. And this can be expressed as
\begin{align}
= -\sigma_{\epsilon} \times \frac{\mathrm{d} \pi_{ii} / \pi_{ii}}{\mathrm{d} d_{ij} / d_{ij}} \times \frac{L_i}{1 - \beta}.
\label{eq:eff-trade-gains-acr}
\end{align}
\end{prp}
Proposition \ref{prp:gains-efficient-allocation} highlights a couple of things. Consistent with intuition from \citeapos{eaton2002technology}, the dispersion parameter matters inversely. If $\sigma_{\epsilon}$ is small, national varieties are ``as if they are near substitutes'' and thus trade flows will respond a lot.

Very similar to the household-level extensive margin elasticity in (\ref{eq:extensive-margin-large}), the aggregate trade elasticity has a term with the marginal utility of consumption times consumption showing up. Like in the discussion above, this term matters for the elasticity in a very intuitive way|country pairs that deliver a lot of utility, on the margin, are the pairs where the planner will be most responsive to changes in trade costs.

The second part of Proposition \ref{prp:gains-efficient-allocation} summarizes the gains from trade. It says that the total change in welfare only reflects those eating that commodity $\pi_{ij} \times L_i$ which is the share of households consuming commodity $j$ times the number of households in country $i$. This is then converted into utils using the elasticity which, as discussed above, is something about the dispersion in shocks and then the rate at which utils are being delivered at current quantities.\footnote{An alternative perspective is to divide through both sides of (\ref{eq:eff-trade-gains}) by the marginal utility of consumption and then the welfare is in money metric units.} This is then discounted for the infinite future, hence the $1/ (1-\beta)$ term.

This is just the direct effect from a reduction in trade costs relaxing the resource constraint and converted to utils appropriately. Behind this result is an envelope-type argument with direct effects only mattering because I'm evaluating the change in welfare at the optimized allocation and any benefits of adjusting consumption and choice probabilities are zero|on the margin.

This result is reminiscent of \citet{AtkesonBurstein2010} who make a similar claim in the context of a model with rich firm heterogeneity. They show that the only first order effect of lower trade costs on welfare is the direct consumption effect and that indirect effects are second order. This is similar, but with household heterogeneity, by saying that, in the efficient allocation all margins are properly equated heterogeneity is irrelevant and the welfare gains only the direct benefits.

The final part of Proposition \ref{prp:gains-efficient-allocation} connects with \citet{arkolakis2012new}. As I show in the appendix, in the efficient allocation the percent change in the home choice probability exactly equals the $ij$ choice probability times the trade elasticity
\begin{align}
\frac{\mathrm{d} \pi_{ii} / \pi_{ii}}{\mathrm{d} d_{ij} / d_{ij}} = -\theta_{ij} \times \pi_{ij},
\label{eq:effecient-home-share}
\end{align}
then inserting (\ref{eq:effecient-home-share}) into  (\ref{eq:eff-trade-gains}) delivers the final line of Proposition \ref{prp:gains-efficient-allocation}. Now the form of (\ref{eq:eff-trade-gains}) is closely related to \citet{arkolakis2012new}. Interestingly, and much like in the decentralized allocation, the change in the home choice probability summarizes a lot. Moreover, now there is an equivalence between \citet{AtkesonBurstein2010}-like logic and \citet{arkolakis2012new}-style formulas.

There are two details: choice probabilities do not necessarily correspond with expenditure shares and the $\sigma_{\epsilon}$ is not the inverse of the trade elasticity However, with $\log$ the trade elasticity becomes $1 / \sigma_{\epsilon}$, choice probabilities are proportional to expenditure shares, and the correspondence between the gains from trade under efficiency and \citet{arkolakis2012new} becomes exact. The case of $\log$ has an additional implication that heterogeneity and market incompleteness does not matter for trade outcomes and I turn to this case next.

\subsection{The Case of $\log$ preferences}\label{sec:log-preferences}

The case of $\log$ preferences over the physical commodity displays some unique features. This very common) preference structure leads to an interesting result where micro-level heterogeneity, market incompleteness completely sperate from the trade side of the economy. So in this one case, trade behaves ``as if'' there were a representative agent Armington-CES consumer.

Consider the following preference structure:
\begin{align}
\tilde{u}( c_{ij,t} ) =  \log(c_{ij,t}) + \epsilon_{j,t}. \nonumber
\end{align}
There is essentially one insight and then everything follows. Examining the problem in (\ref{eq:value_fun_option}) and substituting in the households budget constraint from (\ref{eq:trade-budget-constraint}), then the observation is that the optimal $a'$ conditional on a choice $j$ is \textbf{independent} of the choice $j$. And this observation implies that choice probabilities become independent of states $a$ and $z$. Everything follows from these observations and Proposition \ref{prp:GET} and Proposition \ref{prp:gains-trade} can be applied. Corollary \ref{prp:seperation} states the result and Appendix \ref{apx-sec:log-preferences} works through this logic step-by-step.

\begin{corr}[\textbf{Separation of Trade and Heterogeneity}]\label{prp:seperation} In the dynamic, heterogenous agent trade model where preferences are logarithmic over the physical commodity: The trade elasticity is
\begin{align}
\theta = -\frac{1}{\sigma_{\epsilon}}, \nonumber
\end{align}
and trade flows satisfy a standard gravity relationship
\begin{align}
\frac{M_{ij}}{M_{ii}} = \left( \frac{  w_{j} / A_{j} }{  w_{i} / A_{i} } \right)^{\frac{-1}{\sigma_{\epsilon}}} d_{ij}^{\frac{-1}{\sigma_{\epsilon}}}, \nonumber
\end{align}
and both are independent of the household heterogeneity. And the welfare gains from trade for an individual household are
\begin{align}
\nonumber
\frac{\mathrm{d} v_i(a, z)}{\mathrm{d} d_{ij} / d_{ij}} = \frac{1}{\theta (1-\beta)} \times \frac{\mathrm{d} \pi_{ii} / \pi_{ii}}{\mathrm{d}d_{ij} / d_{ij}} \ \ + \ \
\mathbb{E}_{z} \sum_{t = 0}^{\infty} \beta^{t} \bigg \{ B(a_{t},z_{t}) + C(a_{t},z_{t}) \bigg \}
\end{align}
where the gains from substitution are (i) independent of the household heterogeneity and (ii) summarized by the trade elasticity and the change in the home choice probability and (iii) the other sources of gains are as in Proposition \ref{apx-prp:gains-trade}.
\end{corr}
The first part of this result is heterogeneity plays no role in determining the aggregate trade elasticity. Like in the CES-Armington model or \citet{eaton2002technology}, it's just about how innately substitutable national varieties are. Similarly, aggregate trade satisfies a gravity relationship which no role for household heterogeneity.

Welfare is a bit more subtle. Choice probabilities are independent of states (and proportional to expenditure shares), then applying Proposition \ref{prp:gains-trade} and inserting the expression for the trade elasticity allows me to express the gains from substitution component of Proposition \ref{prp:gains-trade} mimic the results of \citet{arkolakis2012new}. What is still present from the heterogenous agent part of the model is how changes in factor prices influence welfare.

To be honest, I found this result surprising. By looking at the choice probabilities in (\ref{eq:choice-prob}) and noting how the value functions determine choices, not period utility functions, one would suspect that the household's income fluctuations problem would shape aggregate trade outcomes. Corollary \ref{prp:seperation} shows that is not the case but that micro-outcomes and aggregate trade outcomes ``separate.''

Proposition \ref{prp:seperation} is also interesting because it generalizes the results of \citet*{anderson1987ces} and \citet{anderson1992discrete} to a far more complicated economy. They showed that in a static model with log utility and additive logit shocks, the economy behaves \emph{as if} there were a representative agent CES consumer. I recover their result, but I must emphasize the complexity of the economy at the micro-level for which this result stands|households are froward looking, face productivity and taste shocks in the presence of incomplete markets and borrowing constraints. Yet, these details don't matter when the magic of $\log$ kicks in.

\section{Calibration}

This section focuses on my approach to calibrating the model. The next two subsections discuss the preference specification, income and taste shock process, borrowing constraints and how I scale things so the model can deliver balanced growth like properties.



The final section follows the trade literature by picking country specific TFP and trade cost parameters to match bilateral trade flows. How I do this is an indirect inference procedure | ``gravity as a guide'' | to overcome the fact that my model does not admit a closed form map from trade flows to parameters as static, gravity models do. I describe this approach below.

\subsection{Preferences, Shocks, and Constraints}

Table \ref{tb-calibration} provides an overview of the non-country-specific parameters (or if they are country specific, they are all scaled in the same way). Below, I discuss each choice in turn. Some of the choices are made to keep things simple (e.g. the earnings process). Others are guided by the theoretical results above and, hence, the moments chosen appropriately.

Utility over the physical commodity is CRRA with relative risk aversion $\gamma$. This parameter is calibrated (along with the taste shock parameter) to match the price elasticities of the median and poor households in the data of \citet*{auer2022unequal}. And the correspondence between this parameter and these moments are motivated by the results in (\ref{eq:extensive-margin-large}) and (\ref{eq:elasticity-mpc}) showing how the curvature of the utility function partially shapes micro-level elasticities and how they vary across households.

On top of these preferences I do two things to ensure that micro and macro facts can be matched given that there is a sense in which preferences are non-homothetic on the extensive margin.

The first feature I introduce are household-specific quality shifters. Mechanically, I implement quality shifters by introducing a home bias term $\psi_{i}(z,i)$ which additively shifts period utility when consuming the home good $i$ and it differs by the households productivity state $z$. Appendix \ref{apx-sec:quality} provides details, but the key issue is that with the additivity, it only effects shares, not elasticities per se. To reduce $\psi$s dimensionality, I assume that it's a log-linear function of a households permanent productivity state and this function is the same across countries. The slope of this relationship is calibrated to match the fact from \citet{borusyak2021distributional}) that import expenditure shares are essentially the same between US poor (below median income) and rich (above median income) households. And as equation \ref{eq:gains-substitution} motivates, how import shares vary across rich and poor households is an important input into determining how the gains from trade vary across households.

The necessity of quality shifters relates to the discussion around Figure \ref{fig:micro-trade} | that prices and price elasticities solely determine how shares vary across households and these forces lead to a pattern of sorting with poor, high-elasticity households concentrating their expenditures on the cheapest commodities available. Quality shifters that vary with household-specific characteristics are one way to match shares, yet allow for heterogeneity in price elasticities. \citet*{berry1995automobile} make this point and it motivates their modeling of demand with interactions between attributes of the product and household characteristics; \citet{auer2022unequal} allows for this force as well in both their model and empirical specification.

The second feature I want is that things to scale and deliver a ``balanced-growth-like'' property. Specifically, the want-operator here is that if there are two countries one with high TFP and one with low TFP, elasticities (both at the micro and macro level) in the two countries are the same. The way to do this is to make the Type 1 Extreme value parameter country specific and scaled so that $\sigma_{\epsilon,i} = \sigma_{\epsilon} A_i^{1-\gamma}$ and similarly for the quality shifters above. The common component of the taste shock parameter, $\sigma_{\epsilon}$ is calibrated (along with the risk aversion parameter) to match the price elasticities of the median and poor households in the data of \citet*{auer2022unequal}.

The income shock process is set up to be a mixture of a AR(1) persistent component and an iid transitory component and this is calibrated using results from \citet*{krueger2016macroeconomics}. I use their exact parameter values at an annual frequency. It is the same across countries.

The borrowing constraint is set in the following way. I scale it by a country's autarky level of average real labor income. Then it is set so that a household can borrow up to fifty percent of it's autarky level of income. The scaling here is done (again) to deliver a balanced-growth-like property of the model so a households debt capacity is invariant to a country's autarky level of income. The precise number of fifty percent seems reasonable to me, but as a check I show that the marginal propensities to consume and how they vary across households is consistent with the evidence of \citet{kaplan2022marginal}.

\begin{table}[t]
\small
\begin{center}
\refstepcounter{table}
\setlength {\tabcolsep}{4.5mm}
\renewcommand{\arraystretch}{1.60}\label{tb-calibration}
\begin{tabular}[t]{l c l}
\multicolumn{3}{c}{{\normalsize\textbf{Table \ref{tb-calibration}: Preferences, Shocks, and Constraints | Calibrated Parameters}} }
\\\hline \hline
Description & Value & \multicolumn{1}{c}{Target}\\
\cmidrule(lr){1-1} \cmidrule(lr){2-2} \cmidrule(lr){3-3}
Discount Factor, \ $\beta$                          & $0.92$ & \phantom{\} } Global Interest Rate of $1\%$ \\
CRRA parameter, \ $\gamma$                          & $1.45$ & \multirow{2}{*}{\Bigg \} Micro elasticities of {\small \citet{auer2022unequal}} }\\
Type One E-V parameter, \ $1 / \sigma_{\epsilon}$    & $3.0$ & \\
Slope of Quality Shifter, \ $\psi_{ii}(z)$          & $0.60$ & \phantom{\} } Micro moments of {\small \citet{borusyak2021distributional} } \\
Borrowing Constraint \ $\phi_{i}$                   & --- & \phantom{\} } $50\%$ of $i$'s autarky labor income \\
Income Process on \ $z$                             & --- & \phantom{\} } {\small \citet*{krueger2016macroeconomics}} \\
\hline
\end{tabular}
\\[0.5ex]
%\parbox{5.95in}{\footnotesize \textbf{Note:} }
\end{center}
\end{table}

All the quantitative results I show are with the case of financial globalization. In this case there is one interest rate clearing the global asset market. Households in all countries have the same discount factor and I calibrate the discount factor so the equilibrium world interest rate is 1 percent.

\subsection{Using Gravity as a Guide to Match Trade Data}

In analyzing how changes in trade frictions affect outcomes, I to want the model to match the spatial distribution of economic activity in the data. The calibration challenge is that model does not admit a gravity representation that allows researchers to invert trade frictions and productivity levels from trade flows as done in \citet{eaton2002technology}. Similarly, the model does not admit the use of exact-hat algebra which allows the research to construct counterfactuals without the knowledge of primitives like trade frictions or productivity (see, e.g., \citet{costinot2014trade} or the dynamic extension in \citet*{caliendo2015trade}).

My calibration strategy is to use the gravity regression as a guide in an indirect inference procedure where I estimate parameters of the model so that the regression coefficients from a standard gravity regression run on my model's data match the coefficients when the same regression is ran on the data. Here are the details.

The bilateral trade flow data that I use are from \citet{eaton2002technology}. The 19 countries in this data set is a nice size to do what I want to do in about an afternoon. In the 19 country model, the parameters I need to choose are $19 - 1$ country-specific TFP parameters (the $A_i$s with the minus because one normalization is free) and then $(19-1) \times (19-1)$ trade costs (with the minus one since the $ii$ trade costs is normalized to one) to infer. This leaves me under-identified with  only $(19-1) \times (19-1)$ bilateral trade shares and $19 -1$ TFP parameters.

\textbf{Step 0.} I reduce the number of parameters to estimate by placing a restriction on trade costs such that they are a function of observable data. Specifically, I assume that trade costs take the form as in \citet{eaton2002technology} with
\begin{align}
\log d_{ij} = d_{k} + b + l + e_{h} + m_{i},
\label{eq:trade-cost-function}
\end{align}
where trade costs are a logarithmic function of distance, where $d_k$ with $k = 1,2,...,6$, is the effect of distance between country $i$ and $j$ lying in the $k$-th distance interval.\footnote{Intervals are in miles: $[0,375)$; $[375,750)$; $[750,1500)$; $[1500,3000)$; $[3000,6000)$; and $[6000,\mbox{maximum}]$. } The $b$ term is the effect of a shared border in which $b =1$ if country $i$ and $j$ share a border and zero otherwise. Similarly $l$ is a dummy variable if country's $i$ and $j$ share a language, and $e_{h}$ represent two dummy variables for different indicators of European integration. The final part is an importer fixed effect that shifts trade costs up or down depending upon the identity of the importer.

At this point, I've reduced the parameter space to the  coefficients on the trade cost function rather than the complete matrix of trade costs and then the TFP terms.

\textbf{Step 1.} The next step is to run the following gravity regression on the data
\begin{align}
\log \left( {\frac{M_{ij}}{M_{ii}}} \right) = {im_{i}} + {ex_{j}} + {d_{k}} + {b} + {l} + {e_{h}} + \delta_{ij},
\label{eq:gravity-data}
\end{align}
which projects imports between country $i$ and $j$ ( normalized relative to domestic expenditures ) on an importer effect, exporter effect, and then the gravity variables relating to distance, border, language, etc. Finally there is an error term $\delta_{ij}$ that reflects other factors not in this specification.

This is the canonical representation of trade flows | the gravity model. In a standard Armington-CES, \citet{eaton2002technology}, or \citet{melitz2003impact} style model, the importer effects and exporter effects have specific interpretations. And given the point estimates from (\ref{eq:gravity-data}), productivity and the importer fixed effects on the trade cost function are easily recovered. In my model, this is not the case. However, the idea is to use the point estimates from (\ref{eq:gravity-data}) as moments for my model to match. The next step constructs model analogs to (\ref{eq:gravity-data}).

\textbf{Step 2.} To construct model analogs to (\ref{eq:gravity-data}), I guess TFP parameters and coefficients on the trade cost function in (\ref{eq:trade-cost-function}). Define this parameter vector as $\Theta$.

Given $\Theta$, I compute an equilibrium of the world economy. This amounts to: (i) solving for households' dynamic problems|in each country (ii) constructing the stationary distribution of wealth and expenditure patterns|in each country (iii) aggregating and then (iv) finding a vector of prices so goods markets and financial markets clear world wide.

Once I find an equilibrium, I run the same regression as in (\ref{eq:gravity-data}) on the model generated data.\footnote{One approach, that works well with good guesses and takes about an hour (rather than 6-12), is to have the solver simultaneously look for prices that clear markets and coefficients such that the trade moments are matched.} As some notation, the model constructed moments are defined as, e.g., $im_{i}(\Theta)$ which is the importer effect estimated on model generated data under the parameter vector $\Theta$.

\textbf{Step 3.} The final step constructs moment conditions which provide the foundation for estimation / calibration. Define $\mathbf{y}(\Theta)$ as a set of moments conditions comparing the point estimates from the data with the point estimates from the model under the parameter vector $\Theta$. For example, ${im_{i}} - im_{i}(\Theta)$.

My estimation procedure is based on the moment condition
\begin{align}
E\left[\mathbf{y}(\Theta_o)\right] = 0,
\end{align}
where $\Theta_o$ is the true value of $\Theta$. Thus, my method of moments estimator is:
\begin{align}
\hat{\Theta} = \arg\min_{\Theta} \left[\mathbf{y}(\Theta)'\ \mathbf{y}(\Theta)\right], \label{eq:smm-condition}
\end{align}
At a mechanical level, finding the minimum to (\ref{eq:smm-condition}) amounts to returning to \textbf{Step 2.} each time and smartly updating parameter guess for $\Theta$. One of the nice features of this set-up and the dimensionality reduction that I did, is that now this is an exactly identified problem and standard root-finding techniques can be applied to update $\Theta$ and a minimum found.

\subsection{Calibration Results}

\subsubsection{Shares, Elasticities, and MPCs}

The second column of Table \ref{tb-calibration} reports the calibrated parameters associated with preferences, shocks and constraints. The resulting parameter values for the CRRA risk aversion parameter and Type 1 Extreme Value variance parameter are very plausible. The risk-aversion parameter is 1.45, seemingly not far from $\log$, and well within standard benchmarks in the macro-literature. The Type 1 Extreme Value parameter is 3.0 which is plausible in the sense that in the $\log$ model this value implies a trade elasticity of 3.0. This is low, but in the range of estimated aggregate trade elasticities.

Figure \ref{fig:quant-micro-elasticity} reports the resulting elasticities and how they vary across the distribution of expenditure for US households in the model. These are right in the ballpark of \citet{auer2022unequal}. If anything, they slightly understate the heterogeneity in elasticities with rich households being more elastic in my model.

Figure \ref{fig:quant-shares} reports how household-level import expenditure shares vary across the income distribution. The calibration target was such that share of imports out of total expenditure for households below median income is the same as those above the median. This was met. However, there is some slight non-monotonicity with households in the middle of the distribution being slightly more exposed to trade then those at the tails of the distribution. This non-monotonicity is associated with the imperfect correlation between income and assets holdings and that quality shifters only work though income. Overall, the pattern is consistent with the facts of as found in \citet{borusyak2021distributional} and if there is any bias it is for poor households to experience less gains from trade as they are slightly less exposed relative to the median household.

\begin{figure}[t!]
\centering
\begin{subfigure}{.48\textwidth}
\centering
\centering{\includegraphics[scale = .3]{./figures/elasticity-micro.pdf}}
\caption{Trade Elasticities}\label{fig:quant-micro-elasticity}
\end{subfigure}
\begin{subfigure}{.48\textwidth}
\centering
\centering{\includegraphics[scale = .3]{./figures/expenditure-share.pdf}}
\caption{Import Expenditure Shares}\label{fig:quant-shares}
\end{subfigure}\\
\bigskip
\begin{subfigure}{.758\textwidth}
\centering
\centering{\includegraphics[scale = .3]{./figures/mpc.pdf}}
\caption{Marginal Propensities to Consume}\label{fig:quant-mpcs}
\end{subfigure}
\end{figure}

Figure \ref{fig:quant-mpcs} reports marginal propensities to consume (MPC) in the model. Recalling the discussion in Section \ref{sec:trade-elasticity} | how trade elasticities vary across households relates to MPCs in equation (\ref{eq:elasticity-mpc}).  MPCs in the model are computed by endowing households a one time, unanticipated cash transfer of 1,000 USD and then computing how consumption changes relative to the transfer. As with the elasticities, household level MPCs are aggregated across the different consumption baskets by the household's expenditure weights.

MPCs are right in the ballpark of what is typically thought plausible with the median annual MPC being a little under 0.30 implying that a household spends about 30 cents per dollar of transfer on consumption; see, e.g., \citet{kaplan2022marginal}.\footnote{With that said, this calibration achieves high MPCs essentially by having very little wealth in the economy. This model feature is consistent with the small quantity of liquid wealth observed in the US economy, but leaves out large amounts of illiquid wealth.} Not surprisingly, poorer households have substantially higher MPCs and richer households lower. And together with the second column, Table \ref{tb-micro-shares} confirms the connection between how sensitive a household is to prices and how sensitive a household is to cash transfers.

To summarize, the model is quantitatively matching salient facts about (i) poor households have higher price elasticities relative to rich households as in \citet{auer2022unequal}, (ii) that import expenditure shares between rich and poor households are similar as in \citet{borusyak2021distributional}), and (iii) is able to mimic patterns of marginal propensities to consume as seen in micro data and surveyed \citet{kaplan2022marginal}.

\subsubsection{Aggregate Trade and Trade Elasticities}

\begin{figure}[!t]
\centering{\includegraphics[scale = .45]{./figures/trade-fit.pdf}}
\caption{Bilateral Trade: Model vs. Data}\label{fig:model-fit}
\end{figure}

Figure \ref{fig:model-fit} provides a sense of model fit with respect to bilateral trade flows. The y-axis reports bilateral trade data and the x-axis reports the outcome from my model. The fit is very high, and nearly indistinguishable from, for example, how a standard trade would perform. Or the $\log$ preference model which per Proposition \ref{prp:seperation} should (and it does) operate just like a standard trade model.

Table \ref{tb-grav-est} reports another measure of fit and some of the resulting parameter values. The first column are the distance, border, etc. moments from the gravity regression in (\ref{eq:gravity-data}) (and note they exactly correspond with those in the top panel of Table 3 of \citet{eaton2002technology}). The second column reports the moments from the model. Here they exactly line up and are consistent with the argument in Figure \ref{fig:model-fit}, the fit is good and the model is replicating geographic pattern of activity seen in the data.

\begin{table}[t]
\small
\begin{center}
\refstepcounter{table}
\setlength {\tabcolsep}{5.5mm}
\renewcommand{\arraystretch}{1.50}\label{tb-grav-est}
\begin{tabular}[t]{l c c c}
\multicolumn{4}{c}{{\normalsize\textbf{Table \ref{tb-grav-est}: Estimation Results}} }
\\\hline \hline
& & \multicolumn{2}{c}{\textbf{HAT-Model}}  \\
\cmidrule(lr){3-4}
Barrier& Moment & Model Fit & Parameter \\
\hline $[0,375)$                &$-3.10 $           & $-3.10 $              & $2.35$           \\
$[375,750)$                     &$-3.67 $           & $-3.67 $              & $2.81$           \\
$[750,1500)$                    &$-4.03 $           & $-4.03 $              & $3.09$           \\
$[1500,3000)$                   &$-4.22 $           & $-4.22 $              & $3.23$           \\
$[3000,6000)$                   &$-6.06 $           & $-6.06 $              & $4.88$           \\
$[6000,\mbox{maximum}]$         &$-6.56 $           & $-6.56 $              & $5.69$           \\
Shared border                   &$\phantom{-}0.30$  & $\phantom{-}0.30$     & $0.91$  \\
Language                        &$\phantom{-}0.51$  & $\phantom{-}0.51$     & $0.87$  \\
EFTA                            &$\phantom{-}0.04$  & $\phantom{-}0.04$     & $0.98$  \\
European Community              &$\phantom{-}0.54$  & $\phantom{-}0.54$     & $0.89$  \\
\hline
\end{tabular}
\\[0.5ex]
\parbox{5.0in}{\footnotesize \textbf{Note:} The first column reports data moments the HAT-model targets. The second reports the model moments. The third column reports the estimated parameter values.}
\end{center}
\end{table}

The final column reports the primitive estimates on the trade cost function. Each value reports the level effect of being in a distance bin or sharing a border etc.  So if two countries are measured to be in the smallest distance bin and share a border, the trade cost between these two countries is $2.35 \times 0.91$ (first row times seventh row). Or if a country is in the furthest distance bin, its trade costs is 5.69.

How would this compare to a standard model? It's a bit hard since one needs to take a stand on the trade elasticity in the standard model to translate estimates in column one into levels of the trade costs. But an approach is the following: find the trade elasticity so the cost of the nearest distance bin is the same as in my model and then look at how things relate in other bins. In an \citet{eaton2002technology} world, this would correspond with an trade elasticity of about 3.6. Then, for example, one takes the moment in the first column, last distance bin and compares $\exp( - 1 / 3.6 \times  -6.56)$ vs. 5.69.

What comes out of is that closer relationship are a bit more expensive then what a constant elasticity model would predict. And the furthest destinations are meaningfully less expensive, seven and ten percent less, for the last two distance bins. This is picking up a model outcome where trade elasticities are increasing with cost. So far away destinations are relatively elastic destinations, so the cost need not be as large to deter trade.

Figure \ref{fig:bilateral-elasticities} provides an example of the of trade elasticities that come of this model. In this figure, I focus on the US and plot each bilateral trade elasticity versus the price a consumer in the US faces when importing a variety from that country. The balls represent the relative size of US imports from that destination. And these elasticities are constructed from the bottom up via the formula in (\ref{eq:trade-elasticity}).

The feature that stands out very clearly in Figure \ref{fig:bilateral-elasticities} is that trade elasticities systematically increase with price and decrease with the volume of trade.

\begin{figure}[!t]
\centering{\includegraphics[scale = .45]{./figures/us-elasticity.pdf}}
\caption{Trade Elasticities $-\theta_{us,j}$}\label{fig:bilateral-elasticities}
\end{figure}

At the micro-level, there are two opposing forces giving rise to this aggregate relationship. Per the arguments discussed above around Proposition \ref{prp:GET}, the aggregate bilateral trade elasticity reflects both household level elasticities $\theta(a,z)$s and a composition effect that works through the expenditure weights $\omega(a,z)$s. Thus, when prices increase as one moves from one source to a less competitive source, there are two competing forces at work: (i) how do micro-level elasticities change and (ii) how does composition change?

The first force is that as prices increase \emph{both} rich and poor households' elasticities increase. In other words, everyone is more elastic when contemplating a purchase from a more expensive destination. This is a force pushing the model to have elasticities \emph{increase} with price.

The second force | the composition effect| generally works in the opposite direction. As one moves from more cheaper to more expensive destinations, less price sensitive households sort into those varieties. Thus, the composition of households purchasing more expensive varieties are the rich, relatively inelastic households. This is a force pushing the model to have elasticities \emph{decrease} with price. In more standard, BLP-like settings, \citet{nakamura2010accounting} and \citet*{head2021poor} highlight this composition effect in shaping pass-through.

Which one wins? Figure \ref{fig:bilateral-elasticities} shows that the first force dominates the composition effect. One way to view this result is through the lens of \citeapos{mrazova2017not} language that demand in this model endogenously turns out to be ``subconvex'' relative to CES demand and which is equivalent to ``Marshall's Second Law of Demand.'' The endogenous part is important as it's not parameterized as in, say, Kimball Demand which has become a popular tool to allow for non-constant elasticities. Did the model have to deliver this? Per the arguments above, it's not obvious as composition effects could have dominated.\footnote{\citet{head2021poor}, using \citeapos{berry1995automobile} estimated model, illustrate that composition does indeed dominate with pass-through greater than one when heterogeneity in the valuation of product characteristics are shut down.}

There is evidence suggesting that trade elasticities conform to what comes out of my model. Both \citet{novy2013international} and \citet*{carrere2020gravity} find that proxy's for the trade elasticity are larger, the less trade there is between two countries. \citet{chen2022gravity} further confirm this idea by finding that trade cost effects are strong for small bilateral relationships weak or even zero for large trading relationships. Mapping these ideas back into outcomes from my model, a currency union between the US and Canada would likely have a small effect since this is a high volume / low elasticity relationship.

\section{The Welfare Gains From Trade}

In this section, I measure the gains from trade and study how they are distributed across households.

\subsection{Measuring Welfare}

In the previous section, I focused on what amount to level changes in utils to understand mechanisms. Here I define an equivalent variation measure that has more interpretable units and I use in the quantitative results.

As a quick refresher, equivalent variation does the following: given that some price change delivers utility level $V'$, equivalent variation asks ``at the old prices, $p_0$, how much extra income must be provided to be indifferent between $V'$ and $V$.'' To implement this in my model, define the value function of a household at base period prices as
\begin{align}
v_i(a, z ; \ p, R_{i}, w_{i}).
\end{align}
And the value function for the same states, but at counterfactual prices
\begin{align}
v'_i(a, z ; \ p', R'_{i}, w'_{i}), \label{eq:welfare-eqv-cftc}
\end{align}
where I'm evaluating this with the prices prevailing at the new steady state and, hence, there are no $t$ subscripts. I focus on across steady states, not transition paths. Those may be important, but it adds an additional computational challenge which I've decided to abstract from for now.

Given these definitions, my equivalent variation measure is a permanent, proportional increase in income (asset and labor) $\tau_{i,a,z}$, at the old prices such that the new level of utility $v'_i$ is achieved:
\begin{align}
v'_i(a, z ; \ p', R'_{i}, w'_{i}) - v_i(a, z ; \ p, R_{i}\tau_{i,a,z}, w_{i}\tau_{i,a,z})) = 0. \label{eq:welfare-eqv}
\end{align}
This says a household living in country $i$ with states $a,z$ must have their income increased today ( and for the infinite future ) by the number $\tau_{i,a,z}$. The underscore notation indexes this value by the original-type of household I'm looking at, and so there is one number for each type of household with $a,z$ states in country $i$. The $\tau$s that solve (\ref{eq:welfare-eqv}) are my primary measure of welfare at the household level.\footnote{To be clear, there are alternatives like a Lucas-style consumption equivalent variation | this however confronts questions about which consumption. I also explored a lump sum transfer version of \ref{eq:welfare-eqv} as well. The proportional increase measure is my preference because it's essentially the same as that used in \citet{auer2022unequal}.}

\subsection{Quantitative Results: The Welfare Gains Trade}

This section studies the welfare gains from trade in the US under several counterfactual scenarios: (i) a unilateral reduction in trade costs by the US and (ii) a global reduction in trade costs.

The first exercise is a ten percent reduction in all US import trade costs. That is $d_{us,j}$ is shifted down for all $j$ sources by ten percent. Figure \ref{fig:welfare-households} reports the welfare gains (in equivalent variation units) across households. Households are binned by quantiles of the initial distribution of consumption expenditure and the y-axis reports the average gains within each bin. The red bars report the baseline model and the blue bars report the $\log$ preference case when, per Corollary \ref{prp:seperation}, heterogenous price sensitivity does not operate.

\begin{figure}[!t]
\centering{\includegraphics[scale = 0.65]{./figures/ge-welfare-household-vs-log.pdf}}
\caption{US Welfare Gains from a 10\% Reduction in $d_{us,j}$ }\label{fig:welfare-households}
\end{figure}

Figure \ref{fig:welfare-households} clearly shows that gains from trade are strongly pro-poor | across income brackets the poorest households gain four and a half times more from trade than the richest households (2.93 percent vs. 0.65 percent). If one were to convert this to dollar amounts, a US household in the bottom 20th percentile has income around \$20,000. My model implies that a 10 percent reduction in trade costs is equivalent to a permanent transfer of \$600 to the poorest households in the economy at the old prices.

A secondary implication is that the average gains from trade are much larger than representative agent benchmarks. The dashed line in Figure \ref{fig:welfare-households} shows that the average gain is 1.35 percent. In contrast, a calculation of \citet{arkolakis2012new} off of the change in the US import share and an elasticity equal to the average trade-weighted elasticity in the baseline model (see Figure \ref{fig:bilateral-elasticities}) yields a gain of only 0.45 percent. This is a third of the size (1.35 vs. 0.45) of the average gains in my model.

The reason why my model delivers larger average gains is mainly because of the gains in the bottom part of the income distribution. Looking at Figure \ref{fig:welfare-households}, the gains from trade for the rich households are quantitatively similar to those that would come out of my \citet{arkolakis2012new}-style calculation. However, for poor households, the gains are really big. Then when averaging over modest gains for rich households and large gains for poor households, the average gains in my model are three times larger.

The blue bars in Figure \ref{fig:welfare-households} help answer why the gains are pro-poor. These bars are the gains in the $\log$ preference model. Recalling Corollary \ref{prp:seperation}, the gains from substitution in the $\log$ are the same across households. Thus, any heterogenous gains arise from changes in factor prices and potential effects from borrowing constraints. The blue bars in Figure \ref{fig:welfare-households} illustrate is that now the gains from trade are nearly uniform across the distribution.\footnote{The level of gains is lower as well, but this is because trade increases by less in the log model, which I suspect is because GE effects are different across the different models.}

The comparison between the red and blue bars illustrates that heterogeneity in the gains from substitution is the force behind the pro-poor gains from trade. As discussed around equation (\ref{eq:gains-substitution}) there are essentially two issues are at play determining how this aspect of the model leads to heterogenous gains across households: (i) how exposed a household to trade and (ii) how households value a price reduction. The model was calibrated so that exposure was equal across the income distribution, thus (i) is not leading to heterogenous gains from trade.  In contrast, the model was designed and calibrated to match the fact that poor households are very elastic with respect to price and, thus, they strongly value a price reduction. So the second force is the key component driving the pro-poor aspect of these gains from trade.\footnote{For example, I calibrated the model without quality shifters delivering a pattern of micro-level expenditure shares with the poor unexposed to trade similar Figure \ref{fig:micro-trade}. In this case, the equivalent variation gains are flat across the income distribution except for the vary poorest.}

Per the results in Proposition \ref{prp:gains-trade}, what is the role of factor prices? It seems small, but I'll admit that it is hard to separate out in a clean way quantitatively.  In the counterfactual, the ratio of $R/w$ increases by about 4.6 percent and this effect in the $\log$ model is quantitatively similar as well. The reason for the decline is that demand for US goods (and hence labor) falls as households substitute into foreign goods and so US wages fall. The interest rate is essentially globally determined and increases only slightly. The implication is that there is a counteracting pro-rich force behind Figure \ref{fig:welfare-households}. That is for those holding positive assets, these moves in factor prices are beneficial and for debtors these moves hurt. And this is why I suspect the gains are slightly pro-rich in the $\log$ model. But this is very slight, and hence, this is my argument that moves in factor prices are not strongly determining if the gains are pro-rich or poor.

The next exercise is a global reduction in trade costs by ten percent. One virtue of this exercise connects with the previous paragraph. Global trade costs reductions mute changes in factor prices since demand rises for all products and so wages and interest rates don't adjust much. In addition to illustrating the gains from globalization, its also helps isolate the role that factor prices play.

\begin{figure}[!t]
\centering{\includegraphics[scale = 0.65]{./figures/global-welfare-household-vs-log.pdf}}
\caption{US Welfare Gains from a 10\% Reduction in $d$ }\label{fig:welfare-households-global}
\end{figure}

Figure \ref{fig:welfare-households-global} reports the gains from trade. Similar to the previous results the gains are pro-poor and slightly more so with the poorest households gaining a bit more than five times more than the richest households (4.17 percent vs. 0.8 percent). In this sense, globalization is an even stronger force for the poor than a unilateral trade-liberalization that the previous exercise mimics. Because the gains behind globalization are pro-poor, but the gains for the rich are \citet{arkolakis2012new}-like, the average gains across are much larger than representative agent benchmarks. This can be seen with the average gain being about 2 percent for a global reduction in trade costs. The \citet{arkolakis2012new} benchmark is 0.73 percent which is quantitatively similar to the gains for the richest households in the economy of 0.80 percent.

The blue bars in Figure \ref{fig:welfare-households-global} illustrate the gains in the $\log$ preference model. Like the unilateral reduction, the gains are uniformly smaller. Unlike there unilateral reduction, there still is now a pro-poor angle to the gains from trade. This is not about heterogenous gains from substitution, but must be working through changes in factor prices and households moving off their borrowing constraint. As discussed above, because factor prices are not changing that much (less than one percent in both the baseline and $\log$ model), this suggests that the role that the relaxation of the borrowing constraint is a very modest force driving pro-poor gains even absent heterogenous price elasticities.


\section{Conclusion}

This paper developed a model focused on the idea of generating heterogenous price elasticities. From my perspective, a lot of stuff came out of this exercise | how heterogenous elasticities connect with the marginal utility of consumption and in turn shape the gains from trade. How preferences and market incompleteness shape these outcomes. And most surprising is how potent this force is quantitatively with large pro-poor gains from trade. 

This paper does open up many more questions. One that question that I'm pursuing in \citet{waughoptimal} is the socially optimal pattern of trade and what it looks like relative to the pattern of trade we observe. And from this standpoint, the next natural question is if trade policy improve outcomes?  The third question is about the interaction between trade goods and trade in assets. Additively, I probably under-explore this point in this paper. But I think piecing together the trade side and finance side of international economics is an open area ripe for future research.




\appendix

\clearpage
\newpage

\begin{center}
\textbf{\Large Appendix}
\end{center}

\addcontentsline{toc}{section}{Appendices}

\section{The H-A Trade Elasticity}

My definition of the trade elasticity is the partial equilibrium response of imports from $j$ relative to domestic consumption due to a permanent change in trade costs. By partial equilibrium, I mean that wages, interest rates, and the distribution of agents are fixed at their initial equilibrium values. This is consistent with the definition of the trade elasticity in say, \citet{arkolakis2012new} and \citet{simonovska2014elasticity}. By permanent, I mean that the change in trade costs is for the indefinite future and that households correctly understand this. Consistent with this discussion and the notation below, I compute the partial derivatives (not total) of objects with respect to trade costs.

Mathematically, the trade elasticity equals the difference between the elasticities for how trade between $i$ and $j$ change minus how home trade changes:
\begin{align}
\frac{\partial ( M_{ij} / M_{ii} )}{\partial d_{ij}} \times \frac{d_{ij}}{( M_{ij} / M_{ii} )} =& \frac{\partial M_{ij} / M_{ij}}{\partial d_{ij} / d_{ij}}  - \frac{\partial M_{ii} / M_{ii}}{\partial d_{ij} / d_{ij}}.
\label{apx-eq:def_trade_elasticity}
\end{align}
The change in imports between $i$ and $j$ with respect to a change in trade costs is:
\begin{align}
\frac{\partial  M_{ij}}{\partial d_{ij}} = \int_{a,z} \bigg \{\frac{\partial p_{ij}}{\partial d_{ij}} c_{i}(a,z,j) \pi_{ij}(a,z) +  \frac{\partial c_{i}(a,z,j)}{\partial d_{ij}} p_{ij} \pi_{ij}(a,z) +
 \frac{\partial \pi_{ij}(a,z)}{\partial d_{ij}} p_{ij}c_{i}(a,z,j) \bigg \} L_i \lambda_{i}(a,z) da \ dz.
\end{align}
Divide the stuff inside the brackets by household level imports, $p_{ij}c_{i}(a,z,j)\pi_{ij}(a,z)$ and multiply on the outside giving
\begin{align}
\frac{\partial  M_{ij}}{\partial d_{ij}} = \int_{a,z}  \bigg \{ \frac{\partial p_{ij}/p_{ij}}{\partial d_{ij}}  + \frac{\partial c_{i}(a,z,j)/ c_{i}(a,z,j)}{\partial d_{ij}} +
 \frac{\partial \pi_{ij}(a,z) / \pi_{ij}(a,z)}{\partial d_{ij}}  \bigg \} p_{ij}c_{i}(a,z,j)\pi_{ij}(a,z) L_i \lambda_{i}(a,z)da \ dz.
\end{align}
Define the following ``weight'' which is the share of goods that those with states $a,z$ account for in total expenditures from $j$ as
\begin{align}
\omega_{ij}(a,z) = \frac{p_{ij}c_{i}(a,z,j)\pi_{ij}(a,z) L_i \lambda_{i}(a,z)}{M_{ij}},
\end{align}
where the sum of $\omega_{ij}(a,z)$ over states $a,z$ equals one. This gives a nice expression for the import elasticity
\begin{align}
\frac{\partial  M_{ij} / M_{ij}}{\partial d_{ij} / d_{ij}} = 1 + \int_{a,z} \bigg \{ \underbrace{ \frac{\partial c_{i}(a,z,j)/ c_{i}(a,z,j)}{\partial d_{ij} / d_{ij}} }_{\theta_{ij}(a,z)^{I}}+
\underbrace{\frac{\partial \pi_{ij}(a,z) / \pi_{ij}(a,z)}{\partial d_{ij} / d_{ij}} }_{\theta_{ij}(a,z)^{E}} \bigg \} \omega_{ij}(a,z)da \ dz,
\end{align}
or more succinctly as
\begin{align}
\frac{\partial  M_{ij} / M_{ij}}{\partial d_{ij} / d_{ij}} = 1 + \int_{a,z} \bigg \{ \theta_{ij}(a,z)^{I} + \theta_{ij}(a,z)^{E} \bigg \}\omega_{ij}(a,z)da \ dz.
\end{align}
where the elasticity of aggregate imports into $i$ from $j$ is a weighted average of several effects. The value of one out in front arises from the complete pass-through of changes in trade costs to changes in prices. Then the first term within the brackets represent the intensive margin $\theta_{ij}(a,z)^{I}$, so how much do quantities change conditional on choosing to consume variety $j$. The next term $\theta_{ij}(a,z)^{E}$ represents the extensive margin, so how the choice probabilities change.

To complete the derivation, I derive the own-imports term which is similar with
\begin{align}
\frac{\partial  M_{ii}}{\partial d_{ij}} = \int_{a,z} \bigg \{ \underbrace{\frac{\partial p_{ii}}{\partial d_{ij}} c_{i}(a,z,i) \pi_{ii}(a,z)}_{ \ = \ 0} +  \frac{\partial c_{i}(a,z,i)}{\partial d_{ij}} p_{ii} \pi_{ii}(a,z) + \frac{\partial \pi_{ii}(a,z)}{\partial d_{ij}} p_{ii}c_{i}(a,z,i) \bigg \} L_i \lambda_{i}(a,z)da \ dz,
\end{align}
where the first-term is zero because this is a partial equilibrium elasticity. Then after constructing the proper weights and converting everything to elasticity form we have
\begin{align}
\frac{\partial  M_{ii} / M_{ii}}{\partial d_{ij} / d_{ij}} = \int_{a,z} \bigg \{ \theta_{ii,j}(a,z)^{I} + \theta_{ii,j}(a,z)^{E} \bigg \}\omega_{ii}(a,z)da \ dz,
\end{align}
where the $ii, j$ notation means that $\theta_{ii,j}(a,z)^{I}$ reflects how the intensive margin adjusts, conditional on a $ii$ choice, given a change in $ij$ price. Similarly, $\theta_{ii,j}(a,z)^{E}$ represents how the $ii$ choice probability changes given the $ij$ change in price.

Proposition \ref{prp:GET} then follows:

\setcounter{prp}{2}
\begin{prp}[\textbf{The H-A Trade Elasticity}]The trade elasticity between country $i$ and country $j$ is:
{\footnotesize
\begin{align}
\theta_{ij} = 1 + \int_{z,a} \bigg \{ \theta_{ij}(a,z)^{I} + \theta_{ij}(a,z)^{E} \bigg \}\omega_{ij}(a,z)da \ dz - \int_{z,a} \bigg \{ \theta_{ii,j}(a,z)^{I} + \theta_{ii,j}(a,z)^{E} \bigg \}\omega_{ii}(a,z)da \ dz
\label{apx-eq:trade-elasticity}
\end{align}
}which is an expenditure-weighted average of micro-level elasticities. The micro-level elasticities are decomposed into an intensive margin and extensive margin
{\footnotesize
\begin{align}
\nonumber
\theta_{ij}(a,z)^{I} = \frac{\partial c_{i}(a,z,j)/ c_{i}(a,z,j)}{\partial d_{ij} / d_{ij}}, \ \ \ \ \ \ \theta_{ij}(a,z)^{E} = \frac{\partial \pi_{ij}(a,z) / \pi_{ij}(a,z)}{\partial d_{ij} / d_{ij}}, \ \ \ \
\end{align}
}
and the expenditure weights are defined as
{\footnotesize
\begin{align}
\nonumber
\omega_{ij}(a,z) = \frac{p_{ij}c_{i}(a,z,j)\pi_{ij}(a,z) \lambda_{i}(a,z) L_i}{M_{ij}}.
\end{align}
}
\end{prp}

\subsection{Connecting Elasticities with Household Behavior}

To derive the \textbf{intensive margin elasticity}, start from the households budget constraint and differentiate consumption of variety $j$ with respect to price $p_{ij}$ and one gets
\begin{align}
\underbrace{\frac{\partial c_{i}(a,z,j)/ c_{i}(a,z,j)}{\partial d_{ij} / d_{ij}}}_{\theta_{ij}(a,z)^{I}} &= \bigg [-\frac{\partial g_{i}(a,z,j)/ p_{ij}c_{i}(a,z,j)}{\partial p_{ij}/ p_{ij}} - 1 \bigg ],
\label{eq:apx-intensive-margin}
\end{align}
where recall that $g_{i}(a,z,j)$ is the policy function mapping states into asset holdings next period $a'$. To derive the \textbf{extensive margin elasticity}, start from the definition of the choice probability and
\begin{align}
\underbrace{ \frac{\partial \pi_{ij}(a,z) / \pi_{ij}(a,z)}{\partial d_{ij} / d_{ij}} }_{\theta_{ij}(a,z)^{E}} &= \frac{1}{\sigma_{\epsilon}}\frac{\partial v_{i}(a, z, j)}{\partial d_{ij}/d_{ij}} -  \frac{\partial \Phi_{i}(a,z) / \Phi_{i}(a,z)}{\partial d_{ij}/d_{ij}}.
\label{eq:apx-extensive-margin}
\end{align}
I then use the following arguments to unpack how the value function $v_{i}(a, z, j)$ changes:
\begin{align}
\frac{\partial v_{i}(a,z,j)}{\partial d_{ij}/d_{ij}}  =& -u'(c_{i}(a,z,j))c_{i}(a,z,j) + \bigg [ -\frac{u'(c_{i}(a,z,j))}{p_{ij}}\frac{\partial g_{i}(a,z,j)}{\partial p_{ij}/ p_{ij}} \bigg ]  \\
\nonumber \\
&+ \beta \mathrm{E} \bigg \{\frac{\partial v}{\partial a'}\frac{\partial g_{i}(a,z,j)}{\partial p_{ij}/ p_{ij}}\frac{ \partial p_{ij}/ p_{ij}}{\partial d_{ij}/ d_{ij}} +  \frac{\partial v(g_{i}(a,z,j),z')}{\partial p_{ij}/ p_{ij}}\frac{ \partial p_{ij}/ p_{ij}}{\partial d_{ij}/ d_{ij}} \bigg \}
\end{align}
which can then be further expressed in terms of the Euler Equation (derived below in Equation (\ref{eq:apx-euler-equation}))
{\small
\begin{align}
\frac{\partial v_{i}(a,z,j)}{\partial d_{ij}/d_{ij}}  =& -u'(c_{i}(a,z,j))c_{i}(a,z,j) \label{eq:apx-first-term}\\
\nonumber \\
&+ \underbrace{\bigg \{ -\frac{u'(c_{i}(a,z,j))}{p_{ij}} + \beta \mathbb{E} \bigg [ -\sigma_{\epsilon} \frac{\partial \pi_{ii}(a',z') / \pi_{ii}(a',z')}{\partial a'} + u'(c_{i}(a',z',i))R_{i} \bigg ] \bigg \} }_{\mbox{Euler equation in (\ref{eq:apx-euler-equation})}} \frac{\partial g_{i}(a,z,j)}{\partial p_{ij}/ p_{ij}} \\
\nonumber \\
&+  \beta \mathbb{E}\frac{\partial v_{i}(a',z')}{\partial p_{ij}/ p_{ij}} \bigg \}
\end{align}}
The term in the second line is the Euler equation multiplied by how assets change. This term should be zero for small changes. I discuss this more in depth below around the welfare gains calculation, but the argument is that either the Euler equation holds and thus this term is zero, or it does not hold, but then households can't adjust asset holdings and then the outside part is zero. And for small changes households on the margin of a binding constraint or not are on the margin and don't matter.

To add some clarity to this expression assume the number of countries is large. This assumption implies that the $\partial \Phi$ term in (\ref{eq:apx-extensive-margin}) is zero or approximately so. The next implication is that because the ex-ante value function next period $v_{i}(a',z')$ is just a function of $\Phi$ (see its definition in (\ref{eq:big-phi})). Hence, the large number of countries implies future effects don't matter or approximately so. All together using these observations under this assumption in (\ref{eq:apx-extensive-margin}) gives
\begin{align}
\theta_{ij}(a,z)^{E} \approx -\frac{1}{\sigma_{\epsilon}}\bigg[u'(c_{i}(a,z,j))c_{i}(a,z,j)\bigg]. \label{apx-eq:extensive-margin-large}
\end{align}
From here, I can connect (\ref{apx-eq:extensive-margin-large}) with things like relative risk aversion and the marginal propensity to consume. The thought experiment here is ignore all the future effects and ask if a household was a bit wealthier what would the effect be on the $u'(c_{i}(a,z,j))c_{i}(a,z,j)$:
\begin{align}
\frac{\partial (u'(c_{i}(a,z,j))c_{i}(a,z,j))}{\partial a} =& u''(c_{i}(a,z,j))\frac{\partial c_{ij}}{\partial a}c_{i}(a,z,j) + u'(c_{i}(a,z,j))\frac{\partial c_{ij}}{\partial a} \\
\nonumber \\
&= \frac{\partial c_{ij}}{\partial a}\bigg[u''(c_{i}(a,z,j))c_{i}(a,z,j) + u'(c_{i}(a,z,j)) \bigg] \\
\nonumber\\
&= u'(c_{i}(a,z,j))\times \mathbf{MPC}_{ij}(a,z,j) \times \bigg[-\rho_{i}(a,z,j) + 1\bigg]. \label{eq:apx-elasticity-mpc}
\end{align}
And just to emphasize how this works, it's a derivative of $u'(c_{i}(a,z,j))c_{i}(a,z,j)$. So as assets go up, with $\rho > 1$ this implies that $u'(c_{i}(a,z,j))c_{i}(a,z,j)$ goes down. And this is a force for things to be less elastic for rich guys. As assets go down, this implies that $u'(c_{i}(a,z,j))c_{i}(a,z,j)$ goes up, and this is a force for poor guys to be more elastic.

The final elasticity I want to derive is how home choices respond to changes in trade frictions. This is a term that shows up all the time (in the calculations above) and in the welfare expressions, so it's worth computing as well:
\begin{align}
\frac{\partial \pi_{ii}(a,z) / \pi_{ii}(a,z) }{\partial d_{ij} / d_{ij}} = \frac{1}{\sigma_{\epsilon}}\frac{\partial v_{i}(a,z,i)}{\partial d_{ij}/d_{ij}} - \frac{\partial \Phi_{i}(a,z) / \Phi_{i}(a,z)}{\partial d_{ij}/d_{ij}}.
\end{align}
Then the derivative of the term $\Phi$ takes on a unique property where
\begin{align}
\frac{\partial \Phi_{i}(a,z) / \Phi_{i}(a,z)}{\partial d_{ij}/d_{ij}} = \sum_{j} \pi_{ij}(a,z) \frac{1}{\sigma_{\epsilon}}\frac{\partial v_{i}(a,z,j)}{\partial d_{ij}/d_{ij}}
\end{align}
which takes on this flavor of exposure (which are the choice probabilities) times how the household's valuations across the goods change (as represented by the value functions). Then expressing things all relative to how the home valuation changes we have
\begin{align}
\frac{\partial \pi_{ii}(a,z) / \pi_{ii}(a,z) }{\partial d_{ij} / d_{ij}} = \frac{1}{\sigma_{\epsilon}} \sum_{j} \pi_{ij}(a,z) \bigg[ \frac{\partial v_{i}(a,z,i)}{\partial d_{ij}/d_{ij}} - \frac{\partial v_{i}(a,z,j)}{\partial d_{ij}/d_{ij}} \bigg].
\label{eq:apx-change-home-choice}
\end{align}
So what this says is that the change in the home choice completely summarizes how things change (in a relative sense).

\section{The Welfare Gains from Trade}\label{apx-sec:gains-trade}

This section derives the gains from a permanent change in trade costs, across steady states. Like the discussion above, the idea here is that I'm thinking a situation where the change is small and there is an immediate jump to the new steady state. Unlike the trade elasticity, I'm going to take total derivatives encompassing general equilibrium changes in wages and interest rates.

The analysis proceeds in several steps. First, I'll focus on country $i$ and study a change in trade costs $d_{ij}$ with respect to partner $j$. Second, I start by working a utilitarian social welfare function and then drill down into how welfare of an individual household changes.

Social welfare from the perspective of country $i$ is
\begin{align}
W_{i} = \int_{a}\int_{z}  v_{i}(a,z)\lambda_{i}(a,z),
\label{eq:apx-social-welfare}
\end{align}
Then the total change in total welfare is
\begin{align}
\frac{\mathrm{d} W_{i}}{\mathrm{d} d_{ij} / d_{ij}} = \int_{a}\int_{z}  \bigg \{ \frac{\mathrm{d} v_i(a, z)}{\mathrm{d} d_{ij} / d_{ij}}  + v_{i}(a,z) \frac{\mathrm{d} \lambda_{i}(a,z)/ \lambda_{i}(a,z)}{\mathrm{d} d_{ij} / d_{ij}}  \bigg \} \lambda_{i}(a,z).
\label{eq:apx-social-welfare-change}
\end{align}
The first component reflects changes in household-level welfare. The second component is about reallocation, i.e., if|at the old $v$'s|does the distribution change so that social welfare gets better or worse. The change in social welfare is then the weighted average of these two forces with the weights being those at the initial distribution.

How does household-level welfare change? Recall that the value function (with the expectation taken over the different preference shocks) is
\begin{align}
v_i(a, z) =  \sigma_{\epsilon} \log \left\{ \sum_{j'} \exp \left( \frac{  v_{i}(a, z, j')}{\sigma_{\epsilon}} \right) \right\},
\label{eq:apx-epsilon-vfun}
\end{align}
and then I'm going to make the observation that I can substitute out the sum part (\ref{eq:apx-epsilon-vfun}) with the $\exp$ of the home value function relative to the micro-level ``home choice'' so
\begin{align}
\pi_{ii}(a, z) = \exp \left( \frac{ v_{i}(a, z, i) }{\sigma_{\epsilon}} \right) \Bigg / \sum_{j'} \exp \left( \frac{ v_{i}(a, z, j') }{\sigma_{\epsilon}} \right), \\
\nonumber \\
\pi_{ii}(a, z) \times \sum_{j'} \exp \left( \frac{ v_{i}(a, z, j') }{\sigma_{\epsilon}} \right) = \exp \left( \frac{ v_{i}(a, z, i) }{\sigma_{\epsilon}} \right), \\
\nonumber \\
\sum_{j'} \exp \left( \frac{ v_{i}(a, z, j') }{\sigma_{\epsilon}} \right) = \exp \left( \frac{ v_{i}(a, z, i) }{\sigma_{\epsilon}} \right) \Bigg / \pi_{ii}(a, z).
\label{eq:apx-homeshare-vfun}
\end{align}
Then substituting (\ref{eq:apx-homeshare-vfun}) into the value function in (\ref{eq:apx-epsilon-vfun}) gives:
\begin{align}
v_i(a, z) =  \sigma_{\epsilon} \log \left\{ \frac{ \exp \left( \frac{  v_{i}(a, z, i)}{\sigma_{\epsilon}}\right )}{\pi_{ii}(a,z)}  \right\}
\label{eq:apx-homeshare-vfun2}
\end{align}
and recall that the home choice value function that enters into (\ref{eq:apx-homeshare-vfun2}) is
\begin{align}
v_{i}(a, z, i) = u(c_{i}(a,z,i)) + \beta \mathbb{E} v_{i}(g_{i}(a,z,i),z)
\end{align}
where the expectation operator is over the $z$s and the $v_{i}$ is the same value function as in (\ref{eq:apx-epsilon-vfun}) so the taste shocks are integrated out. Taking logs and exp's of the left hand size of (\ref{eq:apx-homeshare-vfun2}) allows for the $v_i$ value function to be represented as
\begin{align}
v_i(a, z) = -\sigma_{\epsilon} \log \pi_{ii}(a,z) + u(c_{i}(a,z,i)) + \beta \mathbb{E} v_{i}(g_{i}(a,z,i),z).
\label{eq:apx-home-valuefun}
\end{align}
Now everything is written with respect to the home choice. What is going on is that the home choice $\pi_{ii}$ summarizes the expected value of those shocks and their benefits. No need to explicitly carry around the $v_{ij}$s. This is essentially the dynamic analog to Equation (15), Footnote 42 of \citet{eaton2002technology} and \citet{arkolakis2012new}.

One more detail, to facilitate interpretation, it will be useful to compute the Euler equation associated with asset holdings when the borrowing constraint does not bind. This euler equation is:
\begin{align}
\frac{u'(c_{i}(a, z, i))}{p_{ii}} = \beta \mathrm{E}_{z'} \left[ -\sigma_{\epsilon} \frac{\partial \pi_{ii}(a',z') / \pi_{ii}(a',z')}{\partial a'} + \frac{u'(c_{i}(a', z', i))R_i}{p_{ii}} \right] \nonumber
\end{align}
This equation is derived below in (\ref{apx-eq:homechoice-euler}).

Now the strategy is to totally differentiate (\ref{eq:apx-home-valuefun}) with respect to trade costs and use the recursive structure to iterate forward and construct the change across time. Totally differentiating the value function gives
{\small
\begin{align}
\frac{\mathrm{d} v_i(a, z)}{\mathrm{d} d_{ij} / d_{ij}} =& \nonumber  \\
\nonumber \\
-\sigma_{\epsilon} & \frac{\mathrm{d} \pi_{ii}(a,z) / \pi_{ii}(a,z)}{\mathrm{d}d_{ij} / d_{ij}}  + u'(c_{i}(a,z,i)) \bigg[ \frac{\mathrm{d} w_{i} / p_{ii}}{\mathrm{d} d_{ij} / d_{ij}}z  +  \frac{\mathrm{d} R_{i} / p_{ii}}{\mathrm{d} d_{ij} / d_{ij}} a  \bigg] \\
\nonumber  \\
& - \frac{u'(c_{i}(a,z,i))}{p_{ii}}\frac{\mathrm{d} g_{i}(a,z,i)}{\mathrm{d} d_{ij} / d_{ij}} + \beta \mathbb{E}_{z'} \frac{\mathrm{d} v_i(g_{i}(a,z,i), z')}{\mathrm{d} d_{ij} / d_{ij}}
\end{align}
}
Then the derivative of the continuation value function is
{\small
\begin{align}
\frac{\mathrm{d} v_i(g(a,z,i), z')}{\mathrm{d} d_{ij} / d_{ij}} = &  \underbrace{\bigg [-\sigma_{\epsilon} \frac{\partial \pi_{ii}(a',z') / \pi_{ii}(a',z')}{\partial a'} + \frac{u'(c_{i}(a',z',i))R_{i}}{p_{ii}} \bigg ]}_{\frac{\partial v_i(g_{i}(a,z,i), z')}{\partial a}}\frac{\mathrm{d} g_{i}(a,z,i)}{\mathrm{d} d_{ij} / d_{ij}} \ \ + \\
\nonumber \\
-\sigma_{\epsilon} & \frac{\mathrm{d} \pi_{ii}(a',z') / \pi_{ii}(a',z')}{\mathrm{d}d_{ij} / d_{ij}}  + u'(c_{i}(a',z',i)) \bigg[ \frac{\mathrm{d} w_{i} / p_{ii}}{\mathrm{d} d_{ij} / d_{ij}}z'  +  \frac{\mathrm{d} R_{i} / p_{ii}}{\mathrm{d} d_{ij} / d_{ij}} a'  \bigg] + \\
\nonumber \\
& - \frac{u'(c_{i}(a',z',i))}{p_{ii}}\frac{\mathrm{d} g_{i}(a',z',i)}{\mathrm{d} d_{ij} / d_{ij}}
+ \beta \mathbb{E}_{z'} \frac{\mathrm{d} v_i(g_{i}(a',z',i), z'')}{\mathrm{d} d_{ij} / d_{ij}}
\end{align}
}
And now collect terms so
{\small
\begin{align}
\frac{\mathrm{d} v_i(a, z)}{\mathrm{d} d_{ij} / d_{ij}} =& \underbrace{-\sigma_{\epsilon} \frac{\mathrm{d} \pi_{ii}(a,z) / \pi_{ii}(a,z)}{\mathrm{d}d_{ij} / d_{ij}}}_{A(a,z)} \\
\nonumber \\
& + \underbrace{u'(c_{i}(a,z,i)) \bigg[ \frac{\mathrm{d} w_{i} / p_{ii}}{\mathrm{d} d_{ij} / d_{ij}}z  +  \frac{\mathrm{d} R_{i} / p_{ii}}{\mathrm{d} d_{ij} / d_{ij}} a  \bigg]}_{B(a,z)}  \\
\nonumber \\
& + \underbrace{\bigg \{- \frac{u'(c_{i}(a,z,i))}{p_{ii}} + \beta \mathbb{E}_{z'} \bigg [-\sigma_{\epsilon} \frac{\partial \pi_{ii}(a',z') / \pi_{ii}(a',z')}{\partial a'} + \frac{u'(c_{i}(a',z',i))R_{i}}{p_{ii}} \bigg ] \bigg \}\frac{\mathrm{d} g_{i}(a,z,i)}{\mathrm{d} d_{ij} / d_{ij}}}_{C(a,z)} \\
\nonumber \\
& + \beta \mathbb{E}_{z'} \bigg \{ -\sigma_{\epsilon} \frac{\mathrm{d} \pi_{ii}(a',z') / \pi_{ii}(a',z')}{\mathrm{d}d_{ij} / d_{ij}} +  u'(c_{i}(a',z',i)) \bigg[ \frac{\mathrm{d} w_{i} / p_{ii}}{\mathrm{d} d_{ij} / d_{ij}}z'  +  \frac{\mathrm{d} R_{i} / p_{ii}}{\mathrm{d} d_{ij} / d_{ij}} a' \bigg] \ \  \ldots
\label{eq:apx-welfare-vterms}
\end{align}
}
Let me walk through the interpretation of each term:
\begin{itemize}
\item[\textbf{A(a,z) -}] The term here is $-\sigma_{\epsilon} \frac{\mathrm{d} \pi_{ii}(a,z) / \pi_{ii}(a,z)}{\mathrm{d}d_{ij} / d_{ij}}$ is a gains from substitution term. I discuss this below, but it summarizes two effects (i) how exposed the household is to market $j$ and an effect from a elasticity term.

\item[\textbf{B(a,z) -}] $\bigg[ \frac{\mathrm{d} w_{i} / p_{ii}}{\mathrm{d} d_{ij} / d_{ij}}z  +  \frac{\mathrm{d} R_{i} / p_{ii}}{\mathrm{d} d_{ij} / d_{ij}} a  \bigg]$ is essentially how a reduction in trade costs affects factor prices | the wage relative to the price of the home good and the interest rate relative to the price of the home good. And these effects are all valued at that household's marginal utility of consumption.

    Two more observations. First, in the perfect competition world I consider $\frac{w_i}{p_{ii}} = A_{i}$ from (\ref{eq:marginal-product}). And thus, $\frac{\mathrm{d} w_{i} / p_{ii}}{\mathrm{d} d_{ij} / d_{ij}} = 0$, so households are perfectly ``hedged'' from any effect on labor earnings.

    Second, there is an effect from how the ``real'' interest rate changes. Here real is in quotes because this is real in units of the home good (and per the observation above, this boils down to the change in the interest rate relative to the wage rate). And because the $a$ can take on positive or negative values, this is a force that could in principal lead to losers from trade.

\item[\textbf{C(a,z) -}]  The third term which I'm labeling as $C(a,z)$ is about changes in asset holdings. For the small / local changes that I'm considering it should zero out, but for larger changes this term should be relevant. Let me expand upon this.

    First, notice that the inside the bracket term is the Euler Equation from (\ref{apx-eq:homechoice-euler}) and its multiplied by the change in policy function.

    The idea is that if the household is unconstrained, then this term is zero as there is no gain through changes in asset behavior. Asset holdings are already chosen optimally so that margins are equated, thus, on the margin any benefit of lower trade costs on changes in asset behavior is zero. Essentially an application of the Envelope Theorem.

    Now in this economy, this term may not be zero because of borrowing constrained households, thus the inside-the-bracket term is positive. However, notice how the outside brackets is multiplied by the change in the asset policy function. What this picks up is that if the household is constrained, then assets can't change so the outside term is zero and, thus, overall the second term is zero.

    Final point, then the only people that benefit and contribute to social welfare through these effects are those on the margin between constrained and not-constrained. But if they are on the margin between being constrained and not-constrained, then they are on their euler equation.

\item The final term is about this continuing on into the infinite future.
\end{itemize}
Iterating on (\ref{eq:apx-welfare-vterms}) into the future, the gains from trade for a household with states $a,z$ today are
\begin{align}
\frac{\partial v_i(a, z)}{\partial d_{ij} / d_{ij}} = \mathbb{E}_{z} \sum_{t = 0}^{\infty} \beta^{t} \bigg \{ -\sigma_{\epsilon} \frac{\mathrm{d} \pi_{ii}(a_{t},z_{t}) / \pi_{ii}(a_{t},z_{t})}{\mathrm{d}d_{ij} / d_{ij}} + B_{t}(a_{t},z_{t}) + C(a_{t},z_{t}) \bigg \}
\label{eq:apx-welfare-v}
\end{align}
Where the first component is the expected discounted gains from substitution, revaluation of asset holdings, and changes in asset holdings. Combining (\ref{eq:apx-welfare-v}) and (\ref{eq:apx-social-welfare-change}) yields the following proposition for the gains from trade.

\begin{prp}[\textbf{The Welfare Gains from Trade}] \label{apx-prp:gains-trade} The welfare gains from trade are given by
{\footnotesize
\begin{align}
\frac{\mathrm{d} W_{i}}{\mathrm{d} d_{ij} / d_{ij}} = \int_{a}\int_{z}  \bigg \{ \frac{\mathrm{d} v_i(a, z)}{\mathrm{d} d_{ij} / d_{ij}}  + v_{i}(a,z) \frac{\mathrm{d} \lambda_{i}(a,z)/ \lambda_{i}(a,z)}{\mathrm{d} d_{ij} / d_{ij}}  \bigg \} \lambda_{i}(a,z).
\nonumber
\end{align}
}which reflects the change in household level gains and how the distribution of households changes. Household level gains are given by
{\footnotesize
\begin{align}
\nonumber
\frac{\partial v_i(a, z)}{\partial d_{ij} / d_{ij}} = \mathbb{E}_{z} \sum_{t = 0}^{\infty} \beta^{t} \bigg \{ -\sigma_{\epsilon} \frac{\mathrm{d} \pi_{ii}(a_{t},z_{t}) / \pi_{ii}(a_{t},z_{t})}{\mathrm{d}d_{ij} / d_{ij}} + B(a_{t},z_{t}) + C(a_{t},z_{t}) \bigg \}
\end{align}
}where each term represents:
\begin{itemize}
\item Gains from substitution: $-\sigma_{\epsilon} \frac{\mathrm{d} \pi_{ii}(a,z) / \pi_{ii}(a,z)}{\mathrm{d}d_{ij} / d_{ij}}$.

\item Gains from asset revaluations: $B(a_{t},z_{t}) = u'(c_{ii}(a_{t},z_{t}))\frac{\mathrm{d} R_{i}/p_{ii}}{\mathrm{d} d_{ij} / d_{ij}}a$

\item Gains from changes in asset holdings:
{\footnotesize
\begin{align}
\nonumber
C(a_{t},z_{t}) = \bigg \{- \frac{u'(c_{i}(a,z,i))}{p_{ii}} + \beta \mathbb{E}_{z'} \bigg [-\sigma_{\epsilon} \frac{\partial \pi_{ii}(a',z') / \pi_{ii}(a',z')}{\partial a'} + \frac{u'(c_{i}(a',z',i))R_{i}}{p_{ii}} \bigg ] \bigg \}\frac{\mathrm{d} g_{i}(a,z,i)}{\mathrm{d} d_{ij} / d_{ij}} = 0
\end{align}}
which is zero for small changes.
\end{itemize}
\end{prp}


The final step is to unpack the gains from substitution term. Now from the elasticity discussion I can convert (\ref{eq:apx-change-home-choice}) into a total derivative form
\begin{align}
\frac{\mathrm{d} \pi_{ii}(a,z) / \pi_{ii}(a,z) }{\mathrm{d} d_{ij} / d_{ij}} = \frac{1}{\sigma_{\epsilon}} \sum_{j'} \pi_{ij'}(a,z) \bigg[ \frac{\mathrm{d} v_{i}(a,z,i)}{\partial d_{ij}/d_{ij}} - \frac{\mathrm{d} v_{i}(a,z,j')}{\mathrm{d} d_{ij}/d_{ij}} \bigg].
\label{apx-eq:homechoice-total}
\end{align}
so change in the home choice summarizes two forces: (i) how exposed a household is to the change through the choice probabilities and then (ii) how value functions change.

Now the value function component is where elasticities enter. Define $\bar{\theta}(a,z) ^E_{ij',j}$ as the extensive margin, cross-price elasticity (how $ij'$ changes with respect to the $j$ change), and in total derivative form (this is what the bar notation denotes). Following the derivation of (\ref{eq:apx-extensive-margin}) this is
\begin{align}
\theta_{ij',j}(a,z)^{E} = \frac{1}{\sigma_{\epsilon}}\frac{\mathrm{d} v_{i}(a, z, j')}{\partial d_{ij}/d_{ij}} -  \frac{\mathrm{d} \Phi_{i}(a,z) / \Phi_{i}(a,z)}{\mathrm{d} d_{ij}/d_{ij}},
\end{align}
which then noticing that the $\mathrm{d} \Phi$ term is independent of option $j'$. This last observation implies that when the cross-price elasticities are substituted into (\ref{apx-eq:homechoice-total}) the $\mathrm{d} \Phi$s difference out. Thus we we can express the change in the home choice in terms of cross-price elasticities
\begin{align}
-\sigma_{\epsilon} \frac{\mathrm{d} \pi_{ii}(a,z) / \pi_{ii}(a,z) }{\mathrm{d} d_{ij} / d_{ij}} = \sigma_{\epsilon} \sum_{j'} \pi_{ij'}(a,z) \bigg[ \bar{\theta}(a,z) ^E_{ii,j} - \bar{\theta}(a,z) ^E_{ij',j}\bigg],
\end{align}
Now let's make the approximation where that all cross-price elasticity terms are zero. Then we have
\begin{align}
-\sigma_{\epsilon} \frac{\mathrm{d} \pi_{ii}(a,z) / \pi_{ii}(a,z) }{\mathrm{d} d_{ij} / d_{ij}} \approx
- \sigma_{\epsilon} \times \pi_{ij}(a,z) \times \bar{\theta}(a,z) ^E_{ij,j}
\end{align}
This expression is interesting because now it is analogous to the gains from trade formula in the efficient allocation. The pure gains from trade component comes from (i) how the taste shock is valued (ii) a households exposure and (iii) the household's elasticity. And this last part, per the arguments above, is about how sensitive the value function is with respect to price.


%\hrulefill
%
%Redo. Digression on chain rule. I'm going to
%\begin{align}
%v(a',z') = v(g(a,z,d), z')
%\end{align}
%where I substitute in the policy function for $a'$. Then first term inside indicates that $v$ depends upon the choice of $a'$ and this works through the policy function. And the dependence of $v$ on $d$ (and not through assets) is implicit. Then the total derivative of this is
%\begin{align}
%\frac{\mathrm{d} v(g(a,z,d), z')}{\mathrm{d} d} =& \frac{\partial v}{\partial a'}\frac{\mathrm{d}g}{\mathrm{d}d} +  \frac{\partial v(\overline{g(a,z,d)},z')}{\partial d}
%\end{align}
%So the first term is the partial change of $v$ with respect to $a'$ times how the policy function totally changes with respect to $d$. The second term is the partial change of $v$ with respect to $d$, \textbf{holding fixed assets} at their chosen level. That's why I'm emphasizing the bar on top. Now what is confusing to me is that this has partial, not total derivatives. But this term is mathematically the same as
%\begin{align}
%\frac{\partial v(\overline{g(a,z,d)},z')}{\partial d} = \frac{\mathrm{d} v(a', z')}{\mathrm{d} d }
%\end{align}
%where the RHS is the total derivative of $v$ treating $a'$ as a parameter. In other words, the LHS says, how does $v$ change (everything else) holding fixed assets. The RHS says how does everything else change holding fixed assets. There the same. And the RHS is the value function evaluated at the new states $a'$, $z'$.
%
%\hrulefill


\section{Gains in the Efficient Allocation}\label{apx-sec:efficient}

This section of the appendix presents abbreviated results from my related paper in \citet{waughoptimal}. Below I discus the planning problem, I state the solution to it, then discuss how I arrive at the gains from trade calculations in Proposition \ref{prp:gains-efficient-allocation}.

I focus on a utilitarian social welfare function:
\begin{align}
W = \sum_{t=0}^{\infty} \sum_{i}  \int\limits_{z} \beta^{t}  v_{i}(z,t) L_{i}\lambda_{i}(z,t),
\label{eq:apx-social-welfare}
\end{align}
and here $v_i$ is a households value function in country $i$ at date $t$. Now, I'm going to place the social welfare function in sequence space and then unpack the benefits from the preference shock in the following way:
\begin{align}
W = \sum_{t=0}^{\infty}  \sum_{i}  \sum_{j}  \int\limits_{z}  \beta^{t} \   \bigg \{  u(c_{i}(z, j, t) ) + \mathrm{E}[ \ \epsilon \ | \ \pi_{ij}(z,t) ] \bigg \}\pi_{ij}(z,t) L_{i} \lambda_{i}(z, t)
\label{eq:apx-social-welfare-2}
\end{align}
so the inner term is period utility given the associated consumption allocation $c_{i}(z, j, t)$ and then the expected value of the preference shock conditional on the choice probability $\pi_{ij}(z,t)$. This inner term is then weighted by the number of households that receive that utility, i.e. the choice probability times the mass of households with shock $z$ at date $t$. The sum across $j$ adds up all households in country $i$. Then the sum across $i$ reflects that this is global welfare.

One more point about the inner term in (\ref{eq:apx-social-welfare-2}), my claim is that with the Type 1 extreme value shocks:
\begin{align}
\mathrm{E}[ \ \epsilon \ | \ \pi_{ij}(z,t) ] = -\sigma_{\epsilon} \log \pi_{ij}(z,t)
\end{align}
where this is like the ``selection correction'' where if $\pi$ becomes smaller, the expected value of the taste shock becomes larger. So only those with the largest relative shocks are chosen and higher utility for those, conditional on being selected, is felt.

Given this formulation, the planner does the following: he chooses consumption and choice probabilities for all country pair combinations, state by state, for the infinite future. The Lagrangian associated with the Planning Problem is:
\begin{align}
\mathcal{L}  = & \sum_{t=0}^{\infty}   \sum_{i} \sum_{j} \int\limits_{z}  \beta^{t} \  \bigg \{  u(c_{i}(z, j, t) ) + \mathrm{E}[ \ \epsilon \ | \ \pi_{ij}(z,t) ] \bigg \}\pi_{ij}(z,t) L_{i} \lambda_{i}(z, t), \label{apx-eq:planner_problem} \\
\nonumber \\
&+ \sum_{t=0}^{\infty} \sum_{i} \beta^{t} \chi_{i}(t) \bigg \{ Y_{it} \  - \ \sum_{j} \int\limits_{z}  d_{ji} c_{j}(z, i, t) \pi_{ji}(z,t) L_{j}\lambda_{j}(z, t) \bigg \} \nonumber \\
\nonumber \\
&+ \sum_{t=0}^{\infty} \sum_{i} \int\limits_{z}  \beta^{t} \chi_{2i}(z,t) \bigg \{1 - \sum_{j}\pi_{ij}(z,t) \bigg \} L_{i} \lambda_{i}(z, t), \nonumber
\end{align}
where the first term is the objective function; the second line is the resource constraint saying that output from country $i$ must equal the consumption of commodity $i$ globally including the transport costs. Then the third line ensures that choice probabilities are probabilities and sum to one. The final thing I'm doing is that I'm scaling the multipliers by $\beta^t$ so that the algebra is easier.

The statement below characterizes the allocation that solves (\ref{apx-eq:planner_problem}):
\begin{prp}[\textbf{The Efficient Allocation}]\label{apx-prp:efficient-allocation} The allocation that satisfies the Centralized Planning Problem in (\ref{apx-eq:planner_problem}) is:
\begin{enumerate}
\item A consumption allocation satisfying:
\begin{align}
 u'(c_{ij}(z,t) ) = \chi_{j}(t) d_{ij}
\end{align}
where $\chi_{j}(t)$ is the shadow price of variety $j$.
\item The choice probabilities are
\begin{align}
\pi_{ij}(t) =\exp \left( \frac{u(c_{ij}(t)) - u'(c_{ij}(t))c_{ij}(t)}{\sigma_{\epsilon}}\right) \bigg / \sum_{j'}\exp \left( \frac{u(c_{ij'}(t)) - u'(c_{ij'}(t))c_{ij'}(t)}{\sigma_{\epsilon}} \right)
\label{apx-eq:planner-choice-prob}
\end{align}
\end{enumerate}
\end{prp}

\textbf{The Gains from Trade.} Given this allocation, I want to compute the social gain to a change in trade costs. First, I express social welfare depending directly upon the trade costs $d$, and then indirectly as the allocations of $c$ and $\pi$s depend upon $d$ as well.
\begin{align}
W(d, c_{i}(j; d), \pi_{ij}(d))
\end{align}
And then totally differentiate social welfare, so
\begin{align}
\frac{\mathrm{d} W}{\mathrm{d}d} = \frac{\partial W}{\partial d} + \frac{\partial W}{\partial c_{i}(j;d)}\frac{\partial c_{i}(j;d)}{\partial d} + \frac{\partial W}{\partial \pi_{ij}(d)}\frac{\partial \pi_{ij}(d)}{\partial d}
\end{align}
and then I invoke the Envelope Theorem. That is I evaluate this derivative at the optimal allocation. But the optimal allocation is optimal, so on the margin any gain from changing consumption or choice probabilities is zero and these indirect effects (at the optimal allocation) are zero. Computing the direct effect gives
\begin{align}
\partial W = - & \sum_{t=0}^{\infty} \beta^{t} \ \chi_{j}(t) c_{i}(j,t) \pi_{ij}(t) L_{i} \partial d_{ij}, \\
\nonumber \\
=& - \sum_{t=0}^{\infty} \beta^{t} \  \ u'(c_{i}(j,t)) c_{i}(j,t) \pi_{ij}(t) L_{i} \partial d_{ij} / d_{ij},
\end{align}
where the first line is how the resource constraint in (\ref{apx-eq:planner_problem}) changes with respect to trade costs. Then the second line inserts the relationship between the multiplier and the marginal utility of consumption.  Breaking it down, this says: $c_{ij}(t) \pi_{ij}(t) L_{i}$ term is how much stuff people in $i$ eat from $j$ and $\partial d_{ij} / d_{ij}$ perturbs it by the percent change in trade costs, then $u'(c_{ij}(t))$ converts it into utils. Imposing stationarity delivers
\begin{align}
\frac{\mathrm{d} W}{\mathrm{d}d_{ij} / d_{ij}} = \frac{\partial W}{\partial d_{ij} / d_{ij}} = -\frac{ u'(c_{i}(j)) c_{i}(j) \pi_{ij} L_{i}}{1- \beta} \label{apx-eq:gains1}
\end{align}

\textbf{The Elasticity of Trade.} Now I compute the trade elasticity in this allocation. I essentially follow the formulas outlined in Proposition \ref{prp:GET}. They apply because they don't depend upon specifics about the environment, just accounting.

Claim \#1: The intensive margin trade elasticity is minus one, i.e. any change in $d_{ij}$ results in a one-for-one increase, $c_{i}(j)$. This follows from the planner directly controlling things and assets are not held or used.

Claim \#2: Next I need to compute the extensive margin elasticity. So I'm going to note that
\begin{align}
\frac{\partial \pi_{ij} / \pi_{ij}}{\partial d_{ij} / d_{ij}} =& \frac{1}{\sigma_{\epsilon}} \bigg [ u'(c_{i}(j,t)\frac{\partial c_{i}(j,t)}{\partial d_{ij}/ d_{ij}} - u''(c_{i}(j,t)\frac{\partial c_{i}(j,t)}{\partial d_{ij}/ d_{ij}}c_{i}(j,t) - u'(c_{i}(j,t)\frac{\partial c_{i}(j,t)}{\partial d_{ij}/ d_{ij}} \bigg] - \frac{\partial \Phi_{i}(t) /\Phi_i(t)}{\partial d_{ij}/ d_{ij}} \\
\nonumber \\
=& -\frac{1}{\sigma_{\epsilon}} \bigg [ u''(c_{i}(j,t)\frac{\partial c_{i}(j,t)}{\partial d_{ij}/ d_{ij}}c_{i}(j,t) \bigg] - \frac{\partial \Phi_{i}(t) /\Phi_i(t)}{\partial d_{ij}/ d_{ij}}
\end{align}
which the first line follows from the quotient rule and where $\Phi_{i}(t)$ is the part of the denominator in the choice probability. Recall the trade elasticity is relative to own trade so
\begin{align}
\frac{\partial \pi_{ii} / \pi_{ii}}{\partial d_{ij} / d_{ij}} = - \frac{\partial \Phi_{i}(t) /\Phi_i(t)}{\partial d_{ij}/ d_{ij}}
\end{align}
Then using my H-A Trade Elasticity formula in Proposition \ref{prp:GET} and canceling terms and noticing as well that the expenditure weights don't matter since they are common across households, I have that:
\begin{align}
\theta_{ij} =& 1 + \left [\theta_{ij}^{I} + \theta_{ij}^{E} \right ]  - \left [ \theta_{ii,j}^{I} + \theta_{ii,j}^{E} \right ]  \\
\nonumber \\
= & 1 + -1 + \frac{-1}{\sigma_{\epsilon}} \bigg [ u''(c_{i}(j,t)\frac{\partial c_{i}(j,t)}{\partial d_{ij}/ d_{ij}}c_{i}(j,t) \bigg] - \frac{\partial \Phi_{i}(t) /\Phi_i(t)}{\partial d_{ij}/ d_{ij}} - 0  - \frac{\partial \Phi_{i}(t) /\Phi_i(t)}{\partial d_{ij}/ d_{ij}} \\
\nonumber \\
= & -\frac{1}{\sigma_{\epsilon}} \bigg [ u''(c_{i}(j,t))\frac{\partial c_{i}(j,t)}{\partial d_{ij}/ d_{ij}}c_{i}(j,t) \bigg].
\end{align}
Then here is a fact I exploit. Starting from the first order condition for consumption and then (i) differentiating both sides with respect to $d_{ij}$ and then multiplying both sides by $d_{ij}$ gives
\begin{align}
u'(c_{i}(j,t) ) = \chi_{j}(t) d_{ij} \Rightarrow \\
\nonumber \\
u''(c_{i}(j,t))\frac{\partial c_{i}(j,t)}{\partial d_{ij} / d_{ij}} = \chi_{j}(t)d_{ij},
\end{align}
which implies that at the optimal allocation
\begin{align}
u'(c_{i}(j,t) ) =  u''(c_{i}(j,t))\frac{\partial c_{i}(j,t)}{\partial d_{ij} / d_{ij}}. \label{apx-eq:muc-fact}
\end{align}
Then the trade elasticity is:
\begin{align}
\theta_{ij}(t) =  -\frac{1}{\sigma_{\epsilon}} \bigg [ u'(c_{i}(j,t)) c_{i}(j,t) \bigg].
\end{align}
where I'll note that the $u'(c)c$ term is marginal utility in semi-elasticity form. So given a percent change in consumption, how much does utility change. Combining the trade elasticity with the gains from trade formula in (\ref{apx-eq:gains1}) gives
\begin{align}
\frac{\partial W}{\partial d_{ij} / d_ij} =  \frac{\sigma_{\epsilon} \ \theta_{ij} \ \pi_{ij} \ L_{i}}{1-\beta} ,
\end{align}
in other words, the gains from trade are how many people are buying $ij$ times the trade elasticity, discounted for the indefinite future. To be clear about signs here, $\theta_{ij}$ is a negative number, all other values are positive. A decline in trade costs means $\partial d_{ij} / d_{ij}$ is negative and hence $\partial W$ is positive and there are gains from trade.

\begin{prp}[\textbf{Trade Elasticities and Welfare Gains in the Efficient Allocation}]\label{apx-prp:gains-efficient-allocation} The elasticity of trade to a change in trade costs between $ij$ in the efficient allocation is:
\begin{align}
\theta_{ij} =  -\frac{1}{\sigma_{\epsilon}} \bigg [ u'(c_{i}(j)) c_{i}(j) \bigg]. \label{apx-eq:eff-trade-elasticity}
\end{align}
And the welfare gains from a reduction in trade costs between $i,j$ are
\begin{align}
\frac{\mathrm{d} W}{\mathrm{d} d_{ij} / d_{ij}} = \frac{\partial W}{\partial d_{ij} / d_{ij}} =  \frac{\sigma_{\epsilon} \ \theta_{ij} \ \pi_{ij} \ L_i}{1-\beta}
\label{apx-eq:eff-trade-gains}
\end{align}
which is the discounted, direct effect from relaxing the aggregate resource constraint.
\end{prp}

As a final step, I connect Proposition \ref{apx-prp:gains-efficient-allocation} with \citet{arkolakis2012new} where the change in the home choice and the dispersion parameter summarize everything. To do, so I first derive the elasticity of home choice probability
\begin{align}
\frac{\partial \pi_{ii} / \pi_{ii}}{\partial d_{ij} / d_{ij}} =& -\frac{\pi_{ij}}{\sigma_{\epsilon}} \bigg \{ u'(c_{i}(j,t))\frac{\partial c_{i}(j,t)}{\partial d_{ij} / d_{ij}} - \bigg [u'(c_{i}(j,t))\frac{\partial c_{i}(j,t)}{\partial d_{ij} / d_{ij}} + u''(c_{i}(j,t))\frac{\partial c_{i}(j,t)}{\partial d_{ij} / d_{ij}}c_{i}(j,t) \bigg ] \bigg \} \\
\nonumber \\
=& \ \ \frac{\pi_{ij}}{\sigma_{\epsilon}}u''(c_{i}(j,t))\frac{\partial c_{i}(j,t)}{\partial d_{ij} / d_{ij}}c_{i}(j,t) \\
\nonumber \\
=& \ \ \frac{\pi_{ij}}{\sigma_{\epsilon}} u'(c_{i}(j)) c_{i}(j) \\
\nonumber \\
=& \ - \theta_{ij} \times \pi_{ij}. \label{apx-eq:change-home-share}
\end{align}
The jump from the second to the third line follows from (\ref{apx-eq:muc-fact}) and the fourth line follows from the definition of the trade elasticity. To be clear on signs here, $-\theta_{ij}$ is positive, the $\pi_{ij}$ is positive. Then a decline in trade costs means $\partial d_{ij} / d_{ij}$ is negative and hence the probability of choosing the home good must decline.  Then inserting (\ref{apx-eq:change-home-share}) into  (\ref{apx-eq:eff-trade-gains}) I have
\begin{align}
\frac{\mathrm{d} W}{\mathrm{d} d_{ij} / d_{ij}} =  -\sigma_{\epsilon} \times \frac{\mathrm{d} \pi_{ii} / \pi_{ii}}{\mathrm{d} d_{ij} / d_{ij}} \times \frac{L_i}{1 - \beta}.
\label{apx-eq:eff-trade-gains-acr}
\end{align}
This says a sufficient statistic for the direct effect of the gains in the efficient allocation is how the home choice probability changes multiplied by the dispersion parameter. Again, regarding the signs here, the change in the home choice probability is positive (declines with decline in trade costs), its multiplied by a negative sign, so welfare goes up with a decline in trade costs.

\section{Log Preferences}\label{apx-sec:log-preferences}

This example is interesting because it retains an aggregate constant trade elasticity and the welfare gains from trade formula looks like ACR kind of thing and the expression in the efficient allocation (\ref{apx-eq:eff-trade-gains-acr}).

\textbf{Step 1: Individual Choices.} With log preferences the $j$ choice value function is
\begin{align}
v_{i}(a, z, j) = &  \max_{\ a' \in \mathcal{A} }\bigg  \{ \log\left (\frac{Ra + wz - a'}{p_{ij}} \right )  + \beta \, \mathbb{E} [v_{i}(a', z')]  \bigg\}
\end{align}
which is then
\begin{align}
v_{i}(a, z, j) = &  \max_{\ a' \in \mathcal{A} }\bigg  \{ \log(Ra + wz - a' )  + \beta \, \mathbb{E} [v_{i}(a', z' )]  \bigg\} - \log p_{ij}
\label{eq:value_fun_option_log_p}
\end{align}
which then leads to the observation that the optimal $a'$ conditional on a choice $j$ is \textbf{independent} of the price and the choice $j$. So what is going on is if you consume an expensive or cheap good, then consumption simply scales up or down so that assets next period are exactly the same. This observation has the implication that expenditures on consumption are the same across choices. Compare households expenditures with the same state $a,z$ but different choices. Equation (\ref{eq:value_fun_option_log_p}) implies
\begin{align}
p_{ij}c_{i}(a,z,j) = p_{ii}c_{i}(a,z,i)
\label{eq:apx-same-spending}
\end{align}
so within states, people always spend the same amount. This observation implies that the choice probabilities are independent of the state only prices matter so
\begin{align}
\pi_{ij}(a, z) = & \exp \left( \frac{ v_{ij}(a, z) }{\sigma_{\epsilon}} \right) \Bigg / \sum_{j'} \exp \left( \frac{ v_{ij'}(a, z ) }{\sigma_{\epsilon}} \right) \\
\nonumber\\
\pi_{ij} = & \exp \left( \frac{  -\log p_{ij} }{\sigma_{\epsilon}} \right) \Bigg / \sum_{j'} \exp \left( \frac{ -\log p_{ij'} }{\sigma_{\epsilon}} \right) \label{apx-eq:shares}
\end{align}
These observations are all consistent with the Generalized Euler Equation below. To see this
\begin{align}
\frac{u'(c_{i}(a, z, j))}{p_{ij}} = \max \left\{ \beta R_{i} \mathbb{E}_{z'} \left[ \sum_{j'} \pi_{ij}(a', z') \frac{u'(c_{i}(a', z', j'))}{p_{ij}} \right] \ , \  u' \left( \frac{R_i a + w_i - \phi_{i}}{p_{ij}} \right) \right \}
\end{align}
and then impose log preferences and notice that
\begin{align}
(Ra + wz - a')^{-1} = \max \left\{ \beta R \mathbb{E} \left[ \sum_{j'} \pi_{ij} (Ra' + wz - a'')^{-1} \right] \ , \   (R a + w - \phi_{i})^{-1} \right \}
\end{align}
and then $\pi_{ij}$'s do not depend upon $a$ or $z$, and then $(Ra' + wz - a'')^{-1}$ not depend upon $j$ either, so simplifying we have
\begin{align}
(Ra + wz - a')^{-1} = \max \bigg \{ \beta R \mathbb{E}_{z'} (Ra' + wz' - a'')^{-1}  \ , \   (R a + w - \phi_{i})^{-1}  \bigg \}.
\end{align}
The variety choice $j$ does not appear at all in this equation, thus the asset choice is independent from the variety choice $j$.

\textbf{Step 2: Micro Trade Elasticities.} Starting with (\ref{eq:apx-intensive-margin}) and because the asset choice is independent of prices, the intensive margin elasticity $\theta_{ij}(a,z)^I$ is -1 and $\theta_{ii,j}(a,z)^I$ is zero as there are no partial effects on prices in $ii$.

The extensive margin elasticity is:
\begin{align}
\theta_{ij}(a,z)^E =& \frac{1}{\sigma_{\epsilon}}\frac{\partial v_{ij}(a,z)}{\partial d_{ij}/d_{ij}} -  \frac{\partial \Phi_{i} / \Phi_{i}}{\partial d_{ij}/d_{ij}}\\
\nonumber \\
=& -\frac{1}{\sigma_{\epsilon}}\frac{\partial p_{ij} / p_{ij}}{\partial d_{ij}/d_{ij}} + \beta \mathbb{E} \frac{\partial v_{i}(a',z')}{\partial d_{ij}/d_{ij}} -  \frac{\partial \Phi_{i} / \Phi_{i}}{\partial d_{ij}/d_{ij}} \\
\nonumber \\
=& -\frac{1}{\sigma_{\epsilon}} + \beta \mathbb{E} \frac{\partial v_{i}(a',z')}{\partial d_{ij}/d_{ij}} -  \frac{\partial \Phi_{i} / \Phi_{i}}{\partial d_{ij}/d_{ij}}
\label{eq:apx-log-partial-valuefun}
\end{align}
where the first line removes the $a,z$ indexing of $\Phi_i$ because only prices matter for choice probabilities, not state variables of the household. The next line then partially differentiates the value function with respect to the change in trade costs and I'm exploiting how with log preferences one can pull out the price term. And then the final line notes that the price elasticity is minus one. One more fact that:
\begin{align}
\theta_{ii,j}(a,z)^E =&  \beta \mathbb{E} \frac{\partial v_{i}(a',z')}{\partial d_{ij}/d_{ij}} -  \frac{\partial \Phi_{i} / \Phi_{i}}{\partial d_{ij}/d_{ij}}
\end{align}
where a key thing to notice is that the $ii,j$ elasticity is the same as the second and third terms above in (\ref{eq:apx-log-partial-valuefun}).


\textbf{Step 3: Expenditure Weights.} Recall that the micro level trade elasticities when aggregated are weighted by
\begin{align}
\omega_{ij}(a,z) = \frac{p_{ij}c_{ij}(a,z)\pi_{ij}(a,z) \lambda_{i}(a,z)}{M_{ij}}.
\end{align}
and I can relabel $p_{ij}c_{ij}(a,z) = x_{i}(a,z)$ given (\ref{eq:apx-same-spending}), that expenditures are independent of the source. With the choice probabilities independent of $a,z$ the weights become
\begin{align}
\omega_{ij}(a,z) =& \frac{x_{i}(a,z)\pi_{ij} \lambda_{i}(a,z)}{\int_{z}\int_{a}x_{i}(a,z)\pi_{ij} \lambda_{i}(a,z)da \ dz}, \\
\nonumber \\
=& \frac{x_{i}(a,z) \lambda_{i}(a,z)}{\int_{z}\int_{a} x_{i}(a,z) \lambda_{i}(a,z)da \ dz}
\end{align}
which is independent of source $j$.

\textbf{Step 4: The Aggregate Trade Elasticity.} Now mechanically follow Proposition \ref{prp:GET}:
\begin{align}
\nonumber
\theta_{ij} =& 1 + \int_{z}\int_{a} \bigg \{ -1 +  -\frac{1}{\sigma_{\epsilon}} + \beta \mathbb{E} \frac{\partial v_{i}(a',z')}{\partial d_{ij}/d_{ij}} -  \frac{\partial \Phi_{i} / \Phi_{i}}{\partial d_{ij}/d_{ij}} \  \bigg \}\omega_{i}(a,z)da \ dz \\
\nonumber \\
& - \int_{z}\int_{a} \bigg \{   \beta \mathbb{E} \frac{\partial v_{i}(a',z')}{\partial d_{ij}/d_{ij}} -  \frac{\partial \Phi_{i} / \Phi_{i}}{\partial d_{ij}/d_{ij}}  \bigg \}\omega_{i}(a,z)da \ dz \\
\nonumber \\
= & -\frac{1}{\sigma_{\epsilon}} \nonumber
\end{align}
where the last line follows because the $a,z$ terms in the micro level trade elasticities exactly cancel given that expenditure weights are source independent. And the aggregate trade elasticity is constant and parameterized by the dispersion in tastes.

\textbf{Step 5: Gravity.} Following the arguments that expenditures are independent of the source, bilateral imports are
\begin{align}
M_{ij} = \pi_{ij} \int_{z}\int_{a} x_{i}(a,z) \lambda_{i}(a,z)da \ dz
\end{align}
where the last term does not depend upon the source. Dividing by home consumption, using (\ref{apx-eq:shares}), and substituting in prices with technology and wages we have
\begin{align}
\frac{M_{ij}}{M_{ii}} = \left( \frac{  w_{j} / A_{j} }{  w_{i} / A_{i} } \right)^{\frac{-1}{\sigma_{\epsilon}}} d_{ij}^{\frac{-1}{\sigma_{\epsilon}}}
\end{align}
which is the same form as in a Armington model or \citet{eaton2002technology}.

\textbf{Step 5: The Grains From Trade.} Then from here I can just follow Proposition \ref{apx-prp:gains-trade}. The individual gains are
{\footnotesize
\begin{align}
\nonumber
\frac{\partial v_i(a, z)}{\partial d_{ij} / d_{ij}} = \underbrace{\frac{1}{\theta (1-\beta)} \times \frac{\mathrm{d} \pi_{ii} / \pi_{ii}}{\mathrm{d}d_{ij} / d_{ij}}}_{ACR} \ \ + \ \
\mathbb{E} \sum_{t = 0}^{\infty} \beta^{t} \bigg \{ B(a_{t},z_{t}) + C(a_{t},z_{t}) \bigg \}
\end{align}
}where the first term is exactly what would arise in the static, representative agent model except for the discounting bit. What facilitates this is that the choice probabilities are independent of $a,z$ and it can be pulled out of the expected discounted sum stuff.

\begin{corr}[\textbf{Separation of Trade and Heterogeneity}] In the dynamic, heterogenous agent trade model where preferences are logarithmic over the physical commodity: The trade elasticity is
\begin{align}
\theta = -\frac{1}{\sigma_{\epsilon}}, \nonumber
\end{align}
and trade flows satisfy a standard gravity relationship
\begin{align}
\frac{M_{ij}}{M_{ii}} = \left( \frac{  w_{j} / A_{j} }{  w_{i} / A_{i} } \right)^{\frac{-1}{\sigma_{\epsilon}}} d_{ij}^{\frac{-1}{\sigma_{\epsilon}}}, \nonumber
\end{align}
and both are independent of the household heterogeneity. And the welfare gains from trade for an individual household are
\begin{align}
\nonumber
\frac{\partial v_i(a, z)}{\partial d_{ij} / d_{ij}} = \underbrace{\frac{1}{\theta (1-\beta)} \times \frac{\mathrm{d} \pi_{ii} / \pi_{ii}}{\mathrm{d}d_{ij} / d_{ij}}}_{\mbox{ACR}} \ \ + \ \
\mathbb{E} \sum_{t = 0}^{\infty} \beta^{t} \bigg \{ B(a_{t},z_{t}) + C(a_{t},z_{t}) \bigg \}
\end{align}
where the gains from substitution are (i) independent of the household heterogeneity and (ii) summarized by the trade elasticity and the change in the home choice probability and the other sources of gains are as in Proposition \ref{apx-prp:gains-trade}.
\end{corr}

\section{Appendix: The Euler Equation and Endogenous Grid Method}

First, I'm going to derive the Euler equation for this model when the household is away from it's borrowing limit. Focus on the within a variety choice component, the households value function can be written as:
\begin{align}
v_{ij}(a, z) = \max_{a'} u \left( \frac{R_i a + w_i z - a'}{p_{ij}} \right) + \beta  \mathrm{E} v(a', z')
\end{align}
then the first order condition associated with this problem is:
\begin{align}
\frac{u'(c_{i}(a, z, j))}{p_{ij}} = \beta \mathrm{E} \frac{\partial v(a', z')}{\partial a'}
\end{align}
which says that, conditional on a variety choice the left hand side is the loss in consumption units which is $1 / p_{ij}$ evaluated at the marginal utility of consumption and this is set equal to the marginal gain from saving a bit more which is how the value function changes with respect to asset holdings. Now we can arrive at the $\frac{\partial v(a', z')}{\partial a'}$ in the following way, so start from the log-sum expression for the expected value function
\begin{align}
\mathbb{E}_{\epsilon} v(a', z') =  \sigma_{\epsilon} \log \left\{ \sum_{j'} \exp \left( \frac{  v_{i}(a', z', j')}{\sigma_{\epsilon}} \right) \right\},
\end{align}
and then differentiate this with respect to asset holdings
\begin{align}
\frac{\partial \mathbb{E}_{\epsilon} v(a', z')}{\partial a'} = \left( \frac{\sigma_{\epsilon}}{\sum_{j'} \exp \left( \frac{  v_{i}(a', z', j')}{\sigma_{\epsilon}}\right)} \right)
\left[ \sum_{j'} \exp \left( \frac{  v_{i}(a', z', j')}{\sigma_{\epsilon}}\right) \frac{1}{\sigma_{\epsilon}} \frac{\partial v_{i}(a', z', j')}{\partial a'}  \right].
\end{align}
Then if you look at this carefully and notice how the choice probabilities are embedded in here, I have
\begin{align}
\frac{\partial \mathbb{E}_{\epsilon} v(a', z')}{\partial a'} = \sum_{j'} \pi_{ij}(a', z) \frac{\partial v_{i}(a', z', j')}{\partial a'},
\label{apx-eq:expected-value-fun-partial}
\end{align}
and then apply the Envelop theorem to the value functions associated with the discrete choices across the options
\begin{align}
\frac{\partial \mathbb{E}_{\epsilon} v(a', z')}{\partial a'} = \sum_{j'} \pi_{ij}(a', z') \frac{u'(c_{i}(a', z', j'))R_{i}}{p_{ij}},
\end{align}
So then putting everything together we have
\begin{align}
\frac{u'(c_{i}(a, z, j))}{p_{ij}} = \beta R_{i} \mathrm{E}_{z'} \left[ \sum_{j'} \pi_{ij'}(a', z') \frac{u'(c_{i}(a', z',j'))}{p_{ij'}} \right],
\label{apx-eq:euler}
\end{align}
where this has a very natural form: set the marginal utility of consumption today equal to the marginal utility of consumption tomorrow adjusted by the return on delaying consumption, and the expected value of the marginal utility of consumption reflects how the uncertainty over both ones' preference over different varieties and shocks to efficiency units. The final step is the generalized version that incorporates the fact that some households are constrained
\begin{align}
\frac{u'(c_{i}(a, z, j))}{p_{ij}} = \max \left\{ \beta R_{i} \mathrm{E}_{z'} \left[ \sum_{j'} \pi_{ij'}(a', z') \frac{u'(c_{i}(a', z', j'))}{p_{ij}} \right] \ , \  u' \left( \frac{R_i a + w_i - \phi_{i}}{p_{ij}} \right) \right \}
\label{eq:apx-euler-equation}
\end{align}
To arrive at the representation of the Euler Equation in home choice form, I make the following observations. As mentioned above, the elasticity of the home choice probability with respect to a change in assets is
\begin{align}
\frac{\partial \pi_{ii}(a,z) / \pi_{ii}(a,z) }{\partial a} = \frac{1}{\sigma_{\epsilon}}\frac{\partial v_{i}(a,z,i)}{\partial a} - \frac{\partial \Phi_{i}(a,z) / \Phi_{i}(a,z)}{\partial a}.
\label{apx-eq:dpi-da}
\end{align}
And then the change in the $\Phi$ index is
\begin{align}
\frac{\partial \Phi_{i}(a,z) / \Phi_{i}(a,z)}{\partial a} = \sum_{j} \pi_{ij}(a,z) \frac{1}{\sigma_{\epsilon}}\frac{\partial v_{i}(a,z,j)}{\partial a}
\label{apx-eq:dphi-da}
\end{align}
And now notice how this is connected with how the value function changes with respect to assets above. Specifically, inserting (\ref{apx-eq:dphi-da}) into (\ref{apx-eq:dpi-da}) and rearranging we have
\begin{align}
-\sigma_{\epsilon} \frac{\partial \pi_{ii}(a,z) / \pi_{ii}(a,z) }{\partial a} + \frac{\partial v_{i}(a,z,i)}{\partial a} =
\sum_{j} \pi_{ij}(a,z) \frac{\partial v_{i}(a,z,j)}{\partial a}
\end{align}
and from here we can insert the expression above into \ref{apx-eq:expected-value-fun-partial} and apply the envelope theorem giving
\begin{align}
\frac{u'(c_{i}(a, z, j))}{p_{ij}} = \beta \mathrm{E}_{z'} \left[ -\sigma_{\epsilon} \frac{\partial \pi_{ii}(a',z') / \pi_{ii}(a',z')}{\partial a'} + \frac{u'(c_{i}(a', z', i))R_i}{p_{ii}} \right]
\label{apx-eq:homechoice-euler}
\end{align}
which \textbf{any} variety choice on the left hand side must respect the $ii$ representation on the right hand side. Note the way the home choice probability enters here, it has a flavor like an asset / shock specific price index.

\subsection{EGM-Discrete Choice Algorithm}

My computational approach exploits the Euler Equation derived above. Below, I describe my algorithm. This focuses on just the consumer side in one country $i$.
\begin{itemize}
\item[\textbf{0.}] Set up an asset grid. Then guess (i) a consumption function $g_{c,i}(a,z,j)$ for each $a$, $z$, and product choice $j$ and (ii) choice specific value function $v_{i}(a,z,j)$.

\item[\textbf{1.}] Compute the choice probabilities from (\ref{eq:choice-prob}) for each $(a,z)$ combination, given the guessed value functions.

\item[\textbf{2.}] Given the consumption function and choice probabilities compute the RHS of (\ref{apx-eq:euler}) first.

\item[\textbf{3.}] Then invert to find the new updated consumption choice so
\begin{align}
c_{i}(\tilde a, z, j) = u^{' -1}\left\{ p_{ij} \max \left\{ \beta R_{i} \mathrm{E}_{z'} \left[ \sum_{j'} \pi_{ij'}(a', z') \frac{u'(c_{i}(a', z',j'))}{p_{ij'}} \right] \ , \  u' \left( \frac{R_i a + w_i - \phi_{i}}{p_{ij'}} \right) \right \} \right \}
\end{align}
where $u^{' -1}$ is the inverse function of the marginal utility of consumption.

\item[\textbf{4.}] The key issue in this method is that we have found  $c_{i}(\tilde a, z, j)$ where the consumption function is associated with some asset level that is not necessarily on the grid. The solution is to (i) use the budget constraint and infer $\tilde a$ given that $a'$ was chosen above (that's where we started), $z$, and $c_{i}(\tilde a, z, j)$. Now we have a map from $\tilde a$ to $a'$ for which one can use interpolation to infer the $a'$ chosen given $a$ where $a$ is on the grid.

\item Do steps \textbf{3.} and \textbf{4.} for each $j$ variety choice. This then makes the function $g_{i}(a,z,j)$ mapping each state and $j$ choice (today) into $a', z'$ states and then from the budget constraint we have an associated consumption function $g_{c,i}(a,z,j)$.

\item[\textbf{5.}] Compute the $\mathrm{E}\left[ v(g_{a,ij}(a,z), z') \right]$. This is performed in the {\tt{make\_Tv.!}} function. It fixes a country $j$, then works through shocks and asset states today and from the policy function $g_{a,ij}(a,z)$ figures out the asset choice tomorrow. Then the $\mathrm{E}\left[ v(g_{a,ij}(a,z), z') \right]$ is (\ref{eq:log_sum}) over the different variety choices tomorrow (this is the integration over $\epsilon$) multiplied by the probability of $z'$ occurring (this is the integration over $z$). This step and the next step is a key difference relative to the traditional approaches using Euler Equations. Here, I need to reconstruct the value function to construct choice probabilities; in traditional approaches the value function is not a required object.

\item[\textbf{6.}] Given \textbf{4.} update the value function using the bellman equation evaluated at the optimal policies:
\begin{align}
Tv_{i}(a, z, j) = u(g_{c,i}(a,z,j)) + \beta \mathrm{E}\left[ v(g_{a,i}(a,z), z', j') \right]
\end{align}

\item[\textbf{7.}] Compare old and new policy functions, old and new value functions, and then update accordingly.
\end{itemize}

\section{Appendix: Quality Version of the Model}\label{apx-sec:quality}

To match micro-level expenditure shares, I introduce are household-specific quality shifters. Mechanically, I implement quality shifters in the following way. Utility associated with the choice of variety $j$ is
\begin{align}
u(c_{ijt}) + \psi_{j} + \epsilon_{jt}, \label{apx-eq:utility-quality}
\end{align}
now there is a shifter $\psi_{j}$ in utility that depends upon the commodity $j$ chosen. Now I'm going to make the assumption that the quality valuation of a household varies with it's and efficiency units. In particular, the assumption will be something along the lines that
\begin{align}
\psi(z, j)
\end{align}
So what this means is that a household, depending upon its situation, may have different valuations for a particular commodity. Then, given this assumption on quality, now we are back to the case where the state variables of a individual household are its asset holdings and efficiency units.

Then I'm going to write the value function of a household in country $i$, after the variety shocks are realized, as
\begin{align}
v_{i}(a, z) = &  \max_{j} \big  \{ \  v_{i}(a, z, j) + \psi(z, j) + \epsilon_{j} \ \big \}
\end{align}
And here, I've pulled out the quality term and the shock term to be more consistent with the code. Specifically, solution methods will work on the $v_{i}(a, z, j)$s and then reconstruct $v_{i}(a, z)$ given the shocks and quality specification. The value function conditional on a choice of variety is
\begin{align}
v_{i}(a, z, j) = &  \max_{\ a' \ }\bigg  \{ u(c_{ij}) + \beta \, \mathbb{E} [v_{i}(a', z')]  \bigg\}\\
\nonumber \\
\mbox{subject to}  \ & (\ref{eq:borrowing-constraint}) \  \mathrm{and} \ (\ref{eq:trade-budget-constraint}). \nonumber
\end{align}
Associated with this are the following choice probabilities for each differentiated good:
\begin{align}
\pi_{ij}(a, z) = \exp \left( \frac{ v_{i}(a, z, j) + \psi(z, j) }{\sigma_{\epsilon}} \right) \Bigg / \Phi_{i}(a,z), \\
\nonumber \\
\mbox{where} \ \ \ \Phi_{i}(a,z) := \sum_{j'} \exp \left( \frac{ v_{i}(a, z, j') + \psi(z, j') }{\sigma_{\epsilon}} \right).
\end{align}
And then the expectation of (\ref{eq:valuefun}) with respect to the taste shocks takes the familiar log-sum form
\begin{align}
v_i(a, z) = \sigma_{\epsilon} \log \left\{ \Phi_{i}(a,z)  \right\}.
\end{align}
Or the equivalent representation of this which I'm now using in the code is
\begin{align}
v_i(a, z) = \sum_{j'} \pi_{ij'}(a, z) \bigg[ v_{i}(a, z, j') + \psi(z, j') - \sigma_{\epsilon} \log (\pi_{ij'}(a, z))  \bigg]
\end{align}
how is the true again? Then there is an Euler Equation for each variety choice $j$. This (I believe) takes a different form so
\begin{align}
\frac{u'(c_{i}(a, z, j))}{p_{ij}} = \beta R_{i} \mathrm{E}_{z'} \left[ \sum_{j'} \pi_{ij'}(a', z') \bigg( \ \frac{u'(c_{i}(a', z', j'))}{p_{ij'}} \bigg) \right].
\end{align}
So fundamentally, nothing really changed by introducing quality shifters.

Now to reduce the dimensionality of these parameters, I set this up as a home bias term $\psi_{i}(z,i)$ which takes on some number and for all other $j$s this is where $\psi_{i}(z,i) = 0$. Then assume that it's a log-linear function of a households permanent productivity state and this function is the same across countries. Slightly abusing notation, this is where $\psi_{i}(z,i) = \tilde \psi \times z$. The slope of this relationship is calibrated to match the fact from \citet{borusyak2021distributional}) that import expenditure shares are essentially the same between US poor (below median income) and rich (above median income) households.

%%
%%\subsection{Appendix: Equivalent Variation Measures}
%%
%%One question: How would this work in ACR world? I'll do this with CRRA and ignore household types. The value function of a household at base period prices is
%%\begin{align}
%%v_i\left ( \frac{w_{i}}{P_{i}} \right).
%%\end{align}
%%And then the value function, but at counterfactual prices
%%\begin{align}
%%v'_i\left( \frac{w'_{i}}{P'_{i}} \right),
%%\end{align}
%%then equivalent variation finds the $\tau$ at the old prices which equate value functions. So with CRRA and infinitely lived households this becomes
%%\begin{align}
%%\frac{\left( \frac{w_{i}}{P_{i}} \tau \right )^{1-\gamma}}{(1 - \beta)(1-\gamma)} = \frac{ \left( \frac{w'_{i}}{P'_{i}}  \right)^{1-\gamma}}{(1 - \beta)(1-\gamma)}
%%\end{align}
%%Then I'm going to insert the result from \citet{arkolakis2012new} that the real wage is proportional to the home trade share
%%\begin{align}
%%\frac{w_{i}}{P_{i}} \propto \pi_{ii}^{\frac{-1}{\theta}}.
%%\end{align}
%%Then inserting this into the expression above and canceling terms gives
%%\begin{align}
%%\tau^{1-\gamma} &= \left( \pi_{ii}^{\prime} \right)^{\frac{1-\gamma}{-\theta}} /  \left( \pi_{ii} \right)^{\frac{1-\gamma}{-\theta}}, \\
%%\nonumber \\
%%\Rightarrow \ \ \  \tau & = \left( \pi_{ii}^{\prime}  /   \pi_{ii} \right)^{\frac{-1}{\theta}}.
%%\end{align}
%%So my equivalent variation measure in an ACR world is the change in the home trade shares taken to the power of one over the trade elasticity. And this is the classic formula, thus adding curvature to the utility function, infinitely lived vs. static does not matter for computing equivalent variation.
%%
%%\newpage
%%
%%\section{Appendix: Endogenous Grid Method}
%%
%%\begin{table}[t]
%%\small
%%\begin{center}
%%\refstepcounter{table}
%%\setlength {\tabcolsep}{5.75mm}
%%\renewcommand{\arraystretch}{1.60}\label{tb-welfare}
%%\begin{tabular}[t]{l c c c}
%%\multicolumn{4}{c}{{\normalsize\textbf{Table \ref{tb-welfare}: 10\% Reduction in US Trade Costs: Interest Rates, Trade, Welfare}} }
%%\\ \hline \hline
%%& 1.5 & 1.65 &  \\
%%0.250 & 4.24 \ 5.80 \ 8.13 &  & \\
%%0.285 & 3.87 \ 5.38 \ 7.51 &  & \\
%%0.30 & 3.73 \ 5.22 \ 7.29  & 3.74 \ 5.82 \ 8.93 & \\
%%0.34 & 3.43 \ 4.88 \ 6.80  & 3.47 \ 5.52 \ 8.45 & \\
%%0.36 & 3.33 \ 4.75 \ 6.60  & 3.36 \ 5.34 \ 8.26 & \\
%%0.39 & 3.15 \ 4.53 \ 6.34  & 3.22 \ 5.14 \ 8.02 & \\
%%\hline
%%\end{tabular}
%%\\[0.5ex]
%%\parbox{5.75in}{\footnotesize \textbf{Note:} Targets are \ 4.4, \ 6.6}
%%\end{center}
%%\end{table}


% Connect with close economy stuff...evidence on prices


%because  within their respective literatures.
%
%
%
%
%%\footnote{Separate from this is computational challenge. Finding an equilibrium requires (i) solving for households dynamic problems|in each country (ii) constructing the stationary distribution of expenditure patterns and wealth|in each country (iii) aggregating and then (iv) finding a vector of prices so goods markets and financial markets clear world wide. Plot twist: It's doable. My github repository and the Appendix provides a complete description of my approach and methods.}
%
%
%then allows me to think explicitly about normative issues
%
%
%As a I show, market incompleteness is not a sufficient condition to




%Lit review:
%\begin{itemize}
%  \item FK paper, about heterogeneity in shares. This is about elasticities. They say model is pro-poor, but the issue is about sector. This is a model of within catagory/sector. Relates to Cravino Levchncho.
%
% \item Cravino Levchenko is about have a within discussion, they just don't see shares. But what they do see (in the is that high income guys buy high priced stuff). Then when constructing the price index, they assume this in some way. Simmilar to Krill. Key is to distingusih, income on price within varity, vs. income and shares (imported or not) and where they come from.

%  \item Note my formulas have similar flavor to fixed share approach. Planner is share times change is welfare gain. Need to look at krill paper.
%
%\item Krill paper... they do find that within category, rich buy more imported stuff than poor. The issue is complicated as you go across categories and then you do not get much action. The KN data set goes a deep dive and finds this as well. China is strange, strong case for thinking about qualit see Page 18 second to last paragraph. There you see the Armington quirk, imported stuff is more expensive, and rich guys are buying imported stuff. Other thing about Krill paper is critique of FK paper...AIDS generates a mechanical pro-poor bias. Non-homothetic CES is flexible to generate pattern in the data, FK is not. Final point, the Krill paper derives the share approach, this turns out to be exactly the same as the gains from trade in planner allocation
%
%\item Torston Jaccard does something simmilar to Krill, Auer. China messe stuff up, Europe strong pro rich exposure. All personal beuty products. Simmilar stuff to Auer. Proably more relavent for work with Simon is the stuff abour urban vs. rural and how this is above and beyond income, suggests something up about retailer. Does nested logit.
%
%\item The hotman paper is strange. This is mostly across sectors and claim they find something like Krill. No elasticity effect. Second, the Layspayers gives a totally different estimate than the model (inflation is high for poor vs rich) Why? Not really told that. The inflation results are suggestive that rich benefited more...but not quite obvious. Also not sure what the role of the firm is.
%
%\item Walsh paper...looks similar to my welfare calculations. Builds on the Moll paper too. Interesting is how they compute money metric by computing change in welfare relative to change in muc. Should put in labor income channel.
%
%\item Moll paper is very slick. Again more related than Walsh. One thing they do is put things in a money metric units by dividing through by muc in period 0. In the efficient economy, this then is back to Krill's formula. This money metric formula is why things are discounted by $R$ in Walsh...because the euler equation implies the ratio of muc's are equal to $R$. The other trick about this is that when assets are in zero net supply, the aggregate welfare gain in money metric units is zero. Key is that it's in money metric units though, in utils space there is true redistribution (worth talking about in either finacial globalization space or finacial autarky world).
%
%\item Is kind of related to Auclart...this the interest rate exposure channel. Less than I thought. As the key thing is more about dc. Tobin quote is nice about heterogenaity in muc. Everything is short term assets, so like in the moll paper sales = stocks. And then some people have different unhedge exposures to changes in interest rates.
%\end{itemize}


%This pattern of elasticities is consistent with recent evidence focusing on expenditure switching at the household level. \citet*{auer2022unequal} explore the heterogenous impact of the 2015 Swiss Franc appreciation and find that the exchange rate shock lead to larger changes in expenditure shares for poor households relative to the rich. In other words, poor households are more sensitive to price than rich households. Outside of the trade literature, there is broader set of evidence in support of the idea that price sensitivity declines with income; e.g. \citet{sangani2022markups} provides evidence in support of this fact using Nielsen Homescan data.
%%Discuss this evidence better. Faber Falley, Dellevingo,


\newpage

\newpage

\bibliography{./bibtex/micro_price_bibtex}

\end{onehalfspacing}

\end{document} 